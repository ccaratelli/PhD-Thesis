\graphicspath{{figures/}}
% Header
\renewcommand\evenpagerightmark{{\scshape\small Conclusions and perspectives}} 
\renewcommand\oddpageleftmark{{\scshape\small Chapter 4}}

%\renewcommand{\bibname}{References}

\hyphenation{}

\chapter[Conclusions and perspectives]%
{Conclusions and Outlook}
\label{ch4}
The role of active sites in catalytic reactions taking place in nanoporous
materials, the many environmental factors affecting the structure of the
active sites, the role of guest species on the catalytic performance, are
all scientific questions which may be answered by molecular modeling. This has
been shown in this thesis on a variety of case studies, which have each been performed in close synergy with various experimental partners.
\npar
The main material of interest in this thesis was the Zr--based UiO--66
material, which is regarded as one of the most stable MOFs available today. Its
stability can be traced back to the high order of connectivity of the ligands to
the Zr atoms. Inspired by this concept, new generation of materials have
been proposed in the previous years. The material has been explored for a
variety of applications. Within this thesis, focus was set on the catalytic applications, inspired by conceptual advances made by our experimental partners to engineer UiO--66 for specific catalytic applications. To activate the material for catalysis, it had to be subjected to an activation procedure, moreover introduction of structural defects showed a large potential to tune the material for dedicated applications.
Inspired by this context, the research of this PhD was performed to obtain a
deep understanding at the nanoscale level into the active sites and reaction
mechanisms. Within the framework of this PhD thesis we particularly studied the
activation processes for the UiO--66 material. Furthermore, the material can be
modified postsynthetically.  
Postsynthetic modifications applied to MOFs allow tuning their physical and
chemical properties by framework alteration after it has been synthesized. Very often the
inorganic and organic components of the material can be modified or removed without
compromising the overall structure stability, introducing active sites for
catalysis. The mechanisms behind these transformations are yet not well
understood, though they are crucial in understanding the catalytic behavior of the materials. Advanced molecular modeling techniques have been used
to gain a deeper understanding in the nature of the active sites and structural
deformations during postsynthetic activation processes. These processes,
realized at elevated temperature, result in removal of species coordinated to
the inorganic brick, drastically influencing the catalytic activity of UiO--66
as has been demonstrated in \textbf{Paper I}. At an elevated temperature, above
523 K, the material can also dehydrate reversibly and for this process a
detailed mechanistic pathway was unraveled. The dehydration mechanism may have a decisive effect on
certain catalytic reactions, where next to the Lewis acid site also the
neighboring Br\o{}nsted base or acid site may take a cooperative role in the
chemical transformation.
\npar
In \textbf{Paper III}, by following the structural changes in UiO--66, during an
activated process at 573 K, some remarkable fast dynamics of the system were
observed. Major rearrangements of the hydroxyl groups connected to the inorganic
brick induce mobility of various linkers, which can easily decoordinate and
recoordinate, rotate and translate. In lower activated regions, the hydroxyl
group could easily travel among the three Zr atoms to which it is originally
connected in a \ce{{\textmu}3}--OH configuration. Once a higher activated
plateau was reached at about 90 kJ/mol, reversible mobility of the linkers was
observed, while the rigidity of the inorganic brick was retained. These motions
were accompanied by remarkable processes of proton shuttling between a
\ce{{\textmu}3}--OH and carboxylic oxygen of a partly decoordinated linker,
which we showed to occur even in defect--free UiO--66 at elevated temperature.
These results expand the earlier concept of dynamic acidity in the defective
UiO--66 to defect--free material where guest protic species are not present.
Such proton shuttles may also be facilitated by the presence of other guest species
in the pores, such as methanol. A reduction of coordination number of the Zr
atoms was observed due to linker decoordination which is important for
activation processes. Such processes include dehydration but also postsynthetic
ligand and cation exchange, which have recently been realized experimentally for very
robust materials like UiO--66. The study performed within the framework of this
thesis on the dehydration mechanism is very promising for future studies as we
showed how molecular modeling can be used to follow \textit{in situ} activation
processes of the material.  Such processes are hard to track on an experimental basis and in this respect molecular modeling proves to be a true added value for the nanoscopic understanding of the material.
\npar
The molecular level characterization of defects in MOFs is extremely challenging
from both a theoretical and an experimental viewpoint. In \textbf{Paper II} a
plethora of defective structures with a different number and position of missing linkers
was generated, for which the free energy of defect formation and mechanical
stability was elucidated.
\npar
For UiO--66, recent discoveries have revealed that after defect engineering the
material possesses not only Lewis acid sites but also a substantial amount of
Br\o{}nsted sites which show a remarkable chemical behavior in the form of
dynamic acidity and may take an interactive role in reactions where proton
transfers are necessary. Various coordination configurations of water and charge balancing hydroxide ions in the place of missing linkers were studied in \textbf{Paper II} and \textbf{Paper VII}. By performing \textit{ab initio} MD simulations, proton shuttles between water molecules physisorbed to the metals and coordinated charge balancing hydroxide anions were observed.
\npar
A deep understanding of the strength of Lewis and Br\o{}nsted sites within MOFs
is very important to tune the catalytic activity for a broad variety of acid,
basic or acid--basic catalyzed reactions. In \textbf{Paper IV}, we used the
Oppenauer oxidation reaction of primary alcohols to rationalize the catalytic
effect of the structurally incorporated defects including linker removal and
hydration/dehydration of the framework. It has been unraveled that the catalytic
activity is spread out over the \ce{Zr-O-Zr} site with a Lewis acid center at
the coordinatively unsaturated Zr--atoms and a Br\o{}nsted site on the oxygen.
The presence of acid and basic centers within molecular distances has been shown to be crucial in the performance of the catalytic reaction as they cooperate in a concerted way during the chemical transformations. Hydrated bricks have stronger basic sites and facilitated protonation steps in dual catalyzed reactions instead, in subsequent deprotonation steps the opposite behavior was observed.
It is clear that there exists a subtle interplay between the Lewis acid and the
Br\o{}nsted base sites on UiO--66 and that the basicity of the material can be
tuned to a large extent by the postsythetic modifications.
\npar
In addition to these findings a key role in dual catalyzed reaction mechanisms
was seen for the aldol condensation reaction (\textbf{Paper V}) and the Fisher
esterification (\textbf{Paper VII}). Furthermore, postsynthetic ligand exchange
leads to functionalized materials in which the host--guest interactions can be
enhanced. In both works a passive role of amino-modulating UiO-66 resulted in a
higher catalytic performance of UiO--66(\ce{NH2}) which was observed both
theoretically and experimentally. The profound effect of functionalization was also discovered for the adsorption of DMMP in the framework cavity in (\textbf{Paper X}).
\npar
Initially, catalysis on MOFs explored mainly reactions that were catalyzed
by Lewis acid sites, this thesis gave a deep understanding on reactions where
next to the Lewis acid site also Br\o{}nsted sites are involved in the best
performance of the catalyst.
The availability of Lewis acid sites in the fully coordinated UiO--66 material
can be obtained by creation of defects. One of the earliest examples showing
this concept of catalytically active coordinatively unsaturated Zr atoms was
shown for the citronellal cyclization.
However, other chemical transformations investigated in this thesis entail
reaction mechanism where next to the Lewis acid activity, Br\o{}nsted sites
play a crucial role in proton transfer steps. Such proton transfers
can be facilitated by the presence of other protic guest molecules, which was
shown for methanol solvent in \textbf{Paper VIII}. This is of high importance as
many heterogeneous catalytic reactions are performed in solution, where the
solvent can have a notable position in facilitating or hindering the reaction
due to its interactions with the catalyst and the solute. 
\npar
Recently other stable Zr materials have been devised which are now
gradually being tested for catalysis (Figure
\ref{fig:MOFstopology}, Chapter 1). The results obtained within this PhD
thesis indicate that for exploration of catalytic properties of other Zr--based MOFs of UiO--66 type it is essential to understand their
intrinsic dynamic properties during activation processes such as dehydration and postsynthetic modification.
The findings regarding the inherent dynamic flexibility of UiO--66 have opened a
perspective towards a comprehensive molecular level understanding of the solvent
assisted cation exchange processes in robust MOFs. The methods used in this
thesis open a lot of perspectives to be used on a broader set of materials for
defect engineering and various catalytic applications.

