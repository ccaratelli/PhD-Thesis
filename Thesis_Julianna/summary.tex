\chapter{Samenvatting}
Nanoporeuze functionele materialen vormen een belangrijke klasse van vaste stof materialen die enorm veel interesse hebben opgewekt
in de chemie en materiaal wetenschappen in de afgelopen decennia. Het belang van nanoporeuze materialen kan niet onderschat worden 
ook vanuit een industrieel standpunt. Sommige materialen zijn overal aanwezig
in industri\"ele toepassingen.
\npar
Binnen de familie van de nanoporeuze materialen zijn zeolieten die opgebouwd
zijn uit puur anorganische atomen vandaag de dag omnipresent in tal van
industri\"ele toepassingen. Hun uitzonderlijke eigenschappen zijn ondermeer toe te schrijven aan de aanwezigheid van uniforme kanalen en caviteiten, 
waarin actieve sites kunnen worden gegenereerd voor katalyse. Door de aanwezigheid 
van adsorptiekanalen met moleculaire dimensies zijn ze ook zeer aantrekkelijk voor tal 
van selectieve adsorptietoepassingen. Dit scala van bijzondere eigenschappen
ligt aan de grondslag voor de grote verscheidenheid aan applicaties binnen de petrochemie, 
katalyse en milieutechnologie. De relatief smalle pori\"en, die zeer typisch
zijn voor zeolieten, zorgen er echter voor dat enkel relatief kleine moleculen kunnen 
diffunderen door de kanalen, waardoor een hypotheek gelegd wordt op sommige toepassingen. 
Verder zijn er beperkingen in het aanbrengen van functionaliteiten in zeolieten waardoor deze 
materialen moeilijk te wijzigen zijn voor specifieke toepassingen die men beoogt. Het totaal aantal 
zeolieten gesynthetiseerd vandaag de dag bedraagt iets meer dan 200, volgens de zeoliet databank.
In de afgelopen decennia, werden verschillende families van nieuwe nanoporeuze materialen voorgesteld. 
Zo werden onder meer metaal--organische roosters (MOFs) ontwikkeld. MOFs zijn
hybride materialen opgebouwd uit organische en anorganische fragmenten. Ze
kunnen worden gesynthetiseerd met aanwezigheid van een uniforme poriestructuur. MOFs vertonen een 
unieke bouwstructuur waarbij organische en anorganische bouwstenen met elkaar verbonden zijn via 
coordinatie bindingen en aldus drie--dimensionele roosters vormen. Dit uniek
bouwconcept ligt mede aan de basis voor de grote interesse die MOFs hebben opgewekt in de 
wetenschappelijke gemeenschap, gezien in principe een zeer groot aantal van
mogelijke combinaties kunnen gevormd worden met potenti\"ele bouwblokken. 
Op die manier kan een groot scala van materialen worden gesynthetiseerd met
verschillende chemische compositie, topologie en porie--volume.
Gecombineerde theoretische en experimentele studies hebben duidelijk gemaakt dat
MOFs in vele gevallen voorzien zijn van structureel ingebouwde defecten, die in grote 
mate de eigenschappen van het materiaal kunnen be\"invloeden voor toepassingen
zoals katalyse en adsorptie. Zoals zal worden aangetoond in dit proefschrift vormen defecten een essentiele schakel
in de activatie van materialen voor katalyse. Dit doctoraatsproefschrift concentreert zich vooral op de studie van de katalytische eigenschappen van MOFs.
\npar
De speciale textuur, het poreus kristallijn netwerk en het hoog metaalgehalte maken MOFs uitermate geschikt voor katalytische doeleinden. 
Er werd ook aangetoond dat deze materialen bovendien kunnen gemodificeerd worden teneinde hun katalytische functionaliteit voor bepaalde
reacties substantieel te verhogen. Een zwak punt van vele initieel voorgestelde MOFs was hun beperkte chemische, thermische en mechanische stabiliteit 
in vergelijking met zeolieten. Echter binnen het groot aantal reeds gesynthetiseerde materialen werden verschillende MOFs voorgesteld die wel degelijk zeer 
stabiel zijn voor tal van toepassingen. De zirconium gebaseerde materialen die in deze thesis worden bestudeerd vertonen een dergelijk hoge stabiliteit. 
Het ontwerp van een ideale katalysator is een belangrijk streefdoel voor tal van toepassingen. Ook het ontrafelen van reactiemechanismen, bepalen van de 
chemische kinetiek, reactieproduct selectiviteit en andere grootheden,
blij\-ven hoog in de lijst van prioriteiten van de chemicus.
Hier is een belangrijke taak weggelegd voor moleculaire modellering. In dit proefschrift werden geavanceerde modelleringstechnieken gebruikt om inzicht 
te bekomen in de gangbare reactiemechanismen en actieve sites bij realistische
werkingscondities. Experimentele karakterisatie van katalysatoren is het eenvoudigst als de katalytische site ge\"isoleerd is. Bij MOFs is dit meestal
niet het geval. De actieve sites zijn zeer complex en zijn zeer onderhevig aan heersende defecten zoals aanwezigheid
van solvent moleculen en structurele wanorde. Experimentele data geven in vele gevallen een gemiddelde waarneming. 
Computationele technieken laten toe om het gedrag van individuele actieve sites te bestuderen. 
Een directe vergelijking met experiment is niet altijd evident, gezien experimentele metingen in vele gevallen opgemengd zijn met aspecten,
die niet of moeilijk in rekening kunnen gebracht worden in een computationele berekening. Ondanks de schijnbare kloof tussen experiment en theorie,
wordt in dit proefschrift aangetoond dat een complementaire benadering van modelleringstechnieken en experimenteel werk, 
aanleiding kan geven tot verregaande inzichten in het katalytisch proces. Dit zal aangetoond worden bij middel van verschillende chemische reacties opgenomen in dit proefschrift.
\npar
Moleculaire modellering is vandaag uitgegroeid tot een bijzonder krachtig hulpmiddel om reactiemechanismen te begrijpen, te ontrafelen en in 
sommige gevallen kan het geheel van modelleringstechnieken zelfs aanleiding geven tot een beter ontwerp van de katalysator. 
Modelleren van metaal--organische roosters is bijzonder uitdagend door de
grootte van de te bestuderen systemen, de diversiteit in mogelijke bouwblokken en de complexe natuur van de actieve sites. 
Verder kunnen verschillende transitiemetalen die worden opgenomen in de anorganische fragmenten een multiconfigurationeel gedrag vertonen waardoor 
geavanceerde elektronische structuurmethoden nodig zijn. Mede door de bijzondere
grote toename van computationele kracht maar ook door de ontwikkeling van steeds meer ingenieuze computationele technieken, zijn we echter in staat om vandaag de dag materialen en chemische transformaties te bestuderen 
met een grote graad van complexiteit.
\npar
Het onderzoek voorgesteld in dit proefschrift, concentreert zich voornamelijk op de beschrijving van adsorptie en katalytische eigenschappen van MOFs en meer 
bepaald van het UiO--66 materiaal. Het laatste materiaal heeft bijzonder veel
aandacht gekregen de afgelopen jaren, gezien het bijzonder stabiel is onder tal
van omstandigheden en op een gemakkelijke manier gemodificeerd kan worden.
Een belangrijke vraagstelling betreft de karakterisatie van de actieve sites in het materiaal. 
De rol van defecten mag hierbij niet onderschat worden en in die zin werden in
deze thesis structuur--activiteitsrelaties bepaald voor verschillende types van
chemische conversies. Verder werd bijzondere aandacht besteed aan het tot stand komen van verschillende defecten.
\npar
Alle simulaties verricht in dit werk zijn gebaseerd op ab initio methoden
steu\-nende op de Dichtheidsfunctionaal theorie (DFT).
Ze zijn er inderdaad op gericht om reacties te beschrijven waarbij chemische bindingen worden gevormd en gebroken. 
Zowel statische als dynamische methoden worden aangewend om de kinetische kinetiek van elementaire reactiestappen te beschrijven. 
Binnen stati\-sche methoden worden aan de hand van analyse van een beperkt
aantal punten op het potentieel energie oppervlak, uitspraken gedaan over het verloop van de reactie. Statische methoden hebben het voordeel dat geavanceerde elektronische 
structuurmethoden gebruikt kunnen worden. Echter het mogelijke bestaan van
verschillende transitietoestanden en aspecten zoals conformationele vrijdom worden niet in rekening gebracht. In de beginfase van het doctoraat werden eindige clusters beschouwd die 
uitgesneden werden uit het periodiek rooster. Dergelijke aanpak was destijds nog heel populair en vooral gedreven door computationele beperkingen. 
Meer recentelijk wordt meer en meer overgeschakeld naar periodieke berekeningen, dankzij de gestage toename van computationele rekenkracht. 
Met periodieke randvoorwaarden wordt de moleculaire omgeving van de actieve site en reactanten op een natuurlijke wijze in rekening in rekening gebracht. 
In eerste instantie werden actieve sites gecre\"eerd door wegnemen van een of
meerdere tereftalaat linkers waardoor de Zr--centers co\"ordinatief niet meer
gesatureerd zijn en optreden als actieve Lewis zure sites. Bovendien worden deze sites ook meer toegankelijk voor reactanten. 
Linker functionalisatie leidt tot een bijkomende complexiteit met creatie van
additionele Lewis en Br\o{}nsted sites. Ook solventen, zoals water, hebben een
grote invloed op de katalytische activiteit. Dit in kaart brengen vergt een structurele 
aanpak en dit is precies wat in de thesis geprobeerd wordt te doen. 
\npar
Het is evident dat simulaties van reacties bij operationele condities van
temperatuur en druk eerder een dynamische aanpak vergen dan een statische.
Het vrij--energie oppervlak is zeer complex met diverse metastabiele toestanden.
Het kan enkel voldoende ge\"exploreerd worden door middel van moleculare
dynamica (MD) technieken. De meeste reacties zijn geactiveerd, en zonder
hulptechnieken geraakt men niet over de barri\`ere binnen een haalbare
simulatietijd. Binnen de context van deze thesis werden geavanceerde moleculaire
dynamicatechnieken, zoals metadynamica (MTD) en umbrella sampling (US) veelvuldig aangewend waardoor
vrije--energie profielen kunnen worden opgesteld met bepaling van vrije energie
barri\`eres bij experimentele condities. Door de aanwezigheid van
co\"ordinatieve bindingen tussen de metaal ionaire clusters en de organische linkers zijn de MOF roosters flexibeler dan origineel gedacht. 
In dit proefschrift werd met behulp van geavanceerde moleculaire dynamica technieken inzicht gegeven in de flexibiliteit van de metaal organische binding. 
De resultaten in dit proefschrift tonen aan dat de liganden van de MOF tijdelijk
kunnen deco\"ordineren en elders bindingen kunnen aangaan. Het is een zeer
dynamisch geheel met proton mobiliteit doorheen de metaal ionencluster, waardoor overal in het kristal tijdelijk 
waterstofbruggen kunnen gevormd worden die dan weer verdwijnen. Bij hoge temperaturen wijzen simulaties ook op het optreden 
van een intrinsieke dynamische aciditeit waarbij een proton hopt van een hydroxyl groep naar een linker. Proton mobiliteit zoals 
hier waargenomen in de MD simulaties heeft een enorme impact op de katalytische
activiteit van het UiO--66 materiaal (zelfs zonder defecten) met een Lewis zure
site en aanwezigheid van dynamische Br\o{}nsted sites.
\npar
Binnen dit proefschrift
werd verder evidentie gegeven voor de complexe natuur van de actieve site. De
duale Lewis--Br\o{}nsted katalytische activiteit van UiO--66 werd onderzocht bij
verschillende reacties, zoals de aldol condensatie, Oppenauer oxydatie en Fischer esterificatie, in samenwerking met experimentele partners. 
Telkens werd een goede overeenkomst bekomen met de experimentele data. Dit laat toe soms interpretaties te geven op puur theoretische basis 
aan fenomenen die experimenteel moeilijk te verklaren zijn. Bij de bovenvermelde reacties werd ook vastgesteld dat de katalytische activiteit 
indirect wordt gefaciliteerd door het optreden van dynamische aciditeit in
UiO--66 aangezien vele reacties doorgaan in oplossing, waarbij het solvent een
niet verwaarloosbare rol kan spelen in het faciliteren van reacties of deze verhinderen 
door interacties tussen de katalysator en de oplossing. Bij reactie condities
zijn de pori\"en van de MOF meestal gevuld met solvent die een actieve of
passieve rol kunnen spelen tijdens de katalytische reactie. De diverse reacties beschreven in de thesis en hun resultaten, 
tonen aan dat simulaties op de moleculaire schaal niet meer weg te denken zijn om chemische transformaties op de nanoschaal te ontleden. 
De resultaten wijzen ook op het belang van het samengaan van de theoreticus met de experimentator en optimaal gebruik te maken van de synergie tussen beiden. 
Verdere uitdagingen binnen dit vakgebied bestaan erin om de kloof tussen theoretische berekeningen en experimentele waarnemingen verder te dichten. 
Computationele methoden starten vanuit een bottom--up benadering terwijl
experimenteel een top--down benadering wordt gevolgd om systematisch beter
inzicht te krijgen in de fenomenen op de kleinste schaal. Het overbruggen van inzichten bekomen uit beide vakgebieden is bijzonder
vruchtbaar en er wordt verwacht dat nieuwe inzichten verder zullen worden
bekomen door samengaan van theoretische en experimentele studies.

\chapter{Summary}
Nanoporous functional materials are a class of solid--state materials which have
become an object of intense study in chemistry and materials science in the last decades. They can be found in different fields in industry and their importance in many processes today is indisputable.
\npar
One of the first nanoporous materials which were discovered are zeolites, which
are purely inorganic frameworks and have a broad range of applications. The
uniform channels and cavities, active sites with non--identical strength, high
adsorption capacity are responsible for their successful applications in
petrochemistry, catalysis and environmental technologies. Nevertheless, the
rather small pore size limits their field of applications to smaller molecules,
which can easily fit in their cavities. Moreover, their limited functionality
and atomic composition restricts their tunability to target specific reactions.
Today the number of available zeolites amount to more than two hundred types
according to the zeolite database. In the last two decades, new families of
porous materials have entered the scene. One of these classes are
metal--organic frameworks (MOFs), which are truly hybrid materials where organic
and inorganic moieties are interconnected through coordinative bonds to form
tridimensional porous frameworks. The attention they have received can be traced
back to their unique building concept based on metal ions or clusters linked
together by multitopic linkers, which yields in principle an infinite number of
frameworks with varying chemical composition, topology, surface area and pore
volume.
Moreover, some parts of the building components, like linkers, metals or nodes could be missing, and it was quickly realized that these imperfections could dramatically influence the properties of the materials, by creating active sites for catalysis and adsorption.
\npar
The exceptional textural properties, porous crystalline network and high metal
contents make MOFs very appealing in the field of catalysis. Moreover, their
outstanding tunable properties allow MOFs to be engineered to maximize the
catalytic effect for a given reaction. The coordinative metal--ligand bond which connects nodes and linkers allows to postsynthetically modify the structures and finetune their properties, for instance by functionalizing linkers and nodes. Post-synthetic modifications open the door to a precise tailoring and design of MOFs for specific applications. Unlike zeolites though, MOFs often lack stability and what could in theory be the perfect material for a given application could not be stable at reaction conditions. The search for an optimal catalyst is at the heart of any chemical process and understanding the kinetics of heterogeneous catalysis is essential to establish product conversion rates and to unravel the underlying reaction mechanisms. The molecular level insight into the behavior of these materials at the nanoscale and knowledge on the active site is of utmost importance to predict their catalytic performance.
\npar
In theory, the experimental characterization of catalysts is the easiest when
the materials possess well--defined isolated sites. Unfortunately, in MOFs, the
active sites are complex and mostly arise from defects and structural disorder.
This makes it impossible to directly pinpoint them in the material, as what
is obtained with experimental measures is an average of the whole system.
Computational modeling has become a crucial component to understand the catalytic function and to validate certain experimental observations. Provided modeling is performed in a way it is representative for true reaction conditions, it can also be used in a predictive manner. The case studies considered in this thesis show that an integrated approach in a close collaboration between experimentalists and theoreticians is necessary to understand the often complex molecular behavior.
\npar
Over the last decades, molecular modeling has developed as a key tool to
support, rationalize, guide experimental efforts and even make predictions for catalyst design. Modeling of MOFs represent a great challenge for computational chemists due to the large sizes of the elementary building units, the complexity of the catalytic active sites, and the intrinsically multiconfigurational character of the electronic configurations of many transition metals. However, computational power has grown exponentially, providing new opportunities to explore larger systems and accurately describe chemical transformations at the nanoscale.
\npar
The research work described in this PhD thesis specifically focuses on
\textit{ab initio} simulations to study the adsorption phenomena and catalytic
reactions on UiO--66, which among the plethora of known frameworks is one of the
most widely investigated Zr--based MOF due to its exceptional stability. A major
fundamental unresolved challenge lies in the characterization of the active
sites present in the material. Special attention was paid to understanding the structure--activity relation by unraveling the nature of the active sites which are strongly connected to the presence of defects in UiO--66. The different mechanisms that lead to creation of active sites and the inherent dynamic behavior of the material upon activation processes were investigated. The chemical nature of these active sites and catalytic properties of the material were further explored by modeling differently catalyzed reactions.
\npar
In this thesis, various molecular modeling techniques were employed to elucidate
different chemical transformations. Static Density Functional Theory (DFT)
enables an accurate description of the chemical kinetics of elementary reaction
steps. A proper description of adsorbed reactants in the pores requires the use
of a MOF model that represents the specific characteristics of the catalyst
including the confinement effect. For the considered reactions, it was possible
to explain the reasons behind experimentally observed features based on cluster
and periodic simulations in combination with transition state theory. The
reaction mechanisms were successfully unraveled with a focus to explain the function of various possible active sites on the material, the influence of water and linker functionalization. Most of the reactions on UiO--66 rely on the Lewis acidity of the coordinatively unsaturated Zr--centers created by removing terephthalate linkers. In many cases the chemical functionality which arises from unsaturated metal sites is not limited to Lewis acidity. Species which are present in the solution can coordinate to these Zr atoms introducing additional Br\o{}nsted sites, which might have a decisive role on different catalytic processes.
\npar
When simulating reactions at true operating conditions of temperature and pressure, the applied static methods to investigate chemical transformations are often too limited. The free energy surface is often very complex and includes many isoenergetic configurations which can be accurately described by applying molecular dynamics techniques. The best accuracy of molecular simulations is obtained when the conditions inside a real catalyst are mimicked as closely as possible. 
\npar
Due to the presence of coordinative bonds, the MOF frameworks are more flexible
than originally anticipated. Even though inert UiO--66 is characterized by an
exceptional stability and rigidity, upon activation processes, the linkers can
easily decoordinate and even exchange with other ligands. Moreover, they can be
stabilized by hydrogen bonds with protons from the nodes. The dynamic acidity of
UiO--66, which was observed at elevated temperature can have a direct and
indirect impact on the catalytic performance of this material. Within this thesis we observed an intrinsic dynamic acidity with rapid proton transfers between hydroxyl groups and linkers, during activation processes. Proton mobility can facilitate various catalytic reactions where next to a Lewis acid site also dynamic Br\o{}nsted sites are required.
The dual Lewis--Br\o{}nsted catalytic activity of UiO--66 was investigated for
different reactions, such as aldol condensation,
Oppenauer oxidation and Fischer esterification, in collaboration with experimental partners. In all these cases, a good agreement with the experimental results was found, allowing to explain observed phenomena which were not easy to interpret on a purely experimental basis.
Moreover, catalytic activity might be indirectly facilitated by the dynamic
acidic behavior of UiO--66 observed here, as many chemical reactions are performed in solution, where the solvent can play a notable role in facilitating or hindering the reaction due to its interactions with the catalyst and the solute. At reaction conditions, the pores of MOFs are often filled with solvent which can serve as a substrate during catalytic reactions or modulate the nature of active sites.
\npar
Apart from the explicit catalytic reactions studied in this thesis, we also
explored the possibility to study the dynamic behavior of the metal--ligand bond
in the UiO--66 material. To this end advanced sampling molecular dynamics techniques were used which enable to simulate activated processes. More in particular we studied the changes in coordination number of the hydroxyl group bridging the Zr atoms of the inorganic brick. The lowering of coordination number occurs typically during the dehydration process, but also in other processes in which defects are formed. Major rearrangements of the hydroxyl groups connected to the inorganic brick, induce mobility of various linkers, which can easily decoordinate and recoordinate, rotate and translate. Overall the results show that the UiO--66 material has a high degree of internal flexibility, which allows to accommodate substantial changes in coordination numbers with serious deformation of linkers and other labile groups, but without compromising its structural integrity.
\npar
The different aspects tackled in this PhD thesis clearly demonstrate the power
of molecular simulations to accurately model and understand chemical
transformations at the nanoscale. These complex processes need a joint
computational and experimental effort, where a bridge needs to be established
between experimental observations and insights at the nanoscale level. 
Such complementary approach may give many answers to unresolved issues but
sometimes is the basis of more intriguing questions to be answered. 
The journey to designing MOFs and modeling catalysis and activation processes
\textit{in situ} has just started thanks to development of advanced molecular
dynamics techniques which allow to sample the system at operating conditions. A multilevel modeling approach, where models are built from experimental data by combining diverse simulation techniques will be the driving force in the study of these complex and beautiful systems.

