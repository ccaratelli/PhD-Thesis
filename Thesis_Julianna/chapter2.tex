\graphicspath{{figures/}}
% Header
\renewcommand\evenpagerightmark{{\scshape\small Modeling MOF catalyzed
reactions}} 
\renewcommand\oddpageleftmark{{\scshape\small Chapter 2}}

%\renewcommand{\bibname}{References}

\hyphenation{}

\chapter[Modeling MOF and zeolite catalyzed reactions]%
{Modeling MOF and zeolite catalyzed reactions}
\label{ch2}
Molecular modeling of MOF and zeolite catalyzed reactions is a field of its
own due to the complex nature of the processes, which combines concepts from
organic, inorganic, material physics and physical chemistry. Theoretical modeling of heterogeneous catalysis takes an indispensable role in
understanding the nature of the active sites or unraveling the reaction
mechanisms of complex chemical transformations, which are hard to track on
purely experimental basis. The steady methodological developments and the
availability of systematically increasing computational power contributed to
the fact that adsorption and reactivity can be currently
described with high accuracy \cite{Piccini2016, Piccini2015}. Nowadays,
computational modeling has become a key component to compare models with real
systems validating experimental observations or even make predictions for
catalyst design.
In this Chapter, the main modeling tools used in this PhD thesis are outlined. 
The Chapter is furthermore based on a book chapter entitled 'Theoretical tool
box for a better catalytic understanding' \cite{Waroquier2017}, and the results
obtained in \textbf{Paper VI}. In the latter paper, it was clearly shown how various modeling techniques can be applied to obtain a correct understanding
of intermediates at true reaction conditions.


\section{Theoretical methods}
\subsection*{Electronic structure methods for adsorption and reactivity}
In this PhD thesis the use of first principles calculations is required as
reactions are being modeled in which bonds are being broken and formed. Currently, also reactive force fields are being developed but their application to describe the chemical reactions is
beyond the scope of this work \cite{Han2010, Huang2013}. The description of
adsorption of guest molecules in the pores of nanoporous materials is theoretically very challenging since an
accurate characterization of noncovalent interactions, which include both
electrostatic and dispersion interactions is mandatory. Different available first principles
methods exist to solve the Schr\"{o}dinger equation associated with a molecular
system, and among these, Density Functional Theory (DFT) provides an excellent
synergy between the computational efficiency and accuracy for computing
electronic properties of large systems. The accuracy of DFT is highly dependent
on the functionals which have to be added to the model to account for exchange
and correlation.
In spite of its advantages, a drawback of commonly applied DFT functionals is
the lack of a proper description of long--range dispersion interactions which
are important to obtain information about the interaction between guest
molecules and host material \cite{Sauer1994}. This problem can be overcome by
implementing a van der Waals scheme to account for the long--range electronic
correlations \cite{Grimme2004}. A complete overview of methods can be found in
dedicated reviews \cite{Sholl2006, Mardirossian2014} and a recent work of Mansoor \textit{et al.}\cite{Mansoor2018} about the impact of dispersive and long--range electrostatic interactions on theoretical
descriptions of adsorption and catalysis in zeolites.


\section{Framework topology}
As introduced in Chapter \ref{ch1}, the catalytic activity of multifunctional
MOFs originates from the presence of Lewis acid and Br\o{}nsted sites.
When describing catalytic centers it is important to include all ingredients that could have a
consequential effect on the reaction. In this respect, the choice of an
appropriate structural model to mimic the composition of the material is not
a trivial task, but nevertheless it is crucial to obtain a realistic
understanding of the function of the material. The topology can either be taken
into account by using an extended cluster or periodic model.
Including different computational cost, both methods have their pros and cons
while modeling catalysis in MOFs. 

\newpage
\subsection*{Extended cluster model}
In the extended cluster model, a representative part of the
material is considered to account for catalytically active sites making these
calculations computationally very efficient (Figure
\ref{fig:cluster}).
Another advantage of the finite cluster approach is the use of codes in which numerical algorithms to search for
transition states are better implemented than in solid--state periodic codes.
The power to apply very accurate electronic structure methods is directly connected
to the limited number of atoms contained in the clusters \cite{Lee2010,
Supronowicz2013, Bordiga2005}. However, by selecting molecular cluster model
from the periodic representation, due to the presence of dangling bonds, a charge
compensation needs to be introduced by anions or cations to reach charge
neutrality. To obtain a realistic estimate of the entropy contributions some
atoms which are on the cluster borders are fixed to prevent unphysical
deformations which can result from omitting the crystallographic
environment \cite{DeWispelaere2018}. This also includes a proper termination of
the selected cluster which may substantially alter the
results \cite{Supronowicz2013}. Furthermore, the calculations might be adequate
for initial guesses and benchmark purposes, but are insufficient to study complex catalytic transformations, including labile species and structural rearrangements.
Up to five years ago clusters were still commonly used in the field of
heterogeneous catalysis, however today the field is more and more shifting
towards using periodic based descriptions of the lattice \cite{Rogge2017,
Bernales2018, Evans2017, Fraux2017}.

\begin{figure}[!h]
	\centering
	\includegraphics[width=1.0\textwidth]{cluster}
	\caption{Selection of a cluster model from the periodic UiO--66 structure.}
	\label{fig:cluster}
\end{figure}
\npar
To obtain a first insight into complex reaction mechanisms, in some studies of
this thesis the UiO--66 framework was described by finite cluster
models \cite{Hajek2015, Vandichel2015, Caratelli2017}. The in--house developed
toolkit Zeobuilder \cite{Verstraelen2008} was used to cut the extended UiO--66
cluster from a supercell that contains four inorganic bricks with unit cell
formula \ce{[Zr6O5(OH)3(RCOO)11]2[Zr6O4(OH)4(RCOO)12]2} in which one linker defect was incorporated. The catalytically active extended cluster of an UiO--66 consisted of inorganic
\ce{Zr6O4(OH)4} brick with eleven linkers, of which seven were replaced by
formic groups while the four remaining linkers surround the active site (Figure
\ref{fig:cluster}).
The terminal oxygens of the terephthalate linkers and the hydrogens of formate
were fixed in the geometry optimization in order to mimic the periodic
environment and properly account for entropy \cite{DeWispelaere2018}. These
calculations were performed with Gaussian '09 by applying a hybrid functional
for the geometry optimization \cite{Gaussian}. To account for dispersion
effects, Grimme D3 corrections were added to the energies, as implemented in the
DFT--D3 program \cite{Grimme2010}. In the last years other more advanced
dispersion schemes have been developed such as the models of Tkatchenko \textit{et al.}\cite{Ambrosetti2014} or many body
dispersion models \cite{Bucko2016}. The influence of different functionals and
dispersion schemes, was tested in \textbf{Paper VI} for
the stability of adsorbed pentene on H--ZSM--5 \cite{Hajek2016}. 

\subsection*{Periodic model}
Periodic models of MOFs are computationally demanding but have the advantage
that the entire topology of the framework is considered in the model. The usage of periodic models is the most proper approach to study the reactions within
their true molecular environment at operating conditions and verify the dynamic
nature of active sites, their location in the pores, influence of guest species
or flexibility of the material. For calculations with periodic models, the
environment is fully described using typically
gradient--corrected exchange--correlation functionals \cite{Yang2010}. The
periodic simulations were carried out using the VASP and CP2K software packages.
In case of UiO--66 the conventional unit cell is composed of four inorganic
bricks, however, to reduce the computational time the unit cell can be
restricted to only two inorganic bricks (Figure \ref{fig:Cluster_Periodic}).
This reduced periodic cell approximation has a limitation that not all defect structures can be considered and periodic images
are closer to each other which might generate unphysical
perturbations \cite{Waroquier2017}.
\npar
Figure \ref{fig:Cluster_Periodic} shows that at 351 K the
adsorption of methanol on the two and four brick periodic unit
cells have almost equal free energies at the PBE--D3 level of theory
\cite{Caratelli2017}.
In \textbf{Paper V}, an extended cluster of UiO--66 was selected to compare with the periodic model the adsorption energies of
substrates using B3LYP functional.
It was shown that qualitatively the reaction mechanism of the aldol
condensation is well described by both approaches, but an accurate
quantification of the confinement effects induced by the framework was observed
in the periodic model of about 20 kJ/mol stronger adsorption of reactants
\cite{Hajek2015}.

\begin{figure}[!htp]
	\centering
	\includegraphics[width=1.0\textwidth]{Cluster_Periodic}
	\caption[In graph, free energy profile at T = 351 K for the deprotonation of
	methanol on UiO--66 on the four and two brick periodic unit cells, level of
	theory PBE--D3. In table, free energy, enthalpic and entropic
	contributions.]{ In graph, free energy profile at T = 351 K for the deprotonation
	of methanol on UiO--66 on the four and two brick periodic unit cells, level of
	theory PBE--D3. In table, free energy, enthalpic and entropic
	contributions. Adapted from Ref. \cite{Caratelli2017} with permission from
	Elsevier, copyright 2017.}
	\label{fig:Cluster_Periodic}
\end{figure}

\newpage
\subsubsection*{Equation of state fit}
When optimizing periodic structures the standard conjugated gradient optimizer
is not always sufficient to avoid the effects of an artificial Pulay stress
which arises when the volume is changed and a finite number of plane waves is
used. For all periodic calculations of unit cells without any adsorbents the
improved optimization scheme proposed by Vanpoucke and Lejaeghere \textit{et
al.}\cite{Vanpoucke2015} was applied. In a first step, the ion position and the
unit cell shape was relaxed. Subsequently, by keeping the volume fixed at discrete points and optimizing the ions and shape at each chosen volume an energy \textit{versus}
volume profile $E(V)$ was constructed. The minimum energy and corresponding
volume were predicted by Birch--Murnaghan equation of state fit and subsequently
the unit cells were optimized allowing ion positions and cell shape
relaxation \cite{Birch1947, Murnaghan1944}. Moreover, in \textbf{Paper II} the
mechanical properties of the structures were obtained from $E(V)$ curve by fitting a polynomial, and taking its second
derivative in the minimum \cite{Vandichel2016}. The bulk modulus $B$ is
expressed as:
\[
B = \left(V\frac{\partial^{2} E }{\partial V^{2}}\right)_{0}
\]
where the subscript $0$ means that the expression is evaluated at the minimum
energy point of the $E(V)$ curve, corresponding to the volume $V_{0}$.


\subsection*{Static DFT calculations}
A first insight in the catalytic processes can be gained by means of static approaches 
on cluster and periodic models in which a limited number of points on the
potential energy surface is considered as illustrated in Figure
\ref{fig:static}. The reaction rates and elementary chemical reactions are described by transition state theory (TST).
Calculations on extended clusters can be used as a first guess to explore reaction pathways and the different adsorption possibilities for the reactants. 
The states can then be optimized in the periodic methods using different algorithms implemented in the used codes. 
Frequency calculations are performed to ensure that the state is either a minimum or a saddle point in the free energy surface. 
To explore more complex reaction pathways or framework rearrangements 
different advanced static methods are available, like Nudged Elastic Band
approach.

\begin{figure}[!h]
	\centering
	\includegraphics[width=1.0\textwidth]{static}
	\caption{1--D free energy profile for a given reaction catalyzed on UiO--66, indicating the adsorbed initial and final states and the localised
	transition state for this reaction. The adsorption free energy, apparent and intrinsic barriers are obtained by static calculations and
	indicated by $\Delta G_{ads}$, $\Delta G_{app}$ and $G^\ddag$, respectively.}
	\label{fig:static}
\end{figure}
\npar
\newpage
\subsubsection*{Nudged Elastic Band approach to unravel a reaction path}
Chemical transformations are characterized by transition states which are
localized along the reaction coordinates and lead the system from the reactant
to the product region. By applying static methods the transition state, which is
a first order saddle point on the multidimensional potential energy surface
needs to be identified. A very useful method to find a minimum energy
path between known reactants and products is Nudged Elastic
Band (NEB) \cite{Sheppard2012, Sheppard2011, Sheppard2008}. The method works by
simultaneously optimizing a number of intermediate images. The energy is minimized perpendicular to the path in which
forces along the path are projected out and replaced with the spring force.
This allows attaining a minimum energy for each structure while remaining the
constant distance between the images, as
illustrated in Figure \ref{fig:NEB}. The point with the highest energy can be driven
up to represent a transition state by inverting the forces along the tangent of the
path. The NEB method was successfully used to study the dehydration of
UiO--66 in \textbf{Paper I} and \textbf{Paper II} \cite{Vandichel2015,
Vandichel2016}. Preliminary insights into the nature of the active sites and
initial structures of reactive intermediates can be obtained by means of first principle static calculations, where only a
distinct number of points on the potential energy surface is considered. The free energy profile
describing the reactants, transition states and products, may be constructed by adding proper thermal
corrections to the electronic energies.

\begin{figure}[!h]
	\centering
	\includegraphics[width=1.0\textwidth]{NEB}
	\caption{Schematic representation of the Nudged Elastic Band method.}
	\label{fig:NEB}
\end{figure}


\subsection*{Normal Mode Analysis}
Thermodynamic quantities like enthalpy, entropy and Gibbs free energy can be
calculated by means of partition functions which include every possible atomic
motion. Such partition functions are calculated by performing a normal mode
analysis (NMA) \cite{Frenkel2002}. This consists of diagonalization of the
mass--weighted Hessian matrix which contains the second order derivatives of the
energy with respect to the atomic positions to obtain its eigenvectors and
eigenvalues.
Typically, the vibrational partition functions are calculated in the harmonic oscillator
approximation. The computational cost of this
analysis increases gradually with the system size, therefore various NMA schemes
have been proposed to optimize the efficiency and accuracy of the frequency
simulations. For nanoporous applications, instead of the Full Hessian
Vibrational Analysis (FHVA) the Partial Hessian Vibrational
Analysis (PHVA) have been developed and implemented in
TAMkin toolkit (the in--house CMM software) \cite{DeMoor2011, Ghysels2010}. In a
PHVA scheme some of the atoms which do not play an active role in the chemical transformation are kept fixed
in space, and as a consequence they have an infinite mass in the NMA approach.
This procedure prevents to obtain spurious imaginary frequencies related to
lattice vibrations. For further details, we refer to the paper by De Moor
\textit{et al.} \cite{DeMoor2011}. The PHVA approach has been applied in this
thesis to calculate vibrational frequencies, partition functions and related thermodynamics.


\subsection*{Thermodynamic quantities and reaction rates}
In order to qualitatively understand how chemical reactions take place the
TST can be used. TST is based on the fundamental
assumption of the existence of an activated complex (transition state) in the phase space that
divides reactants and products. The calculation of
absolute reaction rates requires precise knowledge of potential energy surface,
therefore, for complex catalytic transformations the use of TST might not be
sufficient. Nevertheless, it is one of the
most successful theories in theoretical chemistry to calculate reaction energies and barriers, which are reported in terms of Gibbs free
energy, which is composed of enthalpic and entropic
contributions \cite{Eyring1935}. The total internal energy $U$ (expressed in
kJ/mol) is calculated by adding to the electronic energy $E_{DFT-D}$ the thermal
contributions obtained from the molecular partition function $Q$:
\[
U = U_{0} + R T^{2}\left(\frac{\partial \ln Q}{\partial T}\right)_{NV}
\]
\[ U_{0} = E_{DFT-D} + E_{ZPE} \]
\[ R = N_{A} k_B \]
where $R$ in the gas constant, $N_{A}$ is the Avogadro constant, $\it{k}_{B}$ is
the Boltzmann constant, $T$ is the temperature and $E_{ZPE}$ is zero--point
vibrational energy.\\

\noindent The enthalpy $H$ is given by the sum of internal energy $U$ and the
$pV$ work in the system. For non--interacting particles the ideal gas law
can be applied for 1 mol:
\[
H = U + pV = U + R T
\]
The entropy $S$ is the derivative with respect to temperature at constant
volume.
It can be resolved directly from the partition function such that:
\[
S = R \ln Q + R T \left(\frac{\partial \ln Q}{\partial
T}\right)_{NV}
\]
Ultimately, the Gibbs free energy $G$ is determined as:
\[
G = H - TS
\]
By applying TST, the calculated partition functions and the differences in
electronic energy between the transition states and the reactants allow to
determine a rate coefficient $k(T)$. Depending on the temperature, it quantifies
the rate of an unimolecular chemical reaction, which is calculated as:
\[
k(T) = \frac{{k_B T}}
{h}\frac{{q_{TS,\ddagger}}}{{q_R}} \exp\left(- \frac{\Delta E^{\ddagger}}{k_B
T}\right)
\]
\\
\noindent
in which $\it{k}_{B}$ is the Boltzmann constant and $\it{h}$ is the Planck
constant. The molecular partition function $q_R$ relates to the reactant and
$q_{TS,\ddagger}$ corresponds to the molecular partition function of the
activated complex excluding the contribution for the degree of freedom
representing the reaction coordinate.\\
It should be noted that the zero--point energies of the various vibrational
modes are lifted out from the vibrational partition functions of the different
components of the reaction. They are included in the molecular energy difference 
$\Delta E^{\ddagger}$ at the absolute zero between the activated
complex and the reactant. 
\[
\Delta E^{\ddagger} = E_{0}^{TS} - E_{0}^{R} + \Delta E_{0,vib}
\]
 
\[
\Delta E_{0,vib} = \sum_{i=0}^{N_{dof}-1} \frac{h \nu_{i}^{TS,\ddagger}}{2}
- \sum_{i=0}^{N_{dof}} \frac{h \nu_{i}^{R}}{2}
\]
\noindent
$E_{0}^{TS}$ and $E_{0}^{R}$ are the total binding energies of the activated
complex and the reactant in their electronic ground state configurations,
respectively. In this way, all partition functions should be evaluated with
respect to the zero--point levels of molecules.\\
\noindent
The vibrational contribution to the partition function is obtained by means of
following expression:
\[
q_{vib,i} = \prod_{i=1}^{N_{dof}} \frac{1}{1 - \exp\left(- \frac{h
\nu_{i}}{k_B T}\right)}
\]
\noindent
$N_{dof}$ are the numbers of vibrational degrees of freedom of the system.

\newpage
\subsection*{\textit{Ab initio} molecular dynamics}
In order to sample conformational freedom and adsorption of substrates at true
operating conditions usage of a dynamical approach is required, in which effects
of temperature, pressure and flexibility of the material can be taken into account.
In molecular dynamics (MD) simulations the time--dependent motion of a system is defined at the molecular
level by solving Newton's equation. Within the framework of this thesis various
\textit{ab initio} MD simulations were performed based on a periodic DFT
description including Grimme D3 dispersion corrections \cite{Grimme2010}. In the
MD simulations the temperature and pressure were controlled by a chain of five
Nos\'e--Hoover thermostats and by a Martyna--Tobias--Klein (MTK) barostat,
respectively \cite{Frenkel2002, Martyna1994}. Such simulations can be performed in different thermodynamics ensembles and the most commonly applied are the canonical ensemble (NVT) and the isobaric--isothermal (NPT) ensemble. In what follows, we will illustrate by means of three case studies the pros and cons of various simulations techniques to study adsorption.
\npar
To show the complementarity of various methods to describe adsorption an example
is given from \textbf{Paper X}. The position of adsorbed dimethyl
methylphosphonate (DMMP) as a case study model for the infamous 'nerve agent'
group of alkyl phosphonate compounds was investigated in the defective
UiO--66(\ce{NH2}). The insight in the adsorption of the DMMP molecule in the defective amino--functionalized material was obtain by periodic static and dynamic DFT calculations. Both types of simulations revealed two possible
adsorption sites. The adsorption can either occur in an adjacent position to a
linker defect in the octahedral cage or directly in the position of a missing
linker as shown in Figure \ref{fig:DMMP}. In the first case, long range
interactions between the amino groups and DMMP appear to be important stabilizing the adsorption
position, while in the second case, the molecule interacts closely with
water molecules and hydroxyl groups capping the defect sites on the brick.
For both sites the statically calculated electronic adsorption energy is 
of about 100 kJ/mol. To verify the mobility of the adsorbate in the framework
the MD simulations at 300 K were performed starting from both adsorption
positions. During 20 ps of simulations the DMMP molecule maintained the
initial adsorbed position, which further underlines its strong adsorption on both sites.
The NPT simulations allowed a 
comprehensive study of the synergies that point towards many confined
interactions between the adsorbed molecule and the
UiO--66(\ce{NH2}) framework \cite{Stassen2016}.

\begin{figure}[!h]
	\centering
	\includegraphics[width=1.0\textwidth]{DMMP}
	\caption{Snapshots of two possible adsorption configurations of the DMMP
	molecule in the pores of UiO--66(\ce{NH2}). Right: top panel
	presents an adjacent position of DMMP to a
linker defect in the octahedral cage; bottom panel shows the
	adsorption of DMMP in the position of a missing linker.}
	\label{fig:DMMP}
\end{figure}
\npar
\newpage
A multilevel modeling approach was
applied in \textbf{Paper VIII} to study the adsorption of methanol in
defective UiO--66. The estimation of loading of guest species within the pores
of nanoporous materials has drawn a lot of attention, for which typically, grand
canonical Monte Carlo (GCMC) simulations have been applied. These simulations
are performed with empirical force field models that do not describe the
electronic structure of the host--guest interactions, however, information
about the most plausible adsorption sites can be obtained. In this respect, the
description of methanol solvent introduced in the pores was done by GCMC
applying a universal force field, followed by a series of MD simulations using
the PBE functional at NPT and NVT ensembles. The D3 corrections were included
throughout geometry optimizations of molecular dynamics runs. In the simulations the appearance of various networks between methanol and water on the active
sites was detected. The configurations range from single closed loops of
hydrogen bonded methanol and water molecules to closed loops and open
methanol chains including up to 5--6 molecules (Figure
\ref{fig:methanol}). Interestingly, the structure of bulk methanol shows
some correspondence with the structure in a confined environment
\cite{Caratelli2017a}.

\begin{figure}[!h]
	\centering
	\includegraphics[width=1.0\textwidth]{methanol}
	\caption[Schematic representation of a pore in UiO--66. Left: periodic
	lattice; right: the pore where the missing linker defect is located, with
	included guest molecules. The active sites A and B are indicated
	on the left. Bottom: Ring configurations observed at sites A and B which arise
	from the interaction between the Zr--bonded hydroxyl groups and water and the solvent
molecules.]{Schematic representation of
a pore in UiO--66. Left: periodic lattice; right: the pore where the missing linker defect is located, with
	included guest molecules. The active sites A and B are indicated
	on the left. Bottom: Ring configurations observed at sites A and B which arise
	from the interaction between the Zr--bonded hydroxyl groups and water and the solvent
molecules. Adapted from Ref. \cite{Caratelli2017a}.}
	\label{fig:methanol}
\end{figure}
\npar
\clearpage
Another advanced periodic multilevel modeling approach was used to follow the
adsorption of pentene in H--ZSM--5 in \textbf{Paper VI}. Initially, a series of
\textit{ab initio} MD simulations was performed at 323 K on 1--pentene, 2--
pentene, 2--pentoxide and 3--pentoxide $\pi$--complexes to gain insight into the
mobility of the various adsorbed species (Figure
\ref{fig:pentene}). Preliminary MD runs of about 10 ps
were executed, in which the physisorbed pentene molecule was oriented in the
center of the straight 10--membered ring cavity at about 4 \AA\ from the acid
site. The diffusion of the 2--pentene from an unbiased initial position, in
which the adsorbate only interacts with the framework of the zeolite quickly
evolves towards the acid site to form the $\pi$--complex. The preferential
orientation of the 2--pentene molecule is in the intersection of the two pores,
which corresponds to the methyl tail directed in the zig--zag channel near the
Br\o{}nsted acid site and the longer ethyl tail in the straight cavity (Figure
\ref{fig:pentene}, b).
These initial MD runs were used to select configurations for a subsequent set of extensive MD simulations of 100 ps. For both 1-- and 2--pentene the shortest \ce{C-H} distance remains on average
of about 2 \AA, pointing that the \ce{$\pi$-H} interaction is in place
throughout the simulation. A similar position analysis was made for the alkoxides
indicating that these complexes are also stable during the simulation with
average \ce{C-O} distances of about 1.6 \AA. In the MD simulations at 323 K we
did not observe the formation of carbenium ions.
 
\begin{figure}[!h]
	\centering
	\includegraphics[width=1.0\textwidth]{pentene}
	\caption[MD snapshots of (a) 1--pentene $\pi$--complex, (b) 2--pentene
	$\pi$--complex, (c) chemisorbed 2--alkoxide and (d) chemisorbed 3--alkoxide in
	H--ZSM--5 at 323 K, seen in the direction of the straight channel (camera
	viewpoint). The snapshots correspond to geometries which are most frequently visited during MD runs of
100 ps at 323 K.]{MD snapshots of (a) 1--pentene $\pi$--complex, (b) 2--pentene
	$\pi$--complex, (c) chemisorbed 2--alkoxide and (d) chemisorbed 3--alkoxide in
	H--ZSM--5 at 323 K, seen in the direction of the straight channel (camera
	viewpoint). The snapshots correspond to geometries which are most frequently visited during MD runs of
100 ps at 323 K. Reproduced from Ref. \cite{Hajek2016} with permission of
Elsevier, copyright 2016.}
	\label{fig:pentene}
\end{figure}
\clearpage
Subsequently, based on the
probability distribution plotted in Figure \ref{fig:probability} the adsorption
positions corresponding to the most frequently visited structures during the MD runs were determined to optimize these geometries statically. Thereafter, information about the adsorption enthalpies was obtained from the frequency calculations and the influence of finite temperature effects on the adsorption enthalpies was assessed.
The plotted probability distributions for the distances of the various adsorbed
species in H--ZSM--5 show an asymmetric behavior for the $\pi$--complexes
towards higher \ce{C-H} distances.
The adsorption enthalpy of physisorbed species of 1-- and 2--pentene is heavily
allied with the \ce{C-H} distance.
The broad probability distribution for these two molecules illustrates that the
average ensemble over the MD simulations results in enthalpy of adsorption which
coincides to \ce{C-H} distances larger than 2 \AA.
This contrasts with the static calculations, in which only one minimum on the
potential energy surface is considered excluding configurations with slightly
larger distances which are detected in the MD simulations. Summarizing, the
dynamically averaged values of the adsorption enthalpies give systematically
lower values for the $\pi$--complexes and slightly higher values for the
alkoxides adsorption enthalpic energies. The position of the double bond does not significantly
affect the enthalpy of formation of the $\pi$--complexes which was also observed
by Bhan \textit{et al.} \cite{Bhan2003}. Furthermore, by taking into account
finite temperature effects, the $\pi$--complexes are almost equally stable as the
alkoxide species \cite{Hajek2016}.

\begin{figure}[!h]
	\centering
	\includegraphics[width=1.0\textwidth]{probability}
	\caption[Probability distributions of observing some distances in molecular
	dynamics simulations of $\pi$--complexes and alkoxides in H--ZSM--5 throughout
	the final 60 ps simulation time.]{Probability distributions of observing some distances in molecular
	dynamics simulations of $\pi$--complexes and alkoxides in H--ZSM--5 throughout
	the final 60 ps simulation time \cite{Hajek2016}. Reproduced with
	permission of Elsevier, copyright 2016.}
	\label{fig:probability}
\end{figure}


\subsection*{Enhanced sampling MD methods}
In principle, MD simulations allow access to free energy profiles of chemical
reactions, however, due to the broad range of characteristic time scales related to a molecular system this task is far from trivial \cite{ Fleurat2005}. Chemical reactions are most of the time rare events and the barrier from a reactant to a product state is usually too high, which gives rather low probabilities of happening during a regular MD run of a few ps.  This limitation can be overcome by using different available advanced molecular dynamics techniques. Currently, a whole plethora of these enhanced sampling
methods  is available, which allow exploring low probability regions of
the free energy surface \cite{ Laio2002,  Sutto2012, Carter1989, Darve2001,
Jarzynski1997,  Rosso2002, Gullingsrud1999}. Detailed reviews on the advanced
molecular dynamics simulations are given in the work of Valsson \textit{et
al.}\cite{Valsson2016} and references therein. The enhanced sampling MD techniques have only recently been successfully applied in the field of heterogeneous catalysis to study chemical reactions at true operating conditions of temperature and
pressure \cite{DeWispelaere2016, DeWispelaere2015, VanSpeybroeck2014,
Cnudde2017, Haigis2015, Bucko2011}. An overview of the recent advances in
computational techniques for nanoporous materials can be found in
Ref.\cite{Evans2017, Fraux2017}. In general, two classes of these methods can be
distinguished. In the first class, dynamics techniques enhance the sampling of all degrees of freedom.
The Replica Exchange (RE) \cite{Sugita1999} and Transition Path Sampling
(TPS)\cite{Dellago2006} methods are very useful to discover new reaction
mechanisms without the prior knowledge or definition of transition states. These
methods applied in the field of porous materials 
have the ability to predict selectivity for various products of
zeolite--catalyzed reactions, as was indicated by Bu\u{c}ko \textit{et
al.}\cite{Bucko2009}. 
In the second class of advanced dynamics methods, techniques are defined in
which the sampling of low probability regions is enhanced along certain coordinates. The most common are
Umbrella Sampling (US)\cite{Patey1975, Torrie1977} and Metadynamics
(MTD) \cite{Laio2002}. The latter coordinates are commonly referred to as
collective variables (CV) which may be internal coordinates such as bond lengths, bond angles or more complex variables like coordination
numbers \cite{Rohrdanz2013}. The free energy methods considered in this work are
schematically represented in Figure \ref{fig:dynamic}. 


\subsubsection*{Metadynamics}
MTD first introduced by Laio and co--workers \cite{Laio2002}, is a popular
method to accelerate the occurrence of reactions in which a bias potential by means of Gaussian hills
is added to the potential energy surface. The bias potential is updated
on--the--fly, artificially increasing the potential energy of the already
visited states.
In such a fashion, states other than the local minima are sampled in an MTD
approach. There exist various algorithms to update the potential energy surface, most notably is the variational approach.
When all states become equally probable under the combined action of the potential energy surface and the bias potential, 
the free energy is equivalent to the inverse bias potential. The method serves
as an excellent tool to scan the configuration space in the direction of the
CV for possible minima and has recently been applied
successfully in various zeolite catalysis studies \cite{DeWispelaere2015,
DeWispelaere2016}.

\begin{figure}[!h]
	\centering
	\includegraphics[width=1.0\textwidth]{dynamic}
	\caption[Schematic representation of the free energy methods considered in this work. Each panel represents a different free energy method. Within
each panel, the top figure shows the simulation result and the bottom figure concerns the estimated free energy profile. The color coding is black for
the unknown free energy, green for the simulation results and the estimated free
energy, and purple for the sampling methods. Below the method panel,a possible
2--D representation of the given reaction on the two active sites as obtained using
advanced dynamic techniques, indicating the three critical points on the potential energy surface.]{Schematic representation of the free energy methods considered in
	this work. Each panel represents a different free energy method. Within each panel, the top figure shows the simulation result and the bottom figure concerns the estimated free energy profile. The color coding is black for
the unknown free energy, green for the simulation results and the estimated free
energy, and purple for the sampling methods. Below the method panel,a possible
2--D representation of the given reaction on the two active sites as obtained using
advanced dynamic techniques, indicating the three critical points on the
potential energy surface. Adapted from Ref.\cite{Demuynck2017} and \cite{Rogge2017}.}
	\label{fig:dynamic}
\end{figure}


\subsubsection*{Umbrella sampling}
In US an external potential is added to the true Hamiltonian to improve the
sampling of low probability regions along defined coordinates of the system. By
performing various US simulations, targeting different regions of the CV, the entire range of interest in terms of the CV can be sampled. A free energy profile can be estimated
employing a scheme connecting the local information from the various
simulations, such as the weighted histogram analysis method or multistate
Bennett acceptance ratio \cite{Kumar1992, Patey1975, Torrie1977}. Being highly
parallelizable US is a very efficient sampling method.
Nevertheless, prior knowledge is required both in terms of an accurate
CVs and a descent estimate of the various stable states.


\subsubsection*{Choice of CV}
The success of the MTD and US methods critically relies on a decent
choice of the CVs which enable to lead the
system to parts of interest on the free energy surface. The suitable choice is
not evident and requires a good insight about a studied
reaction \cite{Rohrdanz2013}. When simulating complex chemical transformations,
the use of non geometric parameters by means of coordination numbers are especially advisable to describe CV. The coordination number CN used in the simulation is formally
defined as: 
\[
\mathrm{CN} =\sum_{i,j}\frac{1-(r_{ij}/r_0 )^{nn}}{1-(r_{ij}/r_0 )^{nd}}
\] 
\\
where the sum runs over two sets of atoms ${i}$ and ${j}$, $r_{ij}$ defines the
distance between atoms i and j and $r_0$ is a reference distance. The parameters
$nn$ and $nd$ are usually set to 6 and 12, respectively. 
\npar
The definition of CV in terms of two coordination numbers was used in
\textbf{Paper VI} to study the formation of the activated transitions on the
reaction path on the formation of 2--pentoxide from physisorbed 1--pentene and
2--pentene and that of 3--pentoxide from physisorbed 2--pentene. The results
from MTD simulations to sample the possible occurrence of carbenium ions in the
reaction paths from $\pi$--complexes to alkoxides are shown in Figure
\ref{fig:2_3_pentoxide_FES}. To describe the proton transfer from the zeolite framework to the pentene molecule, the combined coordination number of
\ce{O-H} bond cleavage and \ce{C-H} bond formation was defined as CV1:
\ce{CN(O-H)-CN(C_1-H)}.
The formation of the \ce{C-O} bond between the resulting pentyl carbenium ion
and the zeolite framework was investigated with the second CV2 as \ce{CN(C_2-O)}
as displayed in Figure \ref{fig:2_3_pentoxide_FES}, top.
The carried out MTD simulations resulted in 2--D free energy surface. 
In the case where more than one CV is applied to describe the process, it is
necessary to project the resulting free energy surface of the multidimensional
simulations into a lower dimensional coordinate \cite{DeWispelaere2015}. The
lowest free energy path (LFEP) method proposed by Ensing \textit{et al.}
\cite{Ensing2005} was successfully applied to obtained a 1--D profile. The
lowest free energy paths describing the three transformations in 2--D and
1--D energy surfaces are displayed in Figure \ref{fig:2_3_pentoxide_FES}.
The existence of a metastable state between the $\pi$--complex and the pentoxide is evident and characterized as a pentene carbenium ion.

\begin{figure}[!htp]
	\centering
	\includegraphics[width=0.95\textwidth]{2_3_pentoxide_FES}
	\caption[Top: Schematic representation of the applied collective variables.
	Left: 2--D free energy surface for the formation of 2--pentoxide from a
	1--pentene $\pi$--complex (a) or 2--pentene $\pi$--complex (b) and for the formation
	of 3--pentoxide from a 2--pentene $\pi$--complex (c). The lowest free energy paths
	are displayed. A small well corresponding to the metastable carbenium ion can be observed in the bottom left corner.
Right: corresponding 1--D free energy profile along the lowest free energy path
for the two reactions.]{Top: Schematic representation of the applied collective variables.
	Left: 2--D free energy surface for the formation of 2--pentoxide from a
	1--pentene $\pi$--complex (a) or 2--pentene $\pi$--complex (b) and for the formation
	of 3--pentoxide from a 2--pentene $\pi$--complex (c). The lowest free energy paths
	are displayed. A small well corresponding to the metastable carbenium ion can be observed in the bottom left corner.
Right: corresponding 1--D free energy profile along the lowest free energy path
for the two reactions. Reprinted with permission from Ref.\cite{Hajek2016}.}
	\label{fig:2_3_pentoxide_FES}
\end{figure}
\npar
\newpage
In summary the plethora of techniques presented in this Chapter, shows that
nowadays a rich computational toolbox is available to study chemical reactions in zeolites and MOFs.
At the onset of this PhD thesis, a lot of cluster based models were still used, 
whereas today usage of periodic DFT based methods has become more common.  
A very intriguing evolution within the time frame of this thesis, is the ability
to simulate chemical reactions at operating conditions, using advanced MD
methods. Finally the particular choice of methods is guided by the system at
hand, computational resources and the phenomenon one wishes to describe.
Regarding computational costs, it needs to be mentioned that first principle MD
based methods are computationally very expensive and may not be regarded as the standard method to study any reaction at hand. 
For information, to describe the same chemical transformation static simulations
take approximately 20--25 nodedays and MD simulations around 150--200 nodedays
(Machine specs: 2 x 10--core Intel E5--2660v3 (Haswell--EP @ 2.6 GHz)).



\clearpage{\pagestyle{empty}\cleardoublepage}






