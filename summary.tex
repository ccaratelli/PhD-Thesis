\chapter{Samenvatting}
Nanoporeuze materialen worden intensief bestudeerd de laatste decennia omwille van hun veelzijdigheid aan toepassingen in tal van sectoren. Dit toepassingspotentieel heeft vooral te maken met hun groot intern oppervlak en porie--volume, waardoor ze uitermate geschikt zijn binnen de heterogene katalyse en adsorptie toepassingen. Een groot scala van nanoporeuze materialen zijn momenteel beschikbaar zoals zeolieten en actieve kool die op grootschalige basis worden gebruikt binnen de petrochemie, geneeskunde en milieutoepassingen. 
\npar
Recentelijk werden nieuwe generaties van nanoporeuze materialen ontwikkeld waarvan het bouwconcept gebaseerd is op bouwstenen die modulair kunnen worden samengebracht. Tot deze klasse behoren onder meer metaal--organische roosters (MOFs) en covalente organische roosters (COFs). Vooral MOFs zijn op korte tijd geëvolueerd tot één van de meest onderzochte materialen en dit omwille van hun grote variabiliteit in bouwconcept, wat het mogelijk maakt materialen te ontwerpen voor specifieke toepassingen. Hoewel ze initieel vooral werden onderzocht voor adsorptietoepassingen, wordt het toepassingspalet volop uitgebreid de laatste jaren. Op dit moment zijn reeds talrijke succesvolle voorbeelden beschikbaar van MOFs binnen het domein van de heterogene katalyse. 
\npar
Het uniek bouwconcept berust op het samengaan van anorganische bouwblokken met multitopische organische linkers via coördinatie bindingen. Op die manier worden 1D--, 2D-- of 3D-- kristallijne netten gevormd. Door het feit dat men kan uitgaan van een groot aantal anorganische bouwblokken die op hun beurt kunnen gecombineerd worden met tal van organische linkers, kunnen een bijzonder groot aantal materialen worden gesynthetiseerd. Inderdaad tot op vandaag werden er meer dan 10000 materialen experimenteel gemaakt. De grote uitdaging bestaat er echter in om de materialen zo te ontwerpen zodat ze optimaal geschikt zijn voor een bepaalde toepassing. Naast deze grote verscheidenheid gegenereerd door de grote combinatoriek van bouwblokken, kunnen de materialen ook nog verder worden gefunctionaliseerd worden door middel van post--synthetische modificaties. Het mag duidelijk zijn dat metaal organische roosters omwille van bovenstaande aspecten bijzonder veel potentieel vertonen voor ontwerp en design naar specifieke toepassingen toe. 
\npar
In tegenstelling echter tot zeolieten, de welbekende familie van nanoporeuze anorganische materialen die momenteel op grote schaal worden gebruikt binnen tal van industriële toepassingen, zijn ze echter minder stabiel. Dit heeft te maken met de aanwezigheid van de metaal--ligand binding die inherent zwakker is dan de silicium--oxide binding die aan de basis ligt van zeolieten. Echter zeolieten kunnen niet gefabriceerd worden met een even grote verscheidenheid en lenen zich in die zin minder tot moleculair ontwerp voor specifieke toepassingen. 
\npar
In deze thesis worden metaal organische roosters onderzocht voor toepassingen binnen de katalyse. Zoals reeds gesteld, bieden metaal organische roosters de mogelijkheid om poreuze materialen te ontwerpen met het doel om optimale chemische conversies met gewenste productselectiviteit en opbrengst te bekomen. Naast deze specifieke eigenschappen, behoren ze uiteraard tot het domein van de heterogene katalyse, wat maakt dat ze toelaten meer duurzame processen te ontwerpen gezien inherent product separatie of toxisch afval minder is in vergelijking met processen waarbij homogene katalysatoren worden gebruikt. 
\npar
Een groot vraagstuk bestaat erin te begrijpen hoe actieve sites voor katalyse eruit zien en hoe zij kunnen worden gegenereerd en gemoduleerd. Al vrij vlug na de vlug na de initiële synthese van de eerste metaal organische roosters werd duidelijk dat deze materialen inherent een grote mate van wanorde kunnen bezitten en structurele defecten. Initieel werd dit vooral als een nadeel gepercipieerd, echter binnen de huidige wetenschappelijke context wordt de aanwezigheid van defecten en hun mogelijkheid tot modulatie geëxploiteerd voor specifieke toepassingen. Dit is met name belangrijk voor de creatie van actieve sites voor katalytische reacties. Binnen deze thesis zullen ondermeer materialen worden onderzocht die indien perfect opgebouwd enkel volledig gecoördineerde metaalsites bevatten en in die zin open metaal sites ontbreken waaraan katalytische reacties kunnen plaatsvinden. Het structureel inbouwen van defecten biedt de mogelijkheid om de materialen te activeren voor katalytische toepassingen. Verder kunnen defecten er ook voor zorgen dat de poriegrootte wordt aangepast bijvoorbeeld door structureel ontbreken van anorganische bouwblokken in het materiaalskelet. De grote uitdaging bestaat erin om de aanwezigheid van defecten te controleren en ook hun moleculair gedrag te begrijpen. 
\npar
De experimentele karakterisatie van deze materialen tot op moleculair niveau is bijzonder uitdagend mede door de complexiteit van hun structuur en de aanwezigheid van structurele defecten. De aanwezigheid van wanorde maakt het niet triviaal voor experimentatoren om direct inzicht te verkrijgen in de moleculaire aard van de actieve sites en structuur--eigenschappenrelaties op te stellen. In dit opzicht zijn moleculaire modelleringstudies in nauwe samenwerking met experimentatoren van essentieel belang om het gedrag van MOFs te begrijpen en te voorspellen. Binnen het domein van de moleculaire modellering en de toepassing op nanoporeuze materialen heeft een enorme evolutie plaatsgegrepen de afgelopen tientallen jaren. Met behulp van computationele technieken is het nu mogelijk om MOFs te bestuderen op de nanoschaal. In die zin kunnen computersimulaties aangewend worden om experimentele waarnemingen te verklaren en in sommige gevallen kan gepoogd worden zelfs te voorspellen wat aangewezen materialen zijn voor bepaalde toepassingen of hoe reactieomstandigheden kunnen worden aangepast om een betere selectiviteit, activiteit te bekomen. Men moet zich echter realiseren dat dit laatste een bijzonder grote uitdaging blijft voor huidige modelleringstechnieken, gezien het zeer moeilijk is om theoretische modellen op te stellen die met voldoende nauwkeurigheid experimentele omstandigheden kunnen nabootsen. 
\npar
Een groot scala van moleculaire modelleringstechnieken zijn momenteel voorhanden om nanoporeuze materialen te bestuderen. Tal van keuzes dienen gemaakt te worden die een afweging zijn tussen enerzijds de haalbare computationele kost en anderzijds de accuraatheid waarmee het materiaal en het proces wordt beschreven. Idealiter benaderen moleculaire modellen zo goed als mogelijk de experimentele en reële condities. Afhankelijk van welke aspecten men wenst te bestuderen kunnen verschillende benaderingsmethoden worden toegepast. Indien men bijvoorbeeld geïnteresseerd is in fysische eigenschappen zoals fasetransformaties, kan het volstaan om de chemische interacties op benaderende wijze te beschrijven aan de hand van krachtvelden waardoor meer atomen kunnen worden in rekening gebracht worden. Echter voor de katalyse, is het essentieel om de chemische bindingen voldoende accuraat te beschrijven teneinde de actieve site en het reactiemechanisme correct in kaart te brengen. Dergelijke meer accurate methoden die gebaseerd zijn op een expliciete beschrijving van de elektronische structuur, vergen een hogere computationele kost waardoor de modelsystemen in grootte moeten beperkt worden. Concreet betekent dit het aantal atomen van het systeem beperkt blijft tot een paar honderdtal atomen. Een grote uitdaging bestaat erin de topologie van het materiaal correct in kaart te brengen en ook rekening te houden met procescondities zoals een realistische temperatuur, gasdruk of aanwezigheid van solventen in de poriën van het materiaal. De huidige computationele technieken zijn dermate geëvolueerd dat dergelijke beschrijving van een chemische reactie op procescondities binnen het bereik komt. Dit heeft niet alleen te maken met de opkomst van heel krachtige High Performance computers maar ook met de ontwikkeling van ingenieuze theoretische algoritmes die toelaten om grote moleculaire systemen accuraat te berekenen. 
\npar
Binnen deze thesis werden zirconium materialen bestudeerd voor katalytische toepassingen. Meer in het bijzonder werd een uitgebreid onderzoek verricht van de actieve binnen het zeer stabiel UiO--66 materiaal. Er werd specifiek geopteerd om dit materiaal in detail te onderzoeken, gezien het een van de meest onderzochte materialen is voor verscheidene toepassingen en dit omwille van de zeer hoge stabiliteit binnen de familie van de metaal organische roosters. De grote stabiliteit vindt zijn oorsprong in de opbouw van het materiaal, namelijk zirconium oxide clusters worden geconnecteerd met tereftalaat linkers en dit met een heel hoge graad van connectiviteit. Elke anorganische bouwblok wordt geconnecteerd met 12 organische linkers in het defectvrije materiaal. Dit maakt dat het materiaal bijzonder robuust is in verscheidene chemische omstandigheden en thermische condities, waardoor het volop kan worden ingezet voor tal van katalytische toepassingen. Het defectvrije materiaal bevat geen open metaal sites en is in die zin niet geschikt voor katalytische toepassingen. Lange tijd was het onduidelijk hoe de actieve sites werden gecreëerd binnen dit materiaal. Door middel van geavanceerd fundamenteel wetenschappelijk experimenteel en theoretisch onderzoek, is het duidelijk geworden dat de structurele aanwezigheid van defecten aan de basis ligt van actieve sites voor katalyse. Zo kunnen ondermeer linkers ontbreken, maar aangezien het materiaal inherent een bijzonder hoge graad van connectiviteit vertoont, brengt dit de stabiliteit niet in het gedrang. Dit materiaal met zijn veelzijdigheid van eigenschappen en mogelijkheid tot modulatie vormt een ideaal platform voor dit doctoraatsonderzoek. 
\npar
Binnen dit doctoraatsondezoek werden actieve sites voor katalyse in het UiO--66 materiaal bestudeerd met een groot scala van computationele technieken teneinde een begrip te krijgen van het reactief gedrag op procescondities. Er werd zowel gebruik gemaakt van statische technieken waarbij slechts enkele punten op het potentieel energie oppervlak werden gelokaliseerd als moleculaire dynamica technieken waarbij het gedrag in functie van de tijd op reële temperaturen en in reële omstandigheden van reactanten in aanwezigheid van solventen werd gevolgd. Het potentieel energie oppervlak werd systematisch beschreven met behulp van dichtheidsfunctionaaltheorie. Hierdoor wordt de elektronische structuur van het materiaal expliciet beschreven met een haalbare computationele kost. Initieel werd voor sommige reacties gebruik gemaakt van uitgebreide clustermodellen om een eerste inzicht te krijgen in de lokale omgeving van de katalytisch actieve site. Nadien werd systematisch overgestapt op moleculaire modellen waarbij de topologie van het materiaal met inbegrip van de periodieke randvoorwaarden correct in kaart werd gebracht. Het computationeel werk werd in grote mate verricht in samenwerking met verscheidene experimentele groepen. Verder werd ook samengewerkt met internationale theoretische partners om de kennis aangaande gevanceerde moleculaire dynamica technieken te versterken. 
\npar
Als startpunt van het doctoraatsonderzoek werd de Fischer esterificatie bestudeerd, welke een Lewis zuur gekatalyseerde reactie is en waarvoor een gunstig effect werd waargenomen in het experiment door de aanwezigheid van water. Initieel was het onduidelijk wat het effect was van water op de actieve site. Dankzij het toepassen van moleculaire modelleringstechnieken, werd duidelijk dat de actieve site niet alleen bestaat uit een Lewis zure site maar dat ook de aanwezigheid van Brønsted zure sites essentieel is voor optimaal functioneren van de actieve site. De aanwezigheid van waterige solventen heeft een gunstige invloed op de reactie aangezien het de mobiliteit van protonen faciliteert, waardoor een soort van dynamische aciditeit van het materiaal wordt gegenereerd. 
\npar
Startende van deze specifieke case studie, werd het duidelijk dat het gebruik van statische methoden, niet altijd voldoende is voor het bestuderen van katalytische reacties in deze materialen op procescondities, gezien in de realiteit het materiaal en de actieve site een sterk dynamisch gedrag kunnen vertonen. In groot deel van de thesis werd derhalve gebruik gemaakt van meer complexere modellen waarbij het solvent dat opgesloten is in de poriën van het materiaal expliciet in rekening wordt gebracht op realistische experimentele condities. Door het gebruik van dergelijke geavanceerde moleculaire modelleringstechnieken werd duidelijk dat protische solventen het gedrag van het materiaal in grote mate kunnen moduleren door continu te interacteren met de onder gecoördineerde zirconium clusters. De aanwezigheid van een dergelijke solvent laat verder toe om geladen intermediairen te stabiliseren. Een onderbouwd inzicht werd bekomen in de manier hoe defecten in deze materialen kunnen worden gevormd en gemoduleerd. Door toepassing van geavanceerd moleculaire dynamicatechnieken, werd het duidelijk dat de organische linkers een zekere graad van mobiliteit kunnen vertonen. Zij kunnen tijdelijk loskomen van de anorganische cluster, connecteren met een andere deel van de anorganisch bouwblok en dit terwijl de structurele integriteit van het materiaal behouden blijft. Dergelijke dynamisch gedrag van linkers blijkt tevens aan de basis te liggen van de postsynthetische modificatie van het UiO--66 materiaal. In samenwerking met experimentele partners werd dit proces nader bekeken. Een uniek inzicht werd bekomen in de moleculaire processen die aan de grondslag liggen van postsynthetische modificatieroutes. 
\npar
In een volgende fase werd nagegaan in hoeverre het waargenomen dynamisch gedrag van het UiO--66 materiaal met hoge connectiviteit, kon worden veralgemeend naar andere zirconium gebaseerde materialen. In dit opzicht werd het MOF--808 materiaal onderzocht, wat opgebouwd is uit gelijkaardige zirconium oxide clusters maar waarbij de connectiviteit met organische linkers lager is. Het materiaal bezit een groot porievolume waardoor het actief wordt onderzocht binnen het domein van de katalyse. Het werd duidelijk dat het waargenomen gedrag voor UiO--66 niet rechtstreeks kan worden doorgetrokken naar andere zirconium gebaseerde materialen. De lagere connectiviteit zorgt voor een aantal fundamentele verschillen met betrekking tot stabiliteit. Verdere studies in dit domein naar andere potentieel interessante zirconium materialen bieden een interessant perspectief voor toekomstige katalytische toepassingen. 
\npar
Samenvattend, werden in deze thesis een groot scala van computationele technieken aangewend om de eigenschappen van actieve sites voor katalyse te begrijpen op moleculaire schaal in zirconium gebaseerde materialen. Gezien in deze hoog geconnecteerde materialen de actieve sites worden gegeneerd door de creatie van structurele defecten, is het belang van solventen in de poriën van het materiaal van uitermate belang. Dit alles echter levert een nanogestructureerd platform voor katalyse dat bijzonder complex is maar ook een groot scala van mogelijkheid biedt voor verder exploitatie van katalytische reacties. De modelleringstechnieken die nodig zijn om dergelijk complex systeem te modelleren zijn zeer geavanceerd en inzicht kan enkel bekomen worden door complementaire technieken in te zetten. De bekomen resultaten benadrukken het belang van geavanceerde modelleringstechnieken om het katalytisch gedrag van dergelijke complexe materialen te ontrafelen op moleculaire schaal. 



\chapter{Summary}
In the past decades, nanoporous materials have become an intense field of study due to their numerous potential applications. This has to do mainly to their extremely high surface area and pore volume, that make them suitable in heterogeneous catalysts and adsorption applications. At present time, a wide range of nanoporous materials are available, such as zeolites and activated carbon, widely used in petrochemistry, medicine and environmental applications. 
\npar
Recently, new generations nanoporous materials have been explored which rely on the concept of reticular chemistry and building blocks, such as Metal Organic Frameworks (MOFs) and Covalent Organic Frameworks (COFs). In particular, MOFs have rapidly become one of the most deeply investigated classes of materials due to their great variability in design, that allows the synthesis of materials for specific applications. Although they were initially investigated for applications as adsorbants, their range of applications has been enormously expanded in recent years. Currently, many successful examples of MOFs are available in the field of heterogeneous catalysis.
\npar
The unique construction concept of MOFs is based on the combination of inorganic building blocks with multitopic organic linkers via coordination bonds. This way, 1--, 2-- or 3--D crystalline structures can be formed. A large number of inorganic building blocks can be combined with numerous organic linkers, allowing to construct a plethora of different materials. Indeed, up to today more than 10,000 MOFs have been synthesized experimentally. The big challenge lies in designing material that are optimally suited for a specific application. In addition to the large combination of building blocks, the materials can be further functionalized through post--synthetic modifications. Because of the above aspects, MOFs have a great potential to be be designed to target a specific application. 
\npar
However, in contrast to zeolites, the well--known family of inorganic nanoporous materials that are widely used in many industrial applications, MOFs are less stable. This is due to the presence of metal--ligand bonds that are inherently weaker than the Si--O bonds that lie at the basis of zeolites. However, zeolites cannot be manufactured with an equally large variety of structures and in this sense are less suitable for molecular design to target specific applications.
\npar
In this thesis, MOFs have been investigated for applications within catalysis. As already stated, MOFs offer the possibility to design porous materials to maximize the outcome of a given reaction, in terms of yield and selectivity for a desired product. Moreover, as heterogeneous catalysts, they can bring numerous advantages to industry. As opposed to homogeneous catalysts, they offer the possibility to design more sustainable processes, with less toxic waste and easy product separation.
\npar
A major issue lies in the understanding of how active sites for catalysis look like and how they can be generated and modulated. Soon after the synthesis of the first MOFs, it became clear that these materials inherently possess a high degree of disorder and structural defects. If at first this was seen as a drawback, within the current scientific context it has become clear that the presence of defects and their modulation can be exploited for specific applications. This is particularly important for the creation of active sites for catalytic reactions. Some MOFs, if perfectly constructed only contain fully coordinated metal sites and in this sense are missing open metal sites where reactions can take place. The structural incorporation of defects offers the possibility to activate the materials for catalytic applications. Furthermore, defects can also increase the pore size, for instance if inorganic building blocks are missing. The big challenge lies in investigating the presence of defects and understanding their molecular behavior.
\npar
The experimental characterization of these materials at the molecular level is particularly challenging due to their structural complexity and the presence of structural defects. The presence of disorder makes it not trivial for experimentalists to have direct insight on the molecular nature of the active sites and to understand structure--property relationships. In this sense, modeling studies in close synergy with experimental research are of utmost importance to understand and predict MOF behavior. 
\npar
The past decades have seen an enormous evolution within the domain of molecular modeling and its application to nanoporous materials. Using computational techniques, it is now possible to study MOFs at the nanoscale. In this sense, computer simulations can be used to explain experimental observations and in some cases even attempt to predict what are the appropriate materials for certain applications, or how reaction conditions can be adjusted to achieve better selectivity and activity. It must be realized, however, that the latter remains a major challenge for current modeling techniques, since it is very difficult to develop theoretical models that can simulate experimental conditions with sufficient accuracy.
\npar
Nowadays, a rich toolbox of molecular modeling techniques is available for the study of nanoporous materials which can be applied to MOFs. When studying such complex materials, there is always a trade--off between computational cost and the accuracy in the description of processes and material. Ideally, molecular models describe experimental conditions as closely as possible. Depending on which aspects one wants to focus on, different approaches can be used. For physical properties such as phase transformations, it may suffice to describe the chemical interactions approximately using force fields, so that more atoms can be taken into account in the model. However for catalysis it is essential to accurately describe the chemical bonds to correctly map active sites and reaction mechanisms. Such accurate methods, which are based on an explicit description of the electronic structure, are computationally more expensive, and the model systems must be limited in size. In concrete terms, this means that the number of atoms in the system is limited to a few hundred atoms. It poses a major challenge to correctly map the topology of the material and at the same time take into account operating conditions, such as realistic pressure, temperature and presence of guest molecules. Current computational techniques have evolved to such an extent that a description of chemical reactions at operating conditions is now within reach. This is not only due to the growth in computational power of High Performance Computing facilities, but also to the development of ingenious theoretical algorithms that allow to calculate large molecular systems accurately.
\npar
Within this thesis, zirconium MOFs were examined for catalytic applications. More in particular, an extensive study was conducted on the active sites within the extremely stable UiO--66 material. UiO--66 was specifically selected as it is one of the most investigated MOFs for various applications, and because of its high structural stability within the MOF family. The exceptional stability originates from the structure of the material, in which zirconium--oxide bricks are bridged by terephthalate linkers with a high structural connectivity. In the defect--free material, each inorganic brick is connected to 12 linkers. This makes the material extremely robust and able to withstand the chemical and thermal conditions of numerous catalytic applications. The defect--free UiO--66 does not contain open metal sites and is therefore not suitable for catalytic applications. For a long time it was not understood how the active sites were created within this material. By means of advanced experimental and theoretical research, it has become clear that the presence of structural defects is responsible for the creation of active sites for catalysis. The highly connected material allows for the presence of crystal defects such as missing linkers, without compromising its stability. This material, with its versatility of properties and possibility of modulation forms an ideal platform for this doctoral research.
\npar
Within this doctoral research, the active sites in the UiO--66 material were studied with different advanced modeling techniques in order to gain an understanding of the reactive behavior at operating conditions. Static techniques were used, where only a few points on the potential energy surface are investigated, as well as molecular dynamics techniques, where the behavior was followed at realistic temperatures and operating conditions in presence of solvents. The potential energy surface was systematically described using density functional theory, which allows explicitly describe the electronic structure of the material with a feasible computational cost. For some reactions, extended cluster models were initially used to gain insight into the local environment of the active site. We then made use of periodic boundary conditions to correctly map the topology of the material. The computational work was largely carried out in collaboration with various experimental groups. Furthermore, there was also collaboration with international theoretical partners to strengthen the knowledge on advanced molecular dynamics techniques.
\npar
As a starting point for the doctoral research, the Lewis--catalyzed Fischer esterification was studied, for which a beneficial effect of the presence of water was observed experimentally. Initially, it was unclear what the effect of water was on the active site. By use of molecular modeling techniques, it became clear that the active sites are not only composed of Lewis acid sites, but also Br\o{}nsted sites are needed for the optimal functioning of the catalyst in this reaction. The presence of aqueous solvent has a beneficial effect on the reaction, facilitating the mobility of protons and generating a sort of dynamic acidity in the material.
\npar
Starting from this specific case study, it became clear that static methods are not always sufficient to study the active sites in these materials at operating conditions, since the material can exhibit strong dynamic behavior. In the rest of this thesis, therefore, more complex models were used in which the solvent confined in the pores of the material is explicitly taken into account at realistic experimental conditions. By means of such advanced molecular modeling techniques, it became clear that protic solvents can modulate the behavior of the material to a large extent by continuously interacting with the under coordinated zirconium clusters. The presence of such a solvent also makes it possible to stabilize charged reaction intermediates. A substantial insight was obtained into how defects in these materials can be formed and modulated. By applying advanced molecular dynamics techniques, it became clear that the organic linkers can show a certain degree of mobility. They can temporarily decoordinate from the inorganic cluster and connect with another part of the inorganic building block, while maintaining the structural integrity of the material. Such dynamic behavior of linkers seems to lie at the basis of the post--synthetic modification of the UiO--66 material. This process was further examined in collaboration with experimental partners. A unique insight was obtained into the molecular processes that form the basis of post--synthetic modification routes.  
\npar
In a subsequent phase, it was investigated to what extent the observed dynamic behavior of the UiO--66 material, with high connectivity, could be generalized to other zirconium--based materials. 
In this respect, the MOF--808 material was investigated, composed by similar zirconium--oxide clusters, but with lower structural connectivity. The material has a large pore volume and for this reason it is actively investigated in catalysis. It became clear that the behavior observed for UiO--66 cannot be directly extended to other zirconium- based materials. The lower connectivity provides fundamental differences with regard to stability. Further studies in this domain into other potentially interesting zirconium materials offer an interesting perspective for future catalytic applications.
\npar
In summary, a rich computational toolbox was used in this thesis to gain molecular understanding on the properties of active sites for catalysis in zirconium based material. As the active sites in these highly connected materials are generated by creation of structural defects, the role of solvent confined in the pores is extremely important. This provides a platform for catalysis that is particularly complex, but also offers many possibilities for further exploitation of catalytic reactions. The modeling techniques required to model such a complex system are very advanced and insight can only be obtained by using complementary techniques. The results obtained emphasize the importance of advanced modeling techniques to unravel the catalytic behavior of such complex materials at the molecular scale.
