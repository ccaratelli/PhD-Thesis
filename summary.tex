\chapter{Samenvatting}
Summary in Dutch here

\chapter{Summary}
 In the past decades, nanoporous materials have become an intense field of study due to their numerous potential applications. Their extremely high surface area and pore volume make them very appealing for many processes in different industrial and scientific sectors. 
\npar
Metal Organic Frameworks (MOFs), or Porous Coordination Polymers (PCPs) are hybrid nanoporous materials formed by metal or metal--oxo clusters that are connected by organic linkers to form 1--, 2-- or 3--D crystals. Their particular building block design allows for a plethora of structures that can be generated, with different organic linkers, metals, metallic nodes and topology. Moreover, linkers and clusters can be further functionalized by post--synthetic modifications. The ease in their design makes them very appealing for different applications, as they can be in principle finely tuned to target a specific property. In contrast to zeolites, the most well--known nanoporous catalysts, in which the robust Si--O bonds can be broken only at harsh conditions, MOFs are characterized by weaker metal--ligand (M--L) bonds that can be easily broken. They have a much higher structural diversity and can be more easily tuned in terms of types of metal, topology and chemical functionality. The high (and tunable) porosity, metal content and crystallinity, as well as the possibility to include different functionalities, are very attractive characteristics for catalytic applications. In this sense, MOFs could be engineered to maximize the outcome of a give reaction, in terms of yield and selectivity for a given product. Moreover, as heterogeneous catalysts, they can bring numerous advantages to industry, as they are often less toxic or corrosive and can be easily separated from the products after the catalytic cycle. One of the most striking features which has drawn a lot of attention in the past years is the presence of structural defects and disorder in their structure. If at first this was seen as a drawback, now it has become clear that they can have a beneficial effect on their properties. In applications such as catalysis they can be the source of functionality, both directly, by generation of active sites with undercoordinated metal atoms, and indirectly, by increasing the pore size. However, we are still yet to understand the nature of defects at the molecular level, due to their intrinsic randomness that puts them in a continuum between order and disorder.
\npar
The inherent complexity of MOFs in terms of extremely high number of possible structures and defectivity makes their experimental characterization a particularly challenging task. In these materials the active sites are often generated by intentional creation of defects, and their inherent randomness entails a high complexity in their study. It is impossible to have direct experimental insight on active sites arising from defects and to draw clear structure--property relationship when such disorder is involved. Nevertheless, the molecular understanding of their active sites is crucial to predict their behavior and to design specific MOFs for target applications. For this reason, computational models, guided by experiments, are crucial for a complementary understanding and rationalization of experimental observations, and ultimately for prediction and design of new materials. The past decades have seen a flourishing in computational techniques, which provided new opportunities for exploration and understanding of chemical systems. Computational methods allow the study of MOFs at the nanoscale, offering a platform to understand and rationalize experimental observations. Besides supporting experimental data, they can be used to generate novel concepts, make predictions and design new materials and experiments. Theory can this way support experiments, and experiments can ultimately be used to validate theoretical predictions. 
\npar
Nowadays, a rich toolbox of molecular modeling techniques is available for the study of MOFs. There is always a trade--off between computational cost and complexity of the model. For a given computational cost, one has to make wise choice on what to prioritize between size of the model, length of the simulation, and chemical accuracy. For certain applications, such as the study of defects for adsorption, large systems are used, but with an approximate description of the chemical interactions. For catalysis, however, a high chemical accuracy is needed for the description of the active site, and this is often done at the expenses of the size of the model. The holy grail of modeling catalysis is certainly the description of processes at operating conditions, including all the factors that play a major role during the reaction mechanism and in the nature of the active sites. 
\npar
To understand active sites in MOFs from a computational point of view, it is important to work on well--studied material, where experimental information is easy to obtain. Among the plethora of known MOFs, UiO--66, composed by zirconium--oxide bricks and terephthalate linkers, is characterized by an exceptional stability that arises from its high structural connectivity. In the material, each inorganic SBU is connected to twelve linkers, and the structure is therefore extremely robust and able to withstand the chemical and thermal conditions at which reactive processes may take place. For this reason, it is at present moment one of the more widely investigated MOFs. The highly connected material allows for the presence of crystal defects such as missing linkers, which are responsible for its catalytic activity. UiO--66 and its derivatives represent a prototype MOF that has a high potential for industrial applications. 
\npar
Computational power has grown substantially during the past years, allowing to include more complexity in the model moving towards a description of the chemical processes at operating conditions. In this work of thesis, the active sites in the UiO--66 material were studied with different different advanced modeling techniques such as static period calculations and \textit{ab initio} molecular dynamics simulations. Density functional theory (DFT) was used as the method of choice that offered the best compromise between accuracy and computational cost, while allowing to study complex events involving bonds being broken and formed. Starting from extended cluster models, to gain a first insight into the active sites, we then made use of a proper periodic description of the MOF model that included confinement of the reactants and defective sites with undercoordinated zirconium atoms, that are generated upon removal of a linker. By modeling a Lewis--catalyzed reaction in the hydrated and dehydrated material, we came to the conclusion that the active sites are more complex than originally anticipated. We observed that water has an unexpected beneficial role in the reaction and UiO--66 is a dual Lewis/Br\o{}nsted catalyst where solvent plays a role in the reaction mechanism by providing support for proton transfers. 
\npar
With these observations, it is clear that for such complex phenomena, static methods are not always enough to study the nature of the active sites in these materials, and that a dynamic approach has to be followed. Allowed by the increase in computational power of high--performance computing (HPC) facilities, we increased the complexity of the model by including an explicit treatment of the solvent at realistic experimental conditions. We gained valuable insight into the dynamic events that can take place in the pores of the material. From these simulations, we understood that solvent can indeed modulate the properties of the active sites, by strongly interacting with the undercoordinated zirconium atoms, transferring protons, and providing a platform for stabilization of charged intermediates. Moreover, by means of enhanced sampling MD simulations, we shed light on how the strong interactions around the defective sites can help modulate the defectivity of the material, by inducing dynamic and reversible rearrangements of the structure. Linkers are more mobile than originally anticipated, and can decoordinate, translate, rotate and coordinate again to the zirconium atoms, without disrupting the stability of the structure. By understanding how these M--L bonds can evolve, we shed light onto the post synthetic ligand exchange process (PSLE) in UiO--66, that represents an optimal and easy way to functionalize the material. We observed similar dynamic behavior at dehydration conditions, when water is removed from the brick. With the insight gained from the study of UiO--66, we also focused on the structural stability of MOF--808, another Zr--MOF that shows high potential for catalytic applications. 
\npar
In summary, in this work of thesis we made use of a rich computational toolbox to understand the nature, properties and applications of active sites arising from defects in Zr--MOFs. We started from simple models and we built in complexity towards a description of the system at operating conditions, which has been done for the first time for such materials. The results obtained highlight the importance of computational techniques in elucidating the properties of such complex systems on the molecular level. Theory, combined with experimental insight, forms an efficient and effective way of exploring the world of such fascinating materials. We are just at the start of an era where theory and experiments can go hand in hand, by supporting each other, to better understand and predict realistic phenomena.


