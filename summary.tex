\chapter{Samenvatting}
Summary in Dutch here

\chapter{Summary}
 In the past decades, nanoporous materials have become an intense field of study due to their numerous potential applications. Their extremely high surface area and pore volume make them very appealing for many processes in different scientific and industrial sectors, primarily as heterogeneous catalysts or adsorbents. Examples of stable and well--known nanoporous materials are zeolites and activated carbon, widely used in petrochemistry, medicine and environmental applications. 
\npar
Recently, new branches of nanoporous materials have been explored which rely on the concept of reticular chemistry and building blocks, such as Metal Organic Frameworks (MOFs) and Covalent Organic Frameworks (COFs). In particular, MOFs have rapidly become one of the most deeply investigated classes of materials due to their outstanding properties and potential as catalysts. MOFs, or Porous Coordination Polymers (PCPs) are hybrid nanoporous materials, in which metal or metal--oxo clusters are connected by multitopic organic linkers via coordination bonds to form 1--, 2-- or 3--D crystalline structures. Their particular building block design allows for a plethora of structures that can be generated, with different metals, metallic nodes, organic linkers, topology and pore size. Moreover, these building blocks can be further functionalized by post--synthetic modifications. The ease in their design makes them very appealing for a broad range of applications, as they can in principle be finetuned to target a specific property. In contrast to zeolites, the most well--known inorganic nanoporous materials, in which the robust Si--O bonds can be broken only at harsh conditions, MOFs are characterized by weaker metal--ligand (M--L) bonds. They have a much higher structural diversity and can be more easily tuned in terms of types of metal, topology and chemical functionality. The high (and tuneable) porosity, metal content and crystallinity, as well as the possibility to include different functionalities, are very attractive characteristics for catalytic applications. In this sense, MOFs could be engineered to maximize the outcome of a given reaction, in terms of yield and selectivity for a desired product. Moreover, as heterogeneous catalysts, they can bring numerous advantages to industry, as they are often less toxic or corrosive and can be easily separated from the products after the catalytic cycle. One of the most striking features which has drawn a lot of attention in the past years is the presence of structural defects and disorder in their structure. If at first this was seen as a drawback, currently it has become clear that they can have a beneficial effect on their performance. In applications such as catalysis they can be the source of functionality, both directly, by generation of active sites with undercoordinated metal atoms, and indirectly, by increasing the pore size. However, we are still yet to understand the nature of defects at the molecular level. 
\npar
The inherent complexity of MOFs in terms of extremely high number of possible structures and presence of disorder makes their experimental characterization a particularly challenging task. In these materials, the active sites are often generated by intentional creation of defects, and their inherent randomness entails a high complexity in their study. 
The presence of disorder makes it not trivial for experimentalists to have direct insight on the molecular nature of the active sites and to understand structure--property relationships. In this sense, modeling studies in close synergy with experimental research are of utmost importance to predict their behavior and to design specific MOFs for target applications. The past decades have seen a flourishing in computational techniques, which provided new opportunities for exploration and understanding of chemical systems. Computational methods, guided by experiments, allow the study of MOFs at the nanoscale, offering a platform to understand and rationalize experimental observations. Besides supporting experimental data, theory can be used to generate novel concepts, make predictions and design new materials and experiments. 
\npar
Nowadays, a rich toolbox of molecular modeling techniques is available for the study of nanoporous materials which can be applied to MOFs. When studying such complex materials, there is always a trade--off between computational cost and complexity of the model. At given computational cost, one has to make wise choices on what to prioritize between size of the model (as number of atoms), length of the simulation, and chemical accuracy. For certain applications where physical properties are important, large model systems need to be used. However, to be able to involve a high number of atoms, the chemical interactions can only be described in an approximate way. For catalysis, on the other hand, a high chemical accuracy is needed to correctly describe active sites and reaction mechanisms. In this case, calculations are more expensive, and the size of the model needs to be reduced. 
It poses a challenge to understand in how far confinement effects, pressure, temperature and presence of guest molecules can play a major role in the behavior of the active sites influencing the reaction mechanisms. Luckily, computational power has grown substantially, allowing to include more complexity in the model, moving towards a description of chemical processes at operating conditions.
\npar
As a base case study for the general topic of this thesis, the extremely stable UiO--66 material is selected. To understand active sites in MOFs from a computational point of view, it is important to work on well--studied materials, where experimental information is the most available. Among the plethora of known MOFs, UiO--66, composed by zirconium--oxide bricks and terephthalate linkers, is characterized by an exceptional stability that arises from the nature of its M--L bonds and its high structural connectivity. In the material, each inorganic brick is connected to twelve linkers, and the structure is therefore extremely robust and able to withstand the chemical and thermal conditions at which reactive processes may take place. For this reason, it is at present moment one of the more widely investigated MOFs. Furthermore, the highly connected material allows for the presence of crystal defects such as missing linkers, which are responsible for its catalytic activity. For all these reasons, UiO--66 and its derivatives represent prototype MOFs that could be potentially used for industrial applications. 
\npar
In this thesis, the active sites in the UiO--66 material were studied with different advanced modeling techniques such as static period calculations and \textit{ab initio} molecular dynamics simulations. Density functional theory (DFT) was used as the method of choice that offered the best compromise between accuracy and computational cost, while allowing to study complex events involving bonds being broken and formed. To gain a first insight into the active sites that are generated upon removal of a linker, we started from extended cluster models. We then made use of a proper periodic description of the MOF model that included confinement of the reactants. By modeling Fischer esterification, a Lewis--catalyzed reaction, in the hydrated and dehydrated material, we came to the conclusion that the active sites are more complex than originally anticipated. We observed that water has an unexpected beneficial role in the reaction and UiO--66 is a dual Lewis/Br\o{}nsted catalyst, where Br\o{}nsted sites show an exceptional dynamic acidity. It became clear that solvent plays a role in the reaction mechanism by providing support for proton transfers.
\npar
With these observations, it is clear that for such complex phenomena, static methods are not always sufficient to study the nature of the active sites in these materials, and that a dynamic approach has to be followed. Allowed by the increase in computational power of high--performance computing (HPC) facilities, we increased the complexity of the model by including an explicit treatment of the solvent at realistic experimental conditions. We gained valuable insight into the dynamic events that can take place in the pores of the material. From these simulations, we understood that solvent can indeed modulate the properties of the active sites, by strongly interacting with the undercoordinated zirconium atoms, transferring protons, and providing a platform for stabilization of charged intermediates. Moreover, by means of enhanced sampling MD simulations, we shed light on how the strong interactions around the defective sites can help modulate the defectivity of the material, by inducing dynamic and reversible rearrangements of the structure. Linkers are more mobile than originally anticipated, and can decoordinate, translate, rotate and coordinate again to the zirconium atoms, without disrupting the stability of the structure. By understanding how these M--L bonds can evolve, we gained insight onto the post--synthetic ligand exchange process (PSLE) in UiO--66, that represents an optimal and easy way to functionalize MOF materials. We observed similar dynamic behavior at dehydration conditions, when water is removed from the brick. Starting from the excellent physical and chemical understanding obtained on UiO--66, we could focus on the structural properties of MOF--808, another Zr--MOF that shows high potential for catalytic applications and is not yet so well explored. We understood that the behavior of UiO--66 is rather exceptional, and the results cannot be easily generalized to other MOFs.
\npar
In summary, in this work of thesis we made use of a rich computational toolbox with a multilevel modeling approach to understand the nature, properties and applications of active sites arising from defects in Zr--MOFs and their interaction with solvent. Starting from simple models, the complexity was built towards a description of the system at operating conditions, which has been done for the first time for such materials. 
This work has been done in close synergy with experimental and theoretical collaborators.
The results obtained highlight the importance of computational techniques in elucidating the properties of such complex systems on the molecular level. Theory, combined with experimental insight, forms an efficient and effective way of exploring the world of such fascinating and advanced materials.


%This work shows that theory and experiments have to go hand in hand, by supporting each other, to better understand and predict realistic phenomena.

