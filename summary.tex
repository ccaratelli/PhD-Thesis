\chapter{Samenvatting}
Summary in Dutch here

\chapter{Summary}
In the past decades, nanoporous materials have become an intense field of study due to their numerous potential applications. Their extremely high surface area and pore volume make them very appealing for many processes in different industrial and scientific sectors.
\npar
Metal Organic Frameworks (MOFs), or Porous Coordination Polymers (PCPs) are hybrid nanoporous materials formed by metal or metal--oxo clusters that are connected by organic linkers to form 1--, 2-- or 3--D structures. Their particular building block design allows for a plethora of structures that can be generated, with different organic linkers, metals, metallic nodes and topology. Moreover, linkers and clusters can be further functionalized by post--synthetic modifications. The ease in their design makes them very appealing for different applications, as they can be in principle finely tuned to target a specific property. One of the most striking features is the presence of structural defects and disorder in their structure. If at first this was seen as a negative property, now it has become clear that they can have a beneficial effect on their properties. 
\npar
The high (and tunable) porosity, metal content and crystallinity, as well as the possibility to include different functionalities, are very attractive characteristics for catalytic applications. In contrast to zeolites, the most well--known nanoporous catalysts, MOFs have a much higher structural diversity and can be more easily tuned in terms of topology and chemical functionality. The molecular understanding of their active sites is crucial to predict their behavior and to design specific MOFs for target applications. However, in these materials the active sites are often generated by intentional creation of defects, and their inherent randomness entails a high complexity in their study.
\npar
For this reason, molecular modeling can represent a  
However, their study represents a challenge, as 



