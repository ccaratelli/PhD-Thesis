\graphicspath{{figures/chapter4/}}
% Header
\renewcommand\evenpagerightmark{{\scshape\small Conclusions and perspectives}} 
\renewcommand\oddpageleftmark{{\scshape\small Chapter 4}}

%\renewcommand{\bibname}{References}

\hyphenation{}

\chapter[Conclusions and perspectives]%
{Conclusions and Perspectives}
\label{ch4}
In this thesis, we shed light on the molecular characterization of the active sites in MOFs by means of different computational techniques. We focused in particular on zirconium based MOFs as they have been used extensively for catalysis. Molecular modeling can offer a rich toolbox to understand the properties of materials and how these properties can be tuned to target specific applications. This is especially true in MOF catalysis, where the active sites are often complex and arise from disorder in the material that cannot be easily tracked experimentally. We especially focus on the UiO--66 material, which is an example of an extremely stable MOF that due to its exceptional connectivity can be easily tuned and undergo modifications without losing its crystallinity. In close collaboration with experimental partners, we gained insights into the nature of active sites and the interaction of solvents with the inorganic brick, and on the role of complex events that can be observed when modeling processes at operating conditions. To unravel these processes, it was shown that a complementary modeling approach is mandatory. Within this approach, initial understanding of the local structure of the active sites is obtained from static calculations, and subsequently more advanced techniques are employed to investigate realistic reaction environments. First principles MD simulations in presence of a realistic loading of solvent have been used, as well as enhanced sampling techniques that allow to study events that take place at activated conditions.
\npar
The high thermal, mechanical and chemical stability of UiO--66 allows the material to undergo different activation processes. In this thesis, we focused on the decrease in the zirconium coordination caused by missing linker defects and dehydration. Changes in the zirconium coordination number are necessary to induce catalytic activity of the material.
\npar
At the start of this doctoral research work, we investigated the Fischer esterification in close synergy with the experimental group of Francesc Llabr\'es i Xamena of ITQ  (\textbf{PAPER I}). This Lewis catalyzed reaction was modeled on UiO--66, initially focusing on the role of the hydration state of the defective sites of the material. We used a series of static calculations, first on small cluster models, then taking into account the periodic MOF structure, with thermal corrections. We show that in the most stable configuration, water is coordinated to the defective zirconium atoms, in line with previous reports. In the proposed reaction mechanism, the chemisorbed water acts as a Br\o{}nsted base site that actively takes part in the reaction, along with the zirconium Lewis acid site. 
It was an eye opener to discover that the catalytic centers are not restricted to Lewis sites in this material. 
An alternative mechanism was investigated on the dehydrated material, where the reaction proceeds without the assistance of water. We show that in this case, the inorganic brick is actively involved in the reaction, in which zirconium plays a role as Lewis acid site, while $\mu_{3}$-O acts as Br\o{}nsted base. However, the beneficial role of water is missing, and the catalytic activity is remarkably decreased due to higher energy barriers, in agreement with the experimental findings. 
In both proposed mechanisms, UiO--66 acts as dual Lewis/Br\o{}nsted catalyst where reactions proceed with a remarkable interplay between the two catalytic centers. The acidic/basic character of protons and oxygens on the brick seems to have a major role, and could be further investigated by pKa calculations. Experimentally by potentiometric titration it was shown that in the defective material three types of protons are present, each possessing different acidity. Proton transfers are a fundamental step in many reactions, and the acidity of Br\o{}nsted sites can alter both reaction mechanism and yield. Further theoretical studies could allow to understand the dependency of acidity of various protons of zirconium materials on different conditions and their reactivity.
\npar
A remarkable property of UiO--66 is that it can be reversibly dehydrated upon thermal treatment. Dehydration also lowers the coordination number of zirconium up to 6, therefore introducing Lewis functionalities. In \textbf{PAPER III}, we followed on the fly the behavior of the material during the activation process of dehydration. We discovered that UiO--66 possesses an intrinsic dynamic nature by means of an exceptional linker mobility. We showed that linkers can decoordinate and recoordinate to the brick without disrupting the stability of the material.
\npar
Apart from UiO--66, a series of other zirconium based materials have been synthesized with larger pore sizes and a varying degree of zirconium coordination number. These materials are fully tested and exploited for applications, however it remains an open question in how far concepts found in the stable UiO--66 are generic for other materials. To this end, some of the information obtained on UiO--66 can in principle be extended to other zirconium--MOFs possessing the same inorganic SBU. However, the behavior of UiO--66 cannot be easily generalized, as the exceptionally high connectivity and the local environment can play a major role in the stability and activity of the material. 
%In \textbf{PAPER VI} w
We further investigated how MOF--808 responds to similar activation processes such as the UiO--66 dehydration. We show that the MOF--808 material possesses an exceptional amount of Br\o{}nsted sites that can be created upon hot filtration. By thermal activation, Lewis sites can be created in proximity of the Br\o{}nsted sites without disrupting the stability of the structure, offering great potential for future catalytic applications. However, we also observe that further dehydration generates a collapse of the framework and loss of crystallinity. In MOF--808 the coordination of zirconium cannot be decreased below 7 without causing a collapse of the structure.
\npar
Another process that can influence the catalytic activity of the material is the introduction of linker functionalization, that can be done by PSE procedures. In \textbf{PAPER IV}, we studied the PSE mechanism by means of static periodic calculations at various defect composition and concentration. The work was performed in strong collaboration with the group of Rob Ameloot from KU Leuven. Experimental findings showed the active role of methanol during the process, which can be performed at mild conditions. From the computational and experimental insight, we postulate the presence of dangling linkers as intermediate states during the exchange, that are induced and stabilized by interaction with methanol. These structural rearrangements in the material lead to the replacement of BDC with BDC--\ce{NH2}. Moreover, the modified material is characterized by a specific concentration of defects that does not depend on the initial number of missing linkers or nodes. This gives insight on the low energy barriers associated to the process and on the role of methanol in facilitating the exchange. 
\npar
Experimentally, the presence of amino functionalized linkers that can be introduced by PSE was shown to have a beneficial role in the catalytic performance of the material for different Lewis catalyzed reactions. However, the cause of this improvement was not well understood. For this reason, in \textbf{PAPER I} we further focused on the role of BDC--\ce{NH2} functionalization on the catalysis of Fischer esterification. The electron--donating amino groups are shown to decrease the Lewis acidity of the metal centers, therefore should in principle lower the activity. However, we found that amino groups play a positive role in the reaction although not actively taking part in the mechanism. Stronger adsorptions of reactants and lower energy barriers are observed for the functionalized UiO--66 compared to the pristine material. Investigation of the reactants geometries and experimental results pointed towards an increase on water adsorption around the active site due to a stronger network of hydrogen bonds supported by the amino groups. This stabilization could play a positive role in facilitating specific reactant geometries and supporting metastable intermediate configurations. Theoretical modeling of Fischer esterification performed in \textbf{PAPER I} shed light on the crucial role of water in the reaction and on the complex nature of the active sites and their interaction with protic solvents. These results already hint towards the impact of solvent species in the pores of the material on the reaction outcome.
\npar
Reactions in MOFs are often performed at mild conditions where reactants are in the liquid phase, which adds further complexity to the model. For this reason, the understanding of the role of solvent close to the active site cannot be neglected. Solvent can have an impact on the creation of active sites, be a substrate, determine the formation of particular isomers, influence the rate and selectivity, affect the reaction mechanism, or even activate/deactivate a specific reaction. Moreover, the M--L bonds in MOFs are more labile than in other heterogenous catalysts composed of covalent bonds. This makes MOFs more prone to structural rearrangements and modifications mediated by protic solvents. In order to move towards an operating description of the processes, it is crucial to understand the behavior of the solvent when confined in the pores of the material. 
\npar
The role of solvent was investigated in \textbf{PAPER II} and \textbf{PAPER V}, in which a multilevel modeling approach was employed to study the behavior of defective UiO--66 in presence of a full loading of methanol and water. By means of GCMC simulations the pores of the material were filled with methanol or water solvent, to reproduce the temperature and pressure conditions during a reactive process. The confined solvent behaves differently from the bulk when confined into the pores of the material. Besides pores, UiO--66 possesses both hydrophobic and hydrophilic regions. By performing a series of MD simulations, in \textbf{PAPER V}  we show that on the one hand, the linkers provide hydrophobic confinement, and on the other, solvent strongly interacts with the inorganic SBUs and in particular with the zirconium atoms on the defective sites. 
\npar
We further investigated the possible interactions between solvent and active sites in the case of methanol from a mechanistic point of view (\textbf{PAPER II}). We observed a remarkable dynamic behavior of the solvent. Supramolecular structures stabilized by hydrogen bonds are formed around the active sites and evolve dynamically during the simulations. Within these active sites, a dynamic behavior of the protons is observed, as anticipated in \textbf{PAPER I}. Such proton transfers may play a substantial role in catalytic processes like Fischer esterification, or in structural rearrangements which occur during PSE. Moreover, we show how solvent can transfer protons within the pores and stabilize charged configurations in proximity of the active site. These findings point towards a positive role of protic solvents in reactions involving charged intermediates, that goes beyond simple solvation. These simulations shed light on how the active sites may be modulated via these dynamic interactions. However, regular MD simulations do not allow to track activated processes.
\npar
During regular MD simulations, the coordination number of zirconium atoms does not change. Nevertheless, activated processes often involve coordination changes of zirconium metal centers. In order to understand the dynamic interactions between solvent and material on these sites and explore the rare events that can take place, we relied on enhanced sampling simulations. We performed a series of independent MTD simulations where we explore the changes in the coordination of the zirconium at operating conditions. These simulations show that the total coordination number of zirconium can either decrease from 8 to 7 or increase from 8 to 9. Undercoordination is crucial for the creation of Lewis acid sites while overcoordination can trigger linker decoordination. In these two events we observe dynamic exchange of solvent around the active sites, as well as structural rearrangements which are characteristic of PSE processes. The strong interaction between solvent molecules and defective sites can lead to a dynamic behavior of the linker, similarly as what observed in \textbf{PAPER III}, but this time at milder conditions, by solvent mediation and without altering the brick. These findings shed light on the dynamic interplay between a protic solvent and the UiO--66 material at operating conditions during PSE. Moreover, one can postulate that these active sites may be dynamically opened for catalysis by temporary linker decoordination, giving a variable Lewis acidity to the metals and more opportunities for Lewis catalyzed reactions. The highly connected UiO--66 material allows all these structural rearrangements where a plethora of active sites that can work in synergy can be generated.
\npar
The findings reported in this thesis gave valuable insight into the complexity of the Zr--MOF catalysts, with UiO--66 as main case study. We investigated how the presence of disorder and the interactions with solvent lead to the creation of active sites of different nature that can work in a cooperative fashion during Lewis catalyzed reactions. There is an intriguing interplay between solvent and material, and much work is still to be done in this area of research which is in full exploration. 
The exceptional chemical versatility of MOFs can be exploited only if we gain enough understanding at the molecular level of their properties at operating conditions. The analysis of the relation between intrinsic properties and desired application have become an important research factor. The use of well--studied model systems such as UiO--66 is therefore crucial to connect theoretical and experimental findings and to understand the sources of errors in the models. The methods used in this thesis reveal molecular level insight into the activation of one of the most stable Metal-Organic Frameworks. They open perspectives to study and benchmark new materials, and further explore the role of active sites and disorder at operating conditions. 
\npar
In the past years, computational research in MOFs has evolved from providing simple representations of model systems to more complex models that can represent the events that can occur at operating conditions. The growth in computational power will allow to access larger time and length scales, and to make models that can represent processes at realistic conditions more accurately, especially in complex fields such as heterogeneous catalysis (Figure \ref{fig:perspective}). As the modeling field matured, we went from a stage in which it was limited to the rationalization of experimental data to being able to work in synergy with experiments and make new predictions. The further step is the understanding of how computational modeling can help to design active sites and the screening of structural databases to target a specific property, that can then be validated by experiments. The multiple theoretical and experimental tools that have been developed so far complement each other on the investigation of complex phenomena. For this reason, the use of databases in which different computational and experimental results can be stored and accessed will be more and more important in the future of MOF research. Therefore, the publication of data cannot be limited to successful experiments, but has to be extended to failures, which also carry precious information. This way, with more data available, it will ultimately be possible to use statistical models to predict structures that match specific properties. The new theoretical predictions will be able to guide new experiments, that can in turn provide new data, and ultimately lead to materials and conditions that can target specific applications.

\begin{figure}[!htbp]
%\vspace{-2cm}
	\centering
	\includegraphics[width=1\textwidth]{perspective}
	\caption{Different stages of the close relationship between modeling and experiments.}
	\label{fig:perspective}
\end{figure}