\graphicspath{{figures/}}
% Header
\renewcommand\evenpagerightmark{{\scshape\small Conclusions and perspectives}} 
\renewcommand\oddpageleftmark{{\scshape\small Chapter 4}}

%\renewcommand{\bibname}{References}

\hyphenation{}

\chapter[Conclusions and perspectives]%
{Conclusions and Outlook}
\label{ch4}
In this work of thesis, we shed light on the molecular characterization of the active sites in MOFs by means of different computational techniques. Molecular modeling can offer a rich toolbox to understand the properties of materials and how these properties can be tuned to target specific applications. This is especially true in MOF catalysis, where the active sites are often elusive and arise from disorder in the material that cannot be easily tracked experimentally. We particularly focus on UiO--66, which is an example of an extremely stable MOF that due to its exceptional connectivity can be easily tuned and undergo modifications without losing its crystallinity. In close collaboration with experimental partners, we gained insight into the properties of this promising material, and on the role of complex events that can be observed when modeling processes at operating conditions.
\npar
In \textbf{PAPER I}, a Lewis catalyzed reaction was modeled on UiO--66 in close synergy with experimentalists, focusing on the role of the hydration state of the defective site and the functionalization of the material. From theoretical models, in agreement with previous reports, we show that in the most stable configuration, water is coordinated to the defective zirconium atoms. In the proposed reaction mechanism, the chemisorbed water acts as a Br\o{}nsted site that actively takes part in the reaction, along with the zirconium Lewis acid site. An alternative mechanism was investigated on the dehydrated material, where the reaction proceeds without the assistance of water. We show that in this case, the catalytic activity is remarkably decreased due to higher energy barriers, in agreement with the experimental findings. In both proposed mechanisms, UiO--66 acts as dual Lewis/Br\o{}nsted catalyst where reactions proceed with a close synergy between the two catalytic centers. 
\npar
The role of linker functionalization was also investigated. We show that amino groups play a positive role in the reaction although not actively taking part in the mechanism. Even if the Lewis acidity of the metal centers is decreased, stronger adsorptions and lower energy barriers are observed for the functionalized material. Such counterintuitive result is an indication of the complex nature of the active site on Zr--MOFs. Investigation of the reactants geometries and experimental results points towards an increase on water adsorption around the active site due to a stronger network of hydrogen bonds supported by the amino groups. This stabilization could play a positive role in facilitating certain reactant geometries and supporting intermediate configurations. Theoretical modeling sheds light on the crucial role of water in the reaction and on the complex nature of the active sites and their interaction with protic solvents. Defective UiO--66 and its variants have a high potential as catalysts due to their dual acid/base character and the facile way by which its composition can be tuned. This material can be considered as a prototype MOF that can be designed to possess all the ingredients needed for a dual Lewis/Br\o{}nsted catalyst. For this reason, understanding of the role of solvent close to the active site as substrate for proton transfers cannot be neglected. Solvent molecules could play an active role in other UiO--66 catalyzed reactions, as well as in other Zr--MOF, such as NU--1000 or MOF--808, which possess the same inorganic SBU. 
\npar
In \textbf{PAPER VI} we further investigated how MOF--808 responds to similar activation processes such as the UiO--66 dehydration of \textbf{PAPER I}. We show that the material possesses an exceptional amount of Br\o{}nsted sites that can be created upon hot filtration. By thermal activation, Lewis sites can be created in proximity of the Br\o{}nsted sites without disrupting the stability of the structure, offering great potential for future catalytic applications. However, we also observe that further dehydration generates a collapse of the framework and loss of crystallinity.
\npar
Much of the insight gained on reactive processes in MOFs stems from the more established computational research done in zeolites. Unlike processes in zeolites, however, reactions in MOFs are often performed at milder conditions in presence of a liquid solvent, that adds further complexity to the model. Moreover, the M--L bonds in MOFs are more labile than the covalent bonds in zeolites and this makes these materials more prone to structural rearrangements and modifications. It is a challenge to understand in how far the description of the active site corresponds to reality, and for this reason, it is imperative to understand the behavior of the solvent in the pores of the material. Solvent can have an impact on the creation of active sites, be a substrate, determine the formation of particular isomers, influence the rate and selectivity, affect the reaction mechanism, or even activate or deactivate a specific reaction.
\npar
The role of solvent was further investigated in \textbf{PAPER II}, in which a multilevel modeling approach was employed to study the behavior of defective UiO--66 in presence of solvent methanol. Starting from the insight gained in \textbf{PAPER I}, we further investigated the possible adsorption geometries for water and methanol. We then filled the pores of the material with methanol solvent, to replicate the temperature and pressure conditions during a reactive process. We observed a remarkable behavior of the solvent. Supramolecular structures formed by hydrogen bonds are formed around the active sites and evolve dynamically during the simulations. Within these active sites, a dynamic behavior of the protons is observed, as anticipated in \textbf{PAPER I}. Such proton transfers may play a substantial role in catalytic processes, such as Fisher esterification, or in structural rearrangements like during PSLE. Moreover, solvent can transfer protons on longer distances within the pores and charged configurations in proximity of the active site can be stabilized, pointing towards a positive role in reactions involving charged intermediates, that goes beyond simple solvation. These simulations shed light on how the active sites may be modulated via these dynamic interactions. 
\npar
In \textbf{PAPER IV}, we studied the PSLE mechanism and showed the active role of methanol during the process. From the computational and experimental insight, we postulate the presence of dangling linkers as intermediate states during the exchange, that are induced and stabilized by interaction with the methanol solvent. These structural rearrangements in the material lead to the replacement of BDC with BDC--\ce{NH2} and a resulting structure characterized by a specific concentration of defects that does not depend on the initial defectivity. This gives insight on the low energy barriers that characterize the process.
\npar
\textbf{PAPER III} unraveled the exceptional intrinsic dynamic nature of the UiO--66 material. At dehydration conditions, the material showed reversible mobility of the linkers induced by the brick dehydration. We show that linkers can decoordinate and recoordinate to the brick without disrupting the stability of the material. In \textbf{PAPER V}, we further examine the interaction between a protic solvent and material during the structural rearrangements of the PSLE process. By means of a series of MD simulations, we identified a hydrophobic and a hydrophilic region in the material. We show that on the one hand, the linkers provide hydrophobic confinement, on the other, solvent strongly interacts with the inorganic SBUs and in particular with the zirconium atoms on the defective sites. To understand the dynamic interaction between solvent and material on these sites and explore the events that can take place, we performed a series of independent MTD simulations, where a change in the coordination of the zirconium atoms is induced. These simulations show that Lewis acid sites can be created by decoordination of water, but we also report overcoordination of the zirconium atoms by adsorption of an additional water molecule. Via these two changes in coordination, exchange of solvent molecules on the active site has been observed. Moreover, structural rearrangements in the material have been induced. The strong interaction between solvent molecules and defective sites can lead to a dynamic behavior of the linker, similarly as what observed in \textbf{PAPER III}, but this time at milder conditions, by solvent mediation and without alteration of the brick. These findings shed light on the dynamic interplay between a protic solvent and the UiO--66 material at operating conditions during PSLE. Moreover, one can postulate that these active sites may be dynamically opened for catalysis by temporary linker decoordination, giving a variable Lewis acidity to the metals and more opportunities for Lewis catalyzed reactions. The highly connected UiO--66 structure allows all these structural rearrangements where a plethora of active sites that can work in synergy can be generated without losing its crystallinity.
\npar
The findings reported in this work of thesis shed light into the complexity of the Zr--MOF catalysts, with UiO--66 as main case study. We investigated how the presence of disorder and the interactions with solvent lead to the creation of a plethora of active sites that can work in close synergy to bring a beneficial effect in Lewis catalyzed reactions. The exceptional chemical versatility of MOFs can be exploited only if we gain enough understanding at the molecular level of their properties and how these properties can be affected and tuned. Knowing the sources of error is imperative, and for this reason, theory and experiments have to go hand in hand. For this purpose, the use of well--studied model systems such as UiO--66 is crucial. In the past years, computational research in MOFs has evolved from providing simple representations of model systems to more complex models that can represent the events that can occur at operating conditions. The growth in computational power will allow to access larger time and length scales, and to make models that can more and more represent processes at realistic conditions, especially in complex fields such as heterogeneous catalysis. The further step is the understanding of how computational modeling can help to design active sites and how databases of possible structures can be screened to target a specific property. The multiple theoretical and experimental tools that have been developed represent multiple points of view by which a certain phenomenon can be investigated. For this reason, the creation of databases in which different computational and experimental results can be stored and accessed will be more and more important in the future of MOF research. 
