\graphicspath{{figures/chapter3/}}
% Header
\renewcommand\evenpagerightmark{{\scshape\small Major research results}}
\renewcommand\oddpageleftmark{{\scshape\small Chapter 3}}

%\renewcommand{\bibname}{References}
\hyphenation{}
\chapter[Major research results]%
{Major research results}
\label{ch3}
This Chapter illustrates the main research results obtained in the framework of this thesis. The main goal of this work was the study of the nature of active sites on UiO--66 and MOF--808 upon activation processes, and how the solvent and the functionalization influenced their behavior. The role of active sites on defective UiO--66 was studied for Fischer esterification of free fatty acids (FFA), at it became clear that solvent played an unexpected active and beneficial role in the reaction mechanism. Different molecular modeling techniques based on static and dynamic methods have been applied to gain insight into the interaction between active sites and reactants in the material, as has been introduced in Chapter 2. Contrary to reactions in zeolites, processes in MOFs are often performed ad mild conditions in the presence of a solvent, which adds complexity to the model. So far, solvent in MOFs had been studied only with classic approaches, but with the increase in computational power it was possible to include a full \textit{ab initio} treatment of water and methanol solvent in the UiO--66 pores and to go towards an operando description of activation processes in MOFs. The most important scientific results will be highlighted in this Chapter. More details are to be found in the original articles, enclosed in Part II.

\section{Activation processes in zirconium--MOFs}
One of the main challenges in MOF research is the understanding of how active sites are created and how they impact the properties of the material. For this purpose, molecular modeling offers a platform that allows to study such activation processes and nature of active sites at the molecular level. In this sense, UiO--66, characterized by an exceptional stability, is the perfect MOF archetype where different activation and PSM processes can take place without disrupting the stability of the structure. This MOF represents a showcase example, as the findings can in principle be extended to other MOFs that are more difficult to study.
%%
\begin{figure}[!htbp]
%\vspace{-2cm}
	\centering
	\includegraphics[width=1.0\textwidth]{UiO-66-activation}
	\caption{Schematic representation of the UiO-66 structure with possible configurations of the bricks that give rise to coordinatively unsaturated Zr atoms. The colors indicate the coordination of the Zr atoms.}
	\label{fig:UiO-66-activation}
\end{figure}
%%
\subsection{Missing linker defects}
In \textbf{PAPER I} we investigated the local defect topology upon removal of a linker on UiO—66 as starting point to investigate the catalytic role of defect coordinating species in the mechanism of Fischer esterification. When studying the catalytic behaviour of UiO--66, an essential ingredient to understand the reactivity of the inorganic SBU is the knowledge of the molecular structure of the active sites upon removal of a linker. On these sites, defect coordinating species can be adsorbed and have an impact on the catalytic properties of the MOF by introducing additional sites that can play an active role in reactions. 
%%%%%%
\begin{figure}[!htp]
%\vspace{-2cm}
	\centering
	\includegraphics[width=1.0\textwidth]{adsorption-water}
	\caption{Coordination free energies at reaction temperature of 351 K of one, two and three water molecules at coordinatively unsaturated Zr-bricks in defective UiO-66 with respect to a water coordination free site (site R). The structure of the opposite site B corresponds with configuration 2’ with two water molecules and consistently used in all periodic calculations considered in the figure. Free energies (in black) are given in kJ/mol, and their decomposition into enthalpic $\Delta$H (blue) and entropic -T$\Delta$S (grey) contributions. Energies are resulting from periodic calculations with PBE-D3 level of theory. In each configuration Lewis acid and Brønsted sites are indicated.}
	\label{fig:adsorption-water}
\end{figure}
%%%%%%
As a thought experiment, we can imagine that upon removal of one of the twelve negatively charged BDC linkers from the inorganic \ce{Zr6O4(OH)4} SBU, the positive charge left on the brick has to be compensated. Charge neutralization can be accomplished by either coordinating a negative ion such as hydroxyl group to one of the zirconium atoms, or by removing a proton from the brick. This latter case is characterized by two zirconium open metal sites, and is the type of active site that is obtained upon thermal treatment of the brick at T > 150 C.  It was previously accepted that these Lewis sites were responsible for the catalytic activity of UiO—66, but recent studies point towards the active role of Br\o{}nsted sites in the neighbourhood of defective zirconium atoms \footnote{(Canivet, Fateeva et al. 2014, Ghosh, Colon et al. 2014, Øien, Wragg et al. 2014, Canivet, Vandichel et al. 2016, Ling and Slater 2016, Liu, Klet et al. 2016, Vandichel, Hajek et al. 2016)}. The presence of multiple sites makes it difficult to establish a simple structure--activity relation. In particular, it is important to understand from a mechanistic point of view how the presence of defect coordinating species may affect the catalytic activity on Zr-MOFs. 
This research was done in collaboration with the group of Dr. Francesc X. Llabr\'es i Xamena as experimental partners. The coordination of water species near the active sites can occur with different configuration schematically displayed in Fig. \ref{fig:adsorption-water}. 
In all structures the presence of possible Lewis and Brønsted sites which may play a role in reactions is highlighted. The simplest case taken as reference configuration is the unsaturated site obtained upon thermal activation, containing two µ3-oxygens bridging the Zr-atoms which may act as Br\o{}nsted sites. From this structure, the coordination of one physisorbed water molecule to one of the zirconium atoms is energetically favorable. A decrease in energy is observed when the molecule is deprotonated to the oxo atom as in configuration 1’. The chemisorption of this water molecule shields the Lewis character of the zirconium atom, but introduces an additional Br\o{}nsted site in proximity to the open metal site. The most stable configurations are observed in presence of 2 or 3 adsorbed water molecules on the adjacent zirconium atoms. Two physisorbed water molecules of configuration 2’ can be adsorbed to the two zirconium atoms, followed by an immediate dissociation of one of the two molecules into a hydroxyl group and a proton on the adjacent $\mu_3$ oxygen. This configuration is characterized by a free energy difference of -94.4 kJ/mol at reaction temperature of 351 K compared to the dehydrated defective site. The presence of two zirconium coordinating species has been reported by Lillerud \footnote{cite Oien} by means of Single--crystal X-Ray diffraction (SXRD), showing that the material is most stable when all zirconium atoms are fully coordinated. When starting from the thermally activated material that contains open Lewis sites, at standard condition, water present in the atmosphere will immediately coordinate to restore the 8--fold coordination of the zirconium atoms to give structure 2. This structure is consistent with the one proposed by the group of Farha \footnote{cite Klet Liu 2016} who identified three types of protons from potentiometric titration:  $\mu_3$--OH, \ce{Zr-OH} and \ce{Zr-OH2}. 
A third bridging hydroxyl species as charge neutralizing species was proposed by Yaghi \footnote{cite Trickett} from XRD data. They propose two physisorbed water molecules bridged by an \ce{OH-} counterion that is stabilized by a hydrogen--bond interaction with the neighboring $\mu_3$--OH of the brick. However, a recent study Ling and Slater \footnote{cite Ling Slater} did not succeed in finding a corresponding minimum on the PES. In this work, we observe that the $\mu_3$--OH atom immediately deprotonates in proximity of such \ce{OH-} anion (configuration 3). Similarly as in the previous case, this structure can further stabilized by a deprotonation of one of the two physisorbed water molecules, to yield configuration 3’ of Fig. \ref{fig:adsorption-water}, in agreement with a previous report by Vandichel et al. \footnote{cite Vandichel 2016}. Adsorption of reactants involved on the Fischer esterification reaction was also taken into account. Similar considerations on the deprotonation processes can be drawn when considering methanol instead of water as reported more in detail in \textbf{PAPER I} and \textbf{PAPER II}. The energies obtained clearly demonstrate that water molecules preferentially adsorb on the zirconium atoms and the only limit to the adsorption of more water molecules lies in the entropic penalty. 

\subsection{Nature of active sites for Fischer esterification}
 Among the reactions that can be catalyzed by this MOF, Fischer esterification is an important process in the production of biodiesel, a biofuel that is obtained from renewable sources, such as oils and animal fats. UiO--66 has been shown to be a stable and reusable Lewis catalyst with high conversion rate for the reaction \footnote{cite Cirujano2015, Cirujano2016}. Experimental findings performed on both hydrated and dehydrated UiO--66 show that water has a beneficial role in the reaction, but a theoretical rationalization of the underlying causes was missing. Moreover, amino functionalization was shown to increase the reaction rate. To address these questions, the role of active sites on UiO--66 during the Fischer esterification reaction was studied in \textbf{PAPER I}. Two possible lowest activated reaction pathways were identified for the hydrated and dehydrated active site. \\
\begin{figure}[!htbp]
%\vspace{-2cm}
	\centering
	\includegraphics[width=1.0\textwidth]{esterification}
	\caption{Mechanism and free energy profile for the esterification of propionic acid with methanol on a hydrated and defective UiO-66 material (blue), a hydrated defective UiO-66 material with amino functionalization of the BDC linkers (red), and on a dehydrated defective UiO-66 (black). Periodic calculations at B3LYP-D3//PBE-D3 level of theory, T=351 K. R corresponds with an empty frame with one linker defect and a pool with all reactants to guarantee mass balance. In P the defective Zr-brick is coordinated with two water molecules (configuration 2’). P’ corresponds to the empty frame with the ester as final product and remaining water molecules in gas phase.}
	\label{fig:esterification}
\end{figure}

The proposed reaction mechanism is shown in Fig. \ref{fig:esterification}. From the reactant configuration, the physisorbed water molecule is displaced by the carboxylic acid that coordinates on the zirconium atom. The resulting configuration was identified after a series of static calculations probing possible geometries. In this configuration, the acid carbonyl group is bonded to the zirconium, and at the same time methanol is hydrogen bonded to the hydroxyl group coordinated to zirconium. The adsorption of the acid on the Lewis acid site gives to the carboxylic carbon a more electrophilic character, making it more prone to interact with the alcohol. The oxygen of methanol is at the same time made more nucleophilic due to the hydrogen--bonding interaction with the Br\o{}nsted basic site situated in close proximity. This favors the condensation between activated carboxylic carbon of the acid and methanol. In both transition states, the hydroxyl group plays a role, first as proton acceptor, then as proton donator, while the acid maintains the bond with the carbonyl oxygen. The low energy barriers associated to the two transition states are 28.9 and 30.6 kJ/mol at 351 K. The reaction can proceed both ways, until an equilibrium is reached. This mechanism is characterized by a dual participation of Br\o{}nsted and Lewis sites, and overlays with the experimental findings. \\

Upon thermal treatment at 150 C, UiO--66 loses the adsorbed solvent molecules without compromising the structure of the inorganic SBU\footnote{cite Valenzano 2016}, contrary to the dehydroxylation with release of two water molecule that takes part at T > 300 C \footnote{cite Valenzano 2011}. In principle, these open metal sites should be more catalytically active, but a decrease in catalytic activity is experimentally observed. In the proposed mechanism, two Lewis acid sites, the zirconium atoms, and one Br\o{}nsted basic site, the $\mu_3$ oxygen, play an active role in the reaction. This mechanism is characterized by three transition states, in which: 1) methanol is deprotonated to the oxo atom, 2) an adduct is formed between the electrophilic carbon and the oxygen of methanol 3) the $\mu_3$--OH group deprotonates to form water. This reaction is characterized by higher activation barriers (a total barrier of 90 kJ/mol at 351 K), which explain the lower catalytic activity of the dehydrated material.\\

Other mechanisms that do not make use of either Lewis or Br\o{}nsted sites were investigated without success, as the energy barriers were too high to be likely to occur. Both proposed mechanisms are characterized by a dual participation of Lewis and Br\o{}nsted sites that work complementary to each other.  The presence of acid and basic centers within molecular distances has been shown to be essential in the performance of the catalytic reaction as they cooperate in a concerted way during the chemical transformation. Most of the previous mechanistic studies on the UiO-66 material merely focused on the Lewis acidity of the under-coordinated sites, but it has become more an more clear that the bifunctional nature of the UiO--66 catalyst will play an important role in its future applications.

\subsection{Role of defect topology}
The adsorption of species present in the reaction environment of PSLE process was studied in the framework of \textbf{PAPER IV}. In this process, we wanted to see how defect--free and defective UiO--66 material would interact with water and methanol species. To obtain this information, we made use of a larger unit cell, to be able to represent different amount of missing linker defects. This increase in complexity reflects in the higher computational cost of modeling such structure. The defective structures taken into account in the modeling are obtained by removal of linkers from the conventional unit cell containing four inorganic \ce{Zr6(O)4(OH)4} bricks \footnote{cite cavka 2008}. Different numbers of missing linkers were modeled, to represent the different amount of defects in the experimental samples. A case with a low amount of defects is modeled by a unit cell with one missing linker. In this structure, two bricks are 12--fold coordinated, and two bricks are 11--fold coordinated, with an average of 11.5 linkers per brick. A second type of unit cell was taken with three missing linkers, corresponding to a higher amount of defects, with an average of 10.5 linkers per brick. De Vos et al. and Rogge et al. \footnote{cite Rogge, De Vos} shown in their comprehensive studies that there are multiple topologically diverse possibilities to remove linkers from a 4--brick unit cell. In this work, we chose two possible, discrete cases to represent different coordination of the inorganic brick with this amount of defects. In a first unit cell with three missing linkers, two bricks are 10--fold coordinated and two are 11--fold coordinated. The other represents a more extreme case, with one 9--fold coordinated brick and three 11--fold coordinated bricks. The structures taken into account are denoted as $\mathrm{(10_{b}, 10_{b}, 11, 11)_{334}}$ and $\mathrm{(9_{c}, 11, 11, 11)_{333}}$ in the work of De Vos et al. 
%
\begin{figure}[!htbp]
%\vspace{-2cm}
	\centering
	\includegraphics[width=1.0\textwidth]{psle-capping}
	\caption{Energy diagrams for defective UiO-66 unit cells. Each dot represents a possible distribution of missing-linker defects (1 or 3 in total) within the unit cell; the connecting dotted line represents a weighted average. Values are normalized by the number of missing linkers in the unit cell. (a) Enthalpy difference between defect sites capped in different ways versus the non-defective material at T = 298, 313, 373, 473 K. (b) Temperature-dependence of the free energy difference of the defective structures indicated above (capped with \ce{H2O/OH-}, \ce{H2O/MeO-} and \ce{MeOH/MeO-}) versus the non-defective structure. A representation of the clusters with different missing linker connectivities is also provided.}
	\label{fig:psle-capping}
\end{figure}
%
The zirconium atoms on the defect sites have been capped by all species which can be present in the framework during the activation process, studying different combinations by means of water, hydroxo species, methanol, methoxide and formate. In our models, all Zr atoms have been capped and are fully (8--fold) coordinated, being the most stable state at experimental conditions. On each defect site there is always a negatively charged species, to compensate for the removal of a carboxylate. Given the symmetry of the unit cell, when possible, multiple choices for the positioning of these species were taken into account and modeled. The removal of a negative charge can be also compensated by removal of a proton, as obtained in the dehydrated material, with subsequent physisorption of two neutral species, but these configurations are higher in energy, as seen in \textbf{PAPER I}. These results show a clear enthalpic preference for the methanol/methoxide pair in detriment of formate and water. The defective material synthesized with formate modulator is promptly attacked by methanol species present in solution. The obtained results are in line with a previous report \footnote{cite Yang et al 2016} and highlight the preference of zirconium active sites for MeOH/MeO- substitution as opposed to H2O/OH-, alternatively assumed as preferential charge balancing element for missing linker defects \footnote{cite Trickett 2015, Ling/Slater 2016}.\\

No substantial difference in enthalpy is observed between the unit cells with different amount of defects (energy in each defective structure represented by a dot in Fig. \ref{fig:psle-capping}). This is an indication that the thermodynamics that governs the process does not depend on the starting number of missing linkers, in agreement with the experimental observations of \textbf{PAPER IV}. The effect of temperature on the free energy was also investigated on the relative stability of defective configurations with respect to the pristine unit cell, where the linker is not missing. The first observation is that configurations involving water and methanol are more stable at lower temperature, thus the formation of missing linker defect through ligand exchange is enthalpically driven and favored by lower temperatures. Secondly, at 473 K the energy of the non-defective framework is much lower than the defective cases, which explains why the synthesis of the non-defective UiO--66 is favorable at such high temperature\footnote{cite the one on synthesis}.

\subsection{Activation by dehydration}
The other process that can lead to activation of the UiO--66 material is the reversible dehydration of the brick performed at T $>$ 523 K.  The dehydration mechanism may have a decisive effect on certain catalytic reactions, where next to the Lewis acid site also the neighboring Br\o{}nsted base or acid site may take a cooperative role in the reaction mechanism. At elevated temperatures and low pressures, the Zr core gets dehydrated and rearranged, and two water molecules are subsequently removed from the brick \ref{fig:UiO-66-activation}. The fully dehydrated \ce{Zr6O6} brick contains coordinated vacancies \footnote{cite Valenzano 2011, Shearer 2013, DeCoste 2013, Vandichel 2015}, with coordination of the zirconium atoms ranging from 8 to 6. The structure of UiO--66 is however preserved and the brick can easily be rehydrated with a reversible mechanism, giving evidence that the inorganic SBU can undergo dynamic processes. During these structural rearrangements, the presence of a band related to hydroxy groups was observed by Nishida by means of ultrafast IR\footnote{cite Nishida 2014}. The mechanism of dehydration was first studied by Vandichel et al. by means of nudge elastic band calculations \footnote{cite Vandichel 2016}, showing the presence of loose hydroxyl groups and partially decoordinated linkers. These findings point towards an intrinsic mobility of the framework structure, that can accommodate rearrangements without disrupting its stability. However, the static study of this process possesses serious limitations. In \textbf{PAPER III}, we follow on the fly the fast dynamic of the UiO--66 material at dehydration temperature of 573 K by means of umbrella sampling simulations.  The results show an intrinsic dynamic behavior of the material, with open metal sites being created by continuous changes in the network connectivity due to labile M--L bonds. We identify two types of motions of the linkers, namely translation along the axis connecting the two adjacent zirconium atoms, and rotation along the linker axis connecting the two inorganic SBUs, as shown in Fig. \ref{fig:translation-rotation}. At the same time, these motions are accompanied by a high mobility of hydroxy groups created by decoordination of the $\mu_3$--OH groups. These results show that linkers are more mobile than originally anticipated. The high connectivity between the two SBUs of UiO--66 allows all these reversible rearrangements with activation barriers that can be easily accessible at experimental conditions. 
\begin{figure}[!htp]
%\vspace{-2cm}
	\centering
	\includegraphics[width=1.0\textwidth]{translation-rotation}
	\caption{Umbrella sampling in two windows of CV = 1.45 and 1.51, showing two distinct motions of the linkers. On the left column, a translation of the linker L1 generates a chelated structure and a subsequent shift in the carboxylic oxygen connected to Zr2 (configuration 4t). On the right column, a rotation of the linker L1 and a partial decoordination of linker L2 forming a hydrogen bond with an $\mu_3$--OH hydroxyl group is shown (configuration 4r). A proton transfer between the carboxylic oxygen O3 and the bridging $\mu_3$--O is also observed and is an indication of the occurrence of an intrinsic dynamic acidity. Colors indicate the coordination number of Zr atoms.}
	\label{fig:translation-rotation}
\end{figure}

\subsection{Thermal activation of MOF--808}
The active sites and coordination changes upon thermal activation have been studied in MOF--808 in \textbf{PAPER VI}. MOF--808 shares the \ce{zr6o4oh4} brick with UiO--66, but the brick is connected to only six tritopic BTC linkers, making it the least connected MOF in the Zr--MOF family. The material possesses a high catalytic potential due to the intrinsic presence of defective sites of complex nature.  In the as synthesized material (Fig. \ref{fig:MOF-808-dehydration}), the zirconium atoms that are not connected to BTC linkers are capped by formate groups. The material is activated by hot filtration, whereby each formate is replaced by water and hydroxyl group, for a total of six water molecules and six hydroxyl groups per inorganic SBU. Each pair of adjacent zirconium atoms has a similar configuration as the stable configuration 2 of Fig. \ref{fig:adsorption-water}, but the presence of multiple Br\o{}nsted sites in close proximity gives rise to a more complex nature of the active sites. By thermal activation, similarly to UiO--66, water can be removed from the active sites to open Lewis acid sites for catalysis. In a second stage, up to four of the hydroxyl groups could be also decoordinated by extracting a $\mu_3$--OH proton from the brick to form water, giving rise to mixed--coordination bricks with 6 and 7--fold coordinated zirconium atoms. \\
In \textbf{PAPER VI}, the behaviour and stability of the material upon activation processes was investigated by means of a series of independent MD simulations at 300 K at variable unit cell parameters. We show that the dehydration of the inorganic brick substantially affects the stability of the structure. In the hydrated form, the material possesses a high number of Br/o{}nsted sites that show dynamic acidity in the form of proton transfer between water and hydroxyl groups that are located in close proximity. The physisorbed water is removed leading to a homogeneous distribution of 7--fold coordinated zirconium atoms in the brick. Upon this dehydration, only a slight decrease in the unit cell volume is observed (Fig. \ref{fig:MOF-808-volume}), and the structure remains stable, even though it possesses a high amount of undercoordinated sites. Moreover, these Lewis sites are located in proximity to Br\o{}nsted sites arising from hydroxyl groups, making dehydrated MOF--808 a dual heterogeneous catalyst. Further removal of the hydroxyl groups together with the $\mu_3$--OH protons causes a collapse of the structure. Due to the overall low structure connectivity the adsorbed hydroxyl groups have to be considered as inherent part of the framework composition and their separation results in the collapse of the material. In contrast to UiO--66, that can undergo reversible dehydration processes where the coordination of the zirconium atoms can be lowered to 6 without disrupting the structure, MOF--808 cannot sustain such decrease in coordination.

\begin{figure}[!htbp]
%\vspace{-2cm}
	\centering
	\includegraphics[width=1.0\textwidth]{MOF-808-dehydration}
	\caption{Schematic representation of as synthesized and upon activation MOF-808 structures}
	\label{fig:MOF-808-dehydration}
\end{figure}

\begin{figure}[!htbp]
%\vspace{-2cm}
	\centering
	\includegraphics[width=1.0\textwidth]{MOF-808-volume}
	\caption{Change of volume in time of the three investigated structures with different Zr coordination}
	\label{fig:MOF-808-volume}
\end{figure}

\subsection{Linker functionalization on UiO--66}
The following results concern the linker functionalization on UiO--66 and its effects on the catalytic activity of the material. 
The easiness of functionalization of UiO--66 can be exploited to tune its catalytic properties. 

\subsubsection{Post synthetic linker exchange}
As reported in Chapter 1, functionalization in UiO--66 can be induced by PSLE, exploiting the robustness of the framework, which can easily undergo coordination changes. In \textbf{PAPER IV}, the mechanism was investigated in a dual experimental--computational study performed in collaboration with the group of Prof. Rob Ameloot. The PSLE in UiO--66 that leads to functionalization of the BDC linkers with amino groups was performed in methanol at mild conditions (40$^\circ$C). Results show that the initial amount of missing linkers did not have an effect on the final composition of the material, pointing towards low energy barriers for the exchange. Moreover, the process was accompanied by an initial lowering of the BET surface area. In light of these experimental observations and the computational findings reported in Fig. \ref{fig:psle-capping}, a hypothetical exchange mechanism was modeled starting from both pristine and defective material (Fig. \ref{fig:psle-dangling-linker}). In the proposed mechanism, linker exchange is initiated by coordination of methanol on the zirconium sites, causing a partial hydrolysis of one of the BDC linkers. This metastable state is characterized by a dangling linker which is stabilized by hydrogen bonds with other carboxylic oxygen atoms and the $\mu_3$--OH hydrogen from the brick, in a similar fashion as what observed in \textbf{PAPER III}. This structure causes a hindrance of the pore in agreement with the reduction in BET surface area. From this configuration, exchange of mono-coordinated BDC linkers with \ce{NH2}-BDC quickly ensues, with partial coordination of the new linker. In a following step, methanol is desorbed from the active site and the linker can connect to the zirconium atoms, reestablishing the binding between the two bricks. This preferential adsorption of BDC--\ce{NH2} is explained by the lowering of the enthalpy of about 7 kJ/mol per linker. When starting from a defective material, a similar mechanism can be proposed. In this second scenario, the synthetized material contains formate on both defective sides. Formate will be substituted by methanol present in solution, and the defect healing can proceed in a similar way as in the previous case. All these rearrangements are supposedly characterized by low activation barriers, as the final composition of the material does not depend on the initial percentage of functionalized or missing linkers. During the process, the zirconium atoms rapidly change their coordination number, however, giving preference to 8-fold coordinated active sites that enhance the stability of the structure. 
\begin{figure}[!htbp]
%\vspace{-2cm}
	\centering
	\includegraphics[width=1.0\textwidth]{psle-dangling-linker}
	\caption{Proposed PSE mechanism in non-defective and defective UiO--66. MeOH facilitates ligand exchange through the creation and stabilization of defects. Enthalpy differences are given in kJ/mol at 313 K (PSE temperature)}
	\label{fig:psle-dangling-linker}
\end{figure}

\subsubsection{Effect of linker functionalization on Fischer esterification in UiO--66}
In order to understand the effect of the functionalization of the BDC linkers on the catalytic activity of defective UiO--66, the esterification reaction (Fig. \ref{fig:esterification}) was investigated also on the amino functionalized UiO--669-\ce{NH2}.  Amino functinalization on UiO--66 brings an electrodonatin group that would in principle lower the acidity of the Lewis acid site. However, experimental insights into the Fischer esterification reaction showed that the amino functionalized UiO--66 was more catalytically active than its non--functionalized counterpart. For this reason, it was speculated that amino groups located in proximity to the defect sites would play an active role during the reaction. Our computational results show no possible reaction pathway that involves proton transfers with the amino groups. We therefore studied the reaction following the same pathway that we propose for the hydrated brick (Fig. \ref{fig:esterification}) in the non--functionalized material. The decrease in Lewis acidity upon amino functionalization is indeed confirmed by the increase of the Zr--O distances of the adsorbates \footnote{cite Vermoortele, Vandichel et al. 2012}. However, for UiO--66--\ce{NH2}, stronger stabilization of the adsorbates is observed, as well as a slight decrease in the overall energy barrier of about 11 kJ/mol, which confirms the experimental findings \footnote{cite Cirujano 2015, Cirujano 2015}. The amino groups, although not playing an active role in the reaction mechanism, indirectly modulate the properties of the Lewis and Br\o{}nsted sites. A stronger adsorption of the reactants is caused by the formation of a network of hydrogen bonds with the amino groups that cannot be seen in the pristine material. Moreover, amino groups provide additional sites where solvent can form hydrogen bonds, as displayed in Fig. \ref{fig:ester-amino}. This indirect positive effect of amino groups on the catalytic properties was observed as well by Hajek et al. for aldol condensation \footnote{cite Vandichel, Hajek et al. 2015}. The effect of other functional groups was further analyzed (see Supplementary Material of \textbf{PAPER I}). Electron--withdrawing substituents such as --\ce{NO2} did not decrease the energy barriers, although increasing the Lewis acidity of the metal. Once again, this confirms the dual Lewis/Br\o{}nsted character of the UiO--66 catalyst, that can be enhanced by the presence of additional sites within molecular distance. 
\begin{figure}[!htbp]
%\vspace{-2cm}
	\centering
	\includegraphics[width=0.7\textwidth]{ester-amino}
	\caption{Network of hydrogen bonds between acid, methanol, hydroxyl group and amino groups. a) reactive complex 8, b) an additional water molecule present in solution.}
	\label{fig:ester-amino}
\end{figure}

\section{Role of solvent: towards modeling at operating conditions}
So far, a simple model to represent the active sites was used, where only the species immediately coordinated to the zirconium atoms were considered. A more complex model can be constructed by including microsolvation, with additional solvent molecules. However, the previous findings point towards a complex nature of the active sites in the material, where solvent may play an active role in reactive processes and structural modifications. For this reason, in \textbf{PAPER II} and \textbf{PAPER V} we went a step forward in complexity by including an explicit treatment of the solvent in the pores of the UiO--66 material. 
This increase in complexity of the model allows to take into account the processes that can occur at operating conditions, and to take into account the full solvent environment within the pores at realistic temperatures and pressures. 

\subsection{Interaction between UiO--66 and solvent}

\begin{figure}[!htbp]
%\vspace{-2cm}
	\centering
	\includegraphics[width=1\textwidth]{vibrational}
	\caption{Top: schematic representation of the empty pore, pore with the solvent, and confined solvent without the material. Bottom: vibrational density of states obtained from the velocity autocorrelation function power spectra of selected atoms of the simulation. Bottom left: solvated material compared to the empty material. Bottom right: water in the pores compared to bulk water.}
	\label{fig:vibrational}
\end{figure}

\begin{figure}[!htbp]
%\vspace{-2cm}
	\centering
	\includegraphics[width=1\textwidth]{gr-water}
	\caption{Radial distribution functions or pair correlation functions g(r) between oxygen and hydrogen of water (O(w), H(w)), and different atoms of the material obtained from the simulation with 80 water molecules in the unit cell. Full lines indicate the g(r), dashed lines indicate its integral. Left panel: RDFs between water and linker carbons and hydrogens (C(l), H(l)); middle panel: RDFs between water and oxygen atoms of the linkers and bricks (O(l), µ3-OH, µ3-O); right panel: RDFs between water and zirconium atoms of defective and pristine bricks (Zr(def), Zr(pris)). }
	\label{fig:gr-water}
\end{figure}

\subsection{Interaction between methanol molecules on the UiO--66 defective site}


%%% Major research results
%%%%%%%%%%%%%%%%
%\newpage
%\newpage
%\subsection*{Effect of modulator}
%The strength of MOFs is their versatility and their potential to be tuned for a given application. Specifically for UiO--66 different synthesis procedures were %used to alter the chemical structure of the material. In particular, during the MOF preparation, monocarboxylic acids which are in %competition with bi- or polycipital linkers to coordinate to the metal sites can %prevent the connection between two inorganic bricks, creating linker deficiencies \cite{Vermoortele2013}. Experimentally UiO--66 was synthesized using different modulation approaches and the effect of the type and amount of modulator added during synthesis was perceived to determine the degree to which structural defects are created \cite{Vermoortele2013, Shearer2016, Cliffe2014, Morris2017mod}. From a theoretical viewpoint in %\textbf{Paper I}, we investigated the role of modulating species such as  trifluoroacetic acid (TFA), hydrochloric acid (HCl) and water in the creation and stabilization of linker vacancies \cite{Vandichel2015}. By constructing free energy diagrams, we found that during the synthesis of UiO--66 with an excess concentration of TFA in the reaction mixture, the Zr atoms may adsorb two TFA groups instead of one BDC linker. This project was performed in a close collaboration with the group of Prof. Dirk De Vos in which these features were observed experimentally. We further showed that the presence of HCl facilitates the formation of defects which was also seen in the Fourier--transform infrared spectroscopy (FTIR) by DeCoste \textit{et al.}\cite{DeCoste2013}. Furthermore, at the low temperature synthesis, below 423 K, the addition of hydrochloric acid enables the replacement of hydroxyl groups in the inorganic node by incorporation of chloride anions. As can be seen in Figure \ref{fig:figure_Paper1a}, the formation of linker defects is certainly not thermodynamically driven, but the addition of modulators such as TFA or HCl lowers the free energy required for the formation of linker deficiency. The amount of defects can be concentration driven, and expands with increasing amount of a modulator concentration, which is in competition with BDC. Furthermore, postsynthetic activation procedures at elevated temperatures result in substantial removal of the capping ligands,  leaving more coordinatively unsaturated Zr Lewis acid sites. This concept can be extended to other MOFs.
%
%\begin{figure}[ht]
%	\centering
%	\includegraphics[width=1.0\textwidth]{figure_Paper1a}
%	\caption[Schematic representation of the calculated structures with different
%	capping ligands.]{Schematic representation of the calculated structures with different
%	capping ligands. Free energy differences are reported with respect to
%the reference configuration X(BDC) during synthesis conditions (T = 403 K,
%p = 1 bar). The
%results were obtained in a periodic calculation with the PBE--D3(BJ)
%functional, employing an energy cutoff of 400 eV. Adapted from Ref.
%\cite{Vandichel2015} with permission of the Royal Society of Chemistry, copyright 2015.}
%	\label{fig:figure_Paper1a}
%\end{figure}
%
%\newpage
%\subsection*{Dehydration mechanism of UiO-66}
%Dehydration of UiO--66 occurs at elevated temperature in the range between 523 - 573 K.The entropy driven activation conditions promote the dehydration process of the material, which results in the removal of water from the inorganic brick, and is schematically shown in Figure \ref{fig:figure_Paper1b}. For the first time the mechanistic pathway for the initial dehydration process of the inner \ce{Zr6O4(OH4)} brick of UiO--66 was proposed in \textbf{Paper I} and the reaction was studied on the material with one missing linker with unit cell formula \ce{[Zr6O4(OH)4(RCOO)12][Zr6O6(OH)2(RCOO)10]} \cite{Vandichel2015}.
%
%\begin{figure}[!htp]
%	\centering 
%	\includegraphics[width=1.0\textwidth]{figure_Paper1b}
%	\caption[At the top of the figure, the structure of the modeled UiO--66 crystal is displayed with indication of the unit cell. Zr--brick 1 with formula \ce{[Zr6O4(OH)4(RCOO)12]} is intact with no missing linker, while brick 2 \ce{[Zr6O6(OH)2(RCOO)10]} has one missing terephthalate linker.]{At the top of the figure, the structure of the modeled UiO--66 crystal is displayed with indication of the unit cell. Zr--brick 1 with formula \ce{[Zr6O4(OH)4(RCOO)12]} is intact with no missing linker, while brick 2 \ce{[Zr6O6(OH)2(RCOO)10]} has one missing terephthalate linker. Adapted from Ref. \cite{Vandichel2015} with permission of the Royal Society of Chemistry, copyright 2015.}
%	\label{fig:figure_Paper1b}
%\end{figure}
%\npar
%A two--step procedure was applied to gain a profound molecular insight into this process and involved intermediate structures. Initially, the energy surface was explored by performing a series of MD simulations based on NPT ensemble with variable volume at a pressure of $10^{-9}$ bar. The initial structure for the MD was chosen to have the hydroxyl groups ordered in a fully symmetrical way representing the lowest energetically configuration for UiO--66. During simulations, the temperature was gradually increased to gain structural insight into temperature--induced distortions which  lead to dehydration. The most common distortion pattern which was observed along the MD trajectory showed simultaneous decoordination of a  \ce{{\textmu}3}--OH hydroxyl group together with one BDC linker,  and was taken as an initial structure for the following static geometry optimizations. Subsequently, the electronic energy profile was obtained after performing a series of NEB simulations between initial and end structure (Chapter \ref{ch2}). The potential energy surface reveals the appearance of two transition states and some metastable configurations as displayed in Figure \ref{fig:figure_Paper1c}.
%As it was deduced form the MD simulations, the process was initiated by the \ce{{\textmu}1}--OH hydroxyl group configuration and decoordinated BDC linker. The hydroxyl group changes its coordination from 3 (\ce{{\textmu}3}--OH) to 1 (\ce{{\textmu}1}-OH) by traveling to another site of the inorganic brick and subsequently undergoes protonation with another \ce{{\textmu}3}--OH hydroxyl group to form water. The highest point on the profile which corresponds to a first transition state is encountered in structure 7. In this configuration one linker is decoordinated with the carboxylic oxygen from the Zr atom and the hydroxyl group is detached from two Zr atoms. After that, some intermediate, metastable structures are observed, where the linker is in the chelated state being coordinated with two O atoms to the same Zr atom (structure 12). When the decoordinated hydroxyl group is on the other side of the inorganic brick, the linker recoordinates to the parent position (structure 20). This configuration represents a local minimum on the potential energy surface. Subsequently, the released hydroxyl group abstracts a proton of another available \ce{{\textmu}3}--OH group (structure 23), which results in the formation of a water molecule (structure 27) and partly dehydrated inorganic brick. It should
%be noted that the studied UiO--66 structure can undergo two additional dehydration reactions (Figure  \ref{fig:figure_Paper1b}), which were further investigated in \textbf{Paper II}. Moreover, the presence of missing linkers can also affect this process substantially.
%
%\begin{figure}[!htp]
%%\vspace{-2cm}
%	\centering
%	\includegraphics[width=1.0\textwidth]{figure_Paper1c}
%	\caption[Electronic energy profile for the first dehydration reaction.
%The relevant atoms leading to the final removal of water are encircled.
%The linkers, which are not involved in the dehydration, are not
%displayed. The electronic energies were obtained from an energy
%refinement (on all the images) with the PBE--D3(BJ) functional,
%employing an energy cutoff of 400 eV.]{Electronic energy profile for the first dehydration reaction.
%The relevant atoms leading to the final removal of water are encircled.
%The linkers, which are not involved in the dehydration, are not
%displayed. The electronic energies were obtained from an energy
%refinement (on all the images) with the PBE--D3(BJ) functional,
%employing an energy cutoff of 400 eV. Adapted from Ref.\cite{Vandichel2015} with
%permission of the Royal Society of Chemistry, copyright 2015.}
%	\label{fig:figure_Paper1c}
%\end{figure}
%
%\newpage
%\subsection*{Effect of missing linkers on dehydration}
%A thorough understanding of the influence of various linker deficiency in
%UiO--66 on the dehydration mechanism at reaction conditions was reported in
%\textbf{Paper II} \cite{Vandichel2016}. A total of six defect structures
%were generated with a different number and position of missing linkers.
%The dehydration mechanism remains the same for all structures. However,
%the influence of defects on the dehydration processes is yet unknown and it is very
%challenging to study this effect experimentally. To obtain a molecular insight
%into the dehydration mechanism of the six different structures of UiO--66 the
%same methodology as in the \textbf{Paper I} was applied making use of
%NEB calculations.
%As discussed earlier, the dehydration of the material occurs at elevated
%temperature. To account for the thermal corrections, all local minima and
%transition states were further optimized employing a stronger convergence
%criterion for the electronic self--consistent field (SCF) problem. The vibrational mode analysis was then performed on the
%stationary points by computing full Hessian frequency calculations. In this
%context we described structural deformations, therefore a full Hessian
%approach, with frequency calculations that include all the atoms, was required
%to properly account for the entropic contribution. Only positive eigenvalues were seen for minima, while for transition states the presence of one negative eigenvalue corresponding to the right normal mode verified that the system was in a first order saddle point.
%For all 12--fold coordinated bricks the removal of the second water molecule is
%higher activated than the first one. The endothermic nature of the dehydration
%process was also confirmed experimentally by Shearer \textit{et
%al.}\cite{Shearer2013}. Herein, it was found that water removal becomes easier
%and the process is lower activated when defects are located nearby pointing out that they might be correlated.
%\npar
%In \textbf{Paper I}, we postulated the
%dehydration reaction as a two--step process which proceeds through the
%intermediate dangling OH--groups (configuration 20, Figure
%\ref{fig:figure_Paper1c}). Shearer \textit{et
%al.}\cite{Shearer2013} performed
%\textit{in situ} IR experiment in which the temperature was gradually increased
%and observed the presence of extra OH bands appearing at 3600--3700
%cm\ce{^-^1}. This clearly indicates structural rearrangements during thermal
%activation. Theoretically obtained \textit{ab initio} IR spectrum on the cluster
%model of UiO-66 with a decoordinated, dangling OH group demonstrates that the stretching
%mode frequency of the hydroxyl group shifts to lower wavenumbers,
%entirely confirming experimental observations (Figure \ref{fig:figure_Paper2a}).
%
%\begin{figure}[!htp]
%	\centering
%	\includegraphics[width=1.0\textwidth]{figure_Paper2a}
%	\caption[\textit{Ab initio} IR--spectra of a Zr--brick with a dangling
%	OH--bond. Derived from a cluster model with level of theory B3LYP/(6--31g(d)+
%	LanL2DZ).
%	The \textit{ab initio} spectrum was scaled with a factor 0.96.]{\textit{Ab initio} IR--spectra of a Zr--brick with a
%	dangling OH--bond.
%	Derived from a cluster model with level of theory B3LYP/(6--31g(d)+LanL2DZ).
%	The \textit{ab initio} spectrum was scaled with a factor 0.96. Reprinted from
%	Ref.\cite{Vandichel2016} with permission of the Royal Society of Chemistry,
%	copyright 2016.}
%	\label{fig:figure_Paper2a}
%\end{figure}
%\npar
%\newpage
%It may be anticipated that introducing a large
%concentration of defects has also a large impact on the mechanical properties of the
%material. Recently Yot \textit{et al.}\cite{Yot2016} investigated the
%mechanical behavior of UiO--66 by means of high-pressure powder
%X-Ray diffraction studies up to 3.5 GPa. It was observed that the
%material showed a gradual pressure--induced reversible decrease of
%crystallinity. The bulk modulus of the material was determined to be 17 GPa and
%by introducing amino functionalization this value even 
%increased to 25 GPa. This observation points
%towards a high mechanical robustness of the UiO-66 type of materials on top of
%their already well known thermal and chemical stability. However, the detailed
%effect of the amount and correlation of missing linkers in the material was not
%explored. From theoretical point of view Wu \textit{et al.}\cite{Wu2013} reported, using
%periodic DFT calculations the bulk modulus of 41 GPa for the defect free UiO--66
%structure. Hereby, for each of the proposed defect structures we investigated
%the bulk modulus, to elucidate the influence of a varying number of linker
%deficiencies on the mechanical behavior. Our results showed that bulk modulus is
%strongly affected by the number of missing linkers but also the precise position
%of the defects, which was also observed in a later, more extensive study by
%Rogge \textit{et al.}\cite{Rogge2016}.
%In \textbf{Paper II} we also investigated the evolution of the bulk modulus on
%the dehydrated structures of UiO--66 proving that this process barely affects
%the robustness of UiO--66.
%
%\newpage
%\subsection*{Inherent flexibility of UiO--66}
%In \textbf{Paper III}, we made a major step forward in the structural understanding of the
%inert Zr--based MOFs, by using enhanced sampling MD simulations
%which accelerate sampling of higher activated regions on the free energy 
%surface. Application of theses methods allow following \textit{in situ} the
%fast dynamic changes in the coordination number of the framework metals during
%different activation processes which are realized experimentally. The UiO--66
%material is a showcase example of an extremely stable MOF, 
%which maintains its integrity while allowing structural deformations during
%activation processes such as linker exchange, dehydration and defects
%formation \cite{Kim2012a, Kim2012, Horike2009, Leus2016}. Recently, by using
%ultrafast 2--D IR spectroscopy, structural fluctuations were shown in
%functionalized Zr--based MOFs \cite{Nishida2014}, yet the understanding at a
%molecular level of the intrinsic dynamics that drive these processes represents a puzzle which is almost impossible to fully solve on a purely experimental basis.
%Hereby, we used the US methodology to enhance the sampling of low probability
%regions along certain coordinates of the system. We showed that by examining the UiO-66 system at 573 K, thus at the
%dehydration temperature of this material,
%the coordination number of the hydroxyl group bridging the Zr atoms of the
%inorganic brick changes from 3 to 1 (Figure \ref{fig:figure_Paper3a}). During
%this process internal deformation modes were detected pointing to reversible linker decoordination.
%We defined the collective variable by means of the coordination number based on
%the distance between the hydroxyl group and the three surrounding Zr atoms, as
%schematically shown in Figure \ref{fig:figure_Paper3a}. With this choice, during
%the simulations we discriminated five classes of configurations which vary in
%the degree of coordination of the hydroxyl group with Zr
%atoms going from a 3--fold to a 1--fold coordination, as displayed in Figure
%\ref{fig:figure_Paper3a} \cite{Hajek2018}. A first exploration of the reaction
%path characterized by this CV has been
%performed by means of a constrained MD simulation. This constrained MD method
%takes the form of a moving umbrella which during the whole duration of the MD
%crosses the entire range of CV with a constant velocity. In this simulation, a bias potential is moved smoothly along the reaction
%coordinate to drive the system from the reactant to the product state at the same
%conditions. This is a crucial step in the US methodology which allows selecting
%of the various windows on the reaction path along the CV and determining the
%initial structures in each of these windows.
%
%\begin{figure}[!htp]
%	\centering
%	\includegraphics[width=1.0\textwidth]{figure_Paper3a}
%	\caption[Schematic representation of the applied collective variable. Three
%	critical \ce{Zr-O} distances, represented with solid purple, yellow and red
%	lines are used as input in the formal definition of the collective variable.
%	Free energy profile at 573 K along the reaction path to dehydration. Some
%	essential turning points on the profile are indicated with a schematic display
%	of their respective configurations. Probability distributions regarding the
%	type of coordination of the hydroxyl group with respect to the Zr--centers
%	measured during the US simulation are reported in the bottom.]{Schematic representation of the applied collective variable. Three
%	critical \ce{Zr-O} distances, represented with solid purple, yellow and red
%	lines are used as input in the formal definition of the collective variable.
%	Free energy profile at 573 K along the reaction path to dehydration. Some
%	essential turning points on the profile are indicated with a schematic display
%	of their respective configurations. Probability distributions regarding the
%	type of coordination of the hydroxyl group with respect to the Zr--centers
%	measured during the US simulation are reported in the bottom. Reprinted from
%	Ref.\cite{Hajek2018} with permission of the Royal Society of
%Chemistry, copyright 2018.}
%	\label{fig:figure_Paper3a}
%\end{figure}
%\npar
%Configuration 1 at 573 K corresponds to the equilibrium structure.
%In this case, each of the six Zr atoms in the brick is 8--fold
%coordinated an the four \ce{{\textmu}3}--OH groups are ordered
%symmetrically in a tetrahedral fashion. Activation of the system towards configuration 2 requires
%20 kJ/mol of free energy. From the analysis of the US simulations we can obtain more
%information on the type of connection between the oxygen and the three Zr atoms
%by inspecting the probability distributions for each of
%the configurations. In the pristine UiO--66, the relatively mobile nature of the
%\ce{{\textmu}3}--OH group is displayed by a loosely connected
%OH--bond with one of the Zr atoms, which is equally distributed between the three Zr atoms. Furthermore,
%the dehydration process is initiated by configuration 2, as a subsequent
%evolution towards lower coordination numbers does not reveal any
%\ce{{\textmu}3}--OH configurations. In configuration 3 the
%hydroxyl group is covalently bonded to only one of the Zr atoms, but
%additionally it maintains the coordination bond with one of the two remaining Zr
%atoms.
%This is a crucial point along the reaction path where we see for the first
%time the elongation in the M--L bond and out--of--plane deformation of the
%phenyl ring. These explain the increased free energy and give a clear
%indication that the system enters the transition state region. The broad set of
%structures encountered after configuration 3 possesses substantial linker
%mobility. The internal deformation modes in the configuration of type 4 include
%linker decoordination, translation, rotation and recoordination and are observed
%for the four linkers bonded to the same Zr atom. In Figure \ref{fig:figure_Paper3b} two possible
%motions of the BDC linker are displayed.
%
%\begin{figure}[!htp]
%	\centering
%	\includegraphics[width=0.7\textwidth]{figure_Paper3b}
%	\caption[Umbrella sampling in two windows of CV = 1.45 and 1.51, showing two
%	distinct motions of the linkers. On the left column, a translation of the
%	linker L1 generates a chelated structure and a subsequent shift in the
%	carboxylic oxygen connected to Zr2 (configuration 4t). On the right column, a
%	rotation of the linker L1 and a partial decoordination of linker L2 forming a
%	hydrogen bond with an \ce{{\textmu}3}--OH hydroxyl group is shown
%	(configuration 4r). A proton transfer between the carboxylic oxygen O3 and the
%	bridging \ce{{\textmu}3}--O is also observed, indicating the occurrence of an
%	intrinsic dynamic acidity. Colors indicate the coordination number of Zr atoms.]{Umbrella sampling in two windows of CV = 1.45 and 1.51, showing two
%	distinct motions of the linkers. On the left column, a translation of the
%	linker L1 generates a chelated structure and a subsequent shift in the
%	carboxylic oxygen connected to Zr2 (configuration 4t). On the right column, a
%	rotation of the linker L1 and a partial decoordination of linker L2 forming a
%	hydrogen bond with an \ce{{\textmu}3}--OH hydroxyl group is shown
%	(configuration 4r). A proton transfer between the carboxylic oxygen O3 and the
%	bridging \ce{{\textmu}3}--O is also observed, indicating the occurrence of an
%	intrinsic dynamic acidity. Colors indicate the coordination number of Zr atoms.
%	Reprinted from Ref.\cite{Hajek2018} with permission of the Royal Society of
%	Chemistry, copyright 2018.}
%	\label{fig:figure_Paper3b}
%\end{figure}
%\npar
%The first motion illustrates the
%translation of the linker along the axis of the two Zr atoms to which it is
%bonded (Figure \ref{fig:figure_Paper3b}, left column). In this case the intermediary chelated structure
%was perceived. The second motion which is schematically shown in the right
%column of Figure \ref{fig:figure_Paper3b} 
%corresponds to a rotation mode along the \ce{Zr-O} bond. The direct consequence
%of rotation of this linker is the decoordination of another linker which is
%bonded to the same Zr atom. This second linker is immediately coordinated to the
%nearest \ce{{\textmu}3}--OH gaining the stabilization due to a proton shuttle between the hydroxyl group and the carboxylic oxygen of
%the linker. Such proton shuttles may also be
%facilitated by the presence of other guest species in the pores, such as
%methanol (\textbf{Paper VIII}). These results expand the earlier concept of
%dynamic acidity introduced by Ling and Slater \cite{Ling2016} and in
%\textbf{Paper II} in the defective UiO--66 to defect--free material where guest
%protic species are not present. Configurations of type 5 are characterized by the hydroxyl group being in a \ce{{\textmu}1}--OH
%configuration. In this part of the simulations the system undergoes fast
%structural fluctuations which are rationalized by a combination of the two
%earlier described motions of translation and rotation on a multidimensional free
%energy plateau at about 90 kJ/mol. The complex intrinsic dynamic flexibility of the framework is shown in Figure \ref{fig:figure_Paper3c}.
%
%\begin{figure}[!htp]
%	\centering
%	\includegraphics[width=1.0\textwidth]{figure_Paper3c}
%	\caption[Different structures observed in the simulation window of
%	configuration 5, which are rationalized by the two motions presented in Figure
%	\ref{fig:figure_Paper3b}. The colors of the Zr atoms indicate their
%	coordination number.]{Different structures observed in the simulation window of
%	configuration 5, which are rationalized by the two motions presented in Figure
%	\ref{fig:figure_Paper3b}. The colors of the Zr atoms indicate their
%	coordination number. Reprinted from Ref.\cite{Hajek2018} with permission of the
%	Royal Society of Chemistry, copyright 2018.}
%	\label{fig:figure_Paper3c}
%\end{figure}
%\npar
%During the translation, rotation and combination of these two
%deformation modes, which are observed for configurations of type 4 and 5, we
%also saw substantial rearrangements of the coordination numbers of the various Zr
%atoms. The change in the coordination numbers of Zr can easily vary between 8
%and 6. This is directly connected to a remarkable intrinsic dynamic behavior of
%the system, where neighboring linkers decoordinate and are 
%stabilized by hydrogen bonding interactions with adjacent hydroxyl groups.
%Another important aspect is that linkers have the ability to reversibly
%decoordinate and coordinate to the Zr atoms being responsible for the
%extraordinary stability of the UiO--66 material, which is undoubtedly related to the oxophilic character of the metal ion. This study in which the activation
%process is entropically driven might eventually also lead to the
%formation of water, thus dehydration of UiO--66. All the activated processes in
%which defects on UiO--66 are formed may have a decisive effect on catalytic
%reactions. These defects which create active sites in the material may act as
%Lewis acid but also Br\o{}nsted acid or base sites as it has already been shown in literature.
%The performance of UiO--66 as a multifunctional catalyst was intensively
%investigated in \textbf{Paper IV} and \textbf{Paper V}.
%
%
%\subsection*{Effect of dehydration on catalysis}
%In the dehydration and other processes creating defects, the coordination number
%of Zr changes which has a decisive effect on the catalytic properties of the
%material. A profound molecular level understanding of these modifications on the
%strength of Br\o{}nsted and Lewis acid sites within MOFs is crucial to tune the catalytic
%activity for many acid--base catalyzed reactions. In \textbf{Paper IV} detailed
%molecular and experimental insights were gained on the influence of the degree
%of dehydration on the nature of the active sites in the catalysis of the
%Oppenauer oxidation \cite{Hajek2017}.
%The hydrated and dehydrated UiO--66 material with one linker defect was
%considered as a catalyst involving three different adjacent Zr active sites, as displayed in Figure
%\ref{fig:figure_Paper4a}.
%
%\begin{figure}[!htp]
%	\centering
%	\includegraphics[width=1.0\textwidth]{figure_Paper4a}
%	\caption{Representation of active sites on hydrated and dehydrated UiO--66 and
%	Zr--beta zeolite. In each configuration Zr Lewis acid sites
%	and Br\o{}nsted base sites are
%	encircled.}
%	\label{fig:figure_Paper4a}
%\end{figure}
%\npar
%On the Zr--beta zeolite the reaction mechanism of the
%reverse MPV reduction of aldehydes to ketones was
%unraveled by Boronat \textit{et al.}\cite{Boronat2006}. In the proposed
%mechanism well--defined and isolated Lewis acid and Br\o{}nsted base sites were shown to
%be involved in the lowest activated reaction path. A schematic representation of the
%resemblance between the active sites in UiO--66 and the Zr--beta zeolite is
%shown in Figure \ref{fig:figure_Paper4a}.
%The reactions where dehydrogenation of alcohols is involved represent an example
%in which acid-base centers of intermediate strength are required. Although the
%active sites on the Zr--beta zeolite look similar to the one on UiO--66 and the
%reaction mechanism of the MPV reduction seems to be rather understandable, 
%the role of the acid and base sites and structural modifications in the UiO--66
%remain to be investigated for fine--tuning catalytic properties. To
%study the influence of both missing linkers and dehydration of the structure on
%the catalytic activity and to elucidate the bifunctional, amphoteric nature of
%UiO--66, the Oppenauer oxidation of alcohols to carbonyl compounds was chosen as
%a model reaction.
%The Lewis acid character of differently coordinated Zr atoms was investigated by
%means of DFT calculations using furfural, a strong oxidant as a probe molecule.
%Surprisingly, the adsorption free energy at 393 K of this molecule on the three
%considered active sites varies of about 5 kJ/mol at PBE--D3 level of theory
%(Figure \ref{fig:free_energy_oxidation}, configuration 1).
%This clearly indicates that the Lewis acid strength of Zr atoms on hydrated and
%dehydrated material is preserved. Prenol was chosen as a primary alcohol to be
%oxidized to prenal. The co--adsorption of this molecule on adjacent Lewis acid site has almost equal strength on the three considered active sites indicating that the Lewis acidity is not significantly altered between hydrated and dehydrated material. The specific coordination number of Zr has a minor effect on the Lewis acid strength which is distributed over the Zr metal and the bond with the adjacent \ce{{\textmu}3}--O (configuration 2). The next step in the reaction mechanism
%represents the transition state of the deprotonation of the alcohol to the
%\ce{{\textmu}3}--O atom (Ts1). This step proceeds the fastest on the hydrated
%material indicating that it possesses stronger base sites than the dehydrated
%UiO-66. The free energy barrier amounts to 7.4 kJ/mol on the active site of the
%hydrated material and it increases to 36.7 kJ/mol and 50.9 kJ/mol, on the active
%site on the dehydrated UiO-66 7-- and 6--fold coordinated, respectively.
%Furthermore, the formed alkoxide is the most strongly stabilized on the hydrated material. From the geometrical analysis of Zr--\ce{{\textmu}3}--O distances we observed that the higher the coordination numbers of the involved
%zirconium atoms the more stable the alkoxide becomes. The Lewis acid and
%Br\o{}nsted base sites can attractively interact as electrons are less
%concentrated between the Zr and \ce{{\textmu}3}--O atoms. In the second
%transition state the atoms from the inorganic framework are not directly
%involved into the carbon-to-carbon hydride shift between two reacting molecules.
%The effects is immediately observed by similar energy barriers for this
%transformation on the hydrated and dehydrated UiO--66. The
%hydride shift gives rise to prenal, the desired product of the
%Oppenauer oxidation (configuration 4). Hereby, the hydrated UiO--66 was found as
%a better catalyst than the dehydrated material to facilitate the Oppenauer
%oxidation of prenol to prenal with furfural as an oxidative agent. The experimental results
%on the influence of both linker vacancies and hydration state of the UiO--66 framework on its catalytic performance confirmed theoretical predictions. A much higher activity was systematically found on the hydrated material. Within this work we showed that the properties of the material can be tuned by the pretreatment procedures and there is a fine cooperation between the Lewis acid and Br\o{}nsted base sites in
%the inorganic brick. This study also demonstrates that chemical properties of
%UiO--66 are similar to its \ce{ZrO2} precursor.
%
%\begin{figure}[ht]
%	\includegraphics[width=1.0\textwidth]{free_energy_oxidation}
%	\centering
%	\caption[Free energy profile of the Oppenauer oxidation of prenol with furfural
%	(periodic with PBE--D3) given at 393.15 K. Hydrated brick is indicated by a
%	blue line, while dehydrated 7-- and 6--fold coordinated Zr by a purple and
%	green line, respectively. X corresponds to: X = O in the hydrated and
%	dehydrated 7--fold material, and X=vacancy for the cluster with 6--fold Zr coordination. R
%	corresponds to the reactants in gas phase and the catalyst, P corresponds to
%	the final products in gas phase and the catalyst.]{Free energy profile of the Oppenauer oxidation of prenol with
%	furfural (periodic with PBE--D3) given at 393.15 K. Hydrated brick is indicated by a
%	blue line, while dehydrated 7-- and 6--fold coordinated Zr by a purple and
%	green line, respectively. X corresponds to: X = O in the hydrated and
%	dehydrated 7--fold material, and X=vacancy for the cluster with 6--fold Zr coordination. R
%	corresponds to the reactants in gas phase and the catalyst, P corresponds to
%	the final products in gas phase and the catalyst. Reprinted from Ref.
%	\cite{Hajek2017} with permission from Wiley--VCH Verlag, copyright 2017.}
%	\label{fig:free_energy_oxidation}
%\end{figure}
%
%
%\subsection*{Influence of water on the active catalytic sites}
%The adsorption of water by porous solids and their stability in water
%environment is essential for their application in catalysis. In this
%respect, chemical and hydrothermal stability of UiO--66 was proved by Leus
%\textit{et al.}\cite{Leus2016}. Creation of defects and disorders in the
%material occurs during the synthesis, and in this respect, a missing
%linker defect in the material can be seen as a removal of one negatively charged
%BDC linker. In such a way, a positive charge is created on the brick which has
%to be compensated. This can happen either by removing a positively charged
%proton, or by adsorbing a negative species such as a hydroxyl group to one of the
%coordinatively unsaturated Zr atoms. Experimentally the identification and
%characterization of the nature of defects in MOFs on the molecular level remains a very challenging task to
%achieve. Within the use of single--crystal X--ray diffraction (SXRD) the
%presence of terminating hydroxyl groups and/or coordinating water molecules on the
%defect sites of UiO--66 was observed by the group of Lillerud \cite{Oien2014}
%and the group of Yaghi \cite{Trickett2015}. Furthermore in the work of Yaghi \textit{et
%al.}\cite{Trickett2015}, an additional hydroxide anion per defect site was
%perceived to coordinate with a defective \ce{Zr-O-Zr} site. It was postulated that upon
%removal of one linker, two adjacent Zr atoms are capped with physisorbed water
%molecules and the charge compensating hydroxyl counterion is stabilized by a
%hydrogen bond with a neighboring \ce{{\textmu}3}--OH group. In \textbf{Paper II}
%we considered different configurations of coordinated water molecules on the
%open metal sites to determine the chemical nature of active sites. 
%In the simplest case, upon removal of a proton, also two Br\o{}nsted
%basic sites are created and these are the two oxo atoms from the framework that
%bridge the adjacent Zr atoms. The configuration with coordinatively unsaturated
%adjacent Lewis acid sites was taken as the reference structure R (Figure
%\ref{fig:figure_Paper2b}). In the latter case, the adsorption of one water
%molecule gives rise to an additional Br\o{}nsted acid and base sites, but the
%Lewis acid property of Zr is lost. Subsequently, more water molecules can be
%coordinated to the open metal sites, and several possibilities have been
%screened as reported in Figure \ref{fig:figure_Paper2b}. In all structures
%various possible Lewis and Br\o{}nsted sites are indicated which may have a
%decisive role in a catalytic reaction. 
%
%\begin{figure}[!htp]
%	\centering
%	\includegraphics[width=1.0\textwidth]{figure_Paper2b}
%	\caption[Coordination free energies at reaction temperature of 351 K of one,
%	two and three water molecules at coordinatively unsaturated Zr--bricks in
%	defective UiO--66 with respect to a water coordination free site (site R). Free
%	energies (in black) are given in kJ/mol, and their decomposition into enthalpic
%	$\Delta H$ (blue) and entropic $-T \Delta S$ (gray) contributions. Energies
%	are resulting from periodic calculations with PBE--D3 level of theory. In each
%	configuration Lewis acid and Br\o{}nsted sites are indicated.]{Coordination free energies at reaction temperature of 351 K of one,
%	two and three water molecules at coordinatively unsaturated Zr--bricks in
%	defective UiO--66 with respect to a water coordination free site (site R). Free
%	energies (in black) are given in kJ/mol, and their decomposition into enthalpic
%	$\Delta H$ (blue) and entropic $-T \Delta S$ (gray) contributions. Energies
%	are resulting from periodic calculations with PBE--D3 level of theory. In each
%	configuration Lewis acid and Br\o{}nsted sites are indicated. Reprinted from
%	Ref. \cite{Caratelli2017} with permission from Elsevier, copyright 2017.}
%	\label{fig:figure_Paper2b}
%\end{figure}
%\npar
%\newpage
%The most stable configurations represent the structures of UiO--66 which defect
%site is capped by two or three water molecules. The physisorbed water molecule
%on the Lewis acid site immediately deprotonates to the \ce{{\textmu}3}--O atom.  
%The structure with the chemisorbed hydroxyl group and physisorbed water forms a
%strongly stable complex with a free energy difference of -94.4 kJ/mol compared
%to reference configuration R. The presence of a third water molecule stabilizes 
%this complex slightly more. The last configuration with three water molecules is
%indeed some kJ/mol more bonded on the free energy surface at 351 K. This study
%is complementary to the work of Ling and Slater \cite{Ling2016}, in which a
%detailed understanding into the coordination of water and charge balancing hydroxide ions was obtained by applying an MD approach.
%Ling and Slater \cite{Ling2016} showed that the structure proposed by Yaghi and
%co--workers \cite{Trickett2015} does not represent energetically the most stable
%configuration.
%Instead, the hydroxide ion coordinates to the Zr metal while two water molecules
%are physisorbed to the oxo and Zr atoms as shown in Figure \ref {fig:figure_Paper2b}. The water stabilization pattern is in agreement with what was observed in this work in
%\textbf{Paper II} and \textbf{Paper VII}. However, depending on the
%temperature and concentration of water molecules various proton shuttles may
%occur proving the dynamic Br\o{}nsted acidity of active sites in UiO--66. The
%beneficial influence of water on the reaction cycle of Fischer esterification of
%an organic carboxylic acid with methanol was shown in \textbf{Paper VII} in the combined theoretical and experimental work. The mechanism of this reaction was studied on the hydrated
%and dehydrated Zr Lewis acid sites of UiO--66. The complementary performance of
%Lewis and Br\o{}nsted sites was required to substantially increase the catalytic activity of the material. Moreover, the
%experimental data clearly indicated that the catalytic activity dropped upon
%removal of physisorbed water from UiO--66, which is in complete agreement with
%theoretical model \cite{Caratelli2017}.
%
%
%\subsection*{Effect of linker functionalization}
%In many MOFs catalyzed reactions the observed increased catalytic activity
%results from the formation of a large number of accessible active sites or
%linker functionalization. Vermoortele \textit{et al.}\cite{Vermoortele2012}
%demonstrated that the UiO--66 materials with electron--withdrawing groups
%substitution are notably more active for citronellal cyclization. Furthermore,
%in another study by Vermoortele \textit{et al.}\cite{Vermoortele2013} it was observed that more accessible pores and large amount of available open metal sites enhance the electronic
%effect of the nitro group. In this respect the synthesis
%procedures by combine use of TFA and HCl were crucial to modulate the material for desired
%applications and strengthen the Lewis acid character of the active sites. To
%obtain a profound molecular understanding on the modulation effect of TFA on the
%material reactivity we used the citronellal cyclization as a probe reaction for
%Lewis acid strength. In this simulation, the extended cluster model with
%coordinatively unsaturated adjacent Zr sites surrounded by two terephthalate
%linkers and one TFA was used. The remaining four linkers were replaced by
%formate groups. In Figure \ref{fig:citronellal} the free energy profile for
%conversion of citronellal to isopulegol is shown at 373 K. The
%electronic modulation effect of a TFA group on the nitro functionalized
%UiO--66(\ce{NO2}) results in stronger adsorption and leads to lower apparent
%barriers for the transition and product states when compared to the material
%without incorporated modulator and functionalization.
%
%\begin{figure}[!htp]
%	\centering
%	\includegraphics[width=0.95\textwidth]{citronellal}
%	\caption[Free energy profile of
%	conversion of citronellal towards isopulegol. Level of theory for
%	H, C, N, O, F: B3LYP/6--31+G(d)--D3//B3LYP/  4--31G*; Zr:
%	LANL2DZ, calculated at 373 K.]{Free
%	energy profile of conversion of citronellal towards isopulegol. Level of theory
%	for H, C, N, O, F: B3LYP/6--31+G(d)--D3//B3LYP/4--31G*; Zr:
%	LANL2DZ//LANL2DZ, calculated at 373 K. Adapted
%	form Ref. \cite{Vandichel2015} and Ref.
%	\cite{Vermoortele2012} with permission of the Royal Society of Chemistry,
%	copyright 2015 and Wiley--VCH Verlag, copyright 2012.}
%	\label{fig:citronellal}
%\end{figure}
%\npar
%\newpage
%Vermoortele \textit{et al.}\cite{Vermoortele2011} reported the
%amino--functionalized UiO--66(\ce{NH2}) as a highly selective catalyst for the cross--aldol
%condensation between benzaldehyde and heptanal (Figure \ref{fig:figure_Paper5a}). A similar conclusion 
%for the acetylisation of benzaldehyde was published by Timofeeva \textit{et
%al.}\cite{Timofeeva2014} and for the esterification reactions by Cirujano \textit{et
%al.}\cite{Cirujano2015}.
%
%\begin{figure}[!htp]
%	\centering
%	\includegraphics[width=1.0\textwidth]{figure_Paper5a}
%	\caption{Aldol condensation between benzaldehyde and heptanal (cross-- and
%	self--aldol condensations, respectively) and representation of two
%	catalysts with indicated possible active sites.}
%	\label{fig:figure_Paper5a}
%\end{figure}
%\npar
%The observed increased activity with comparison to the
%pristine UiO--66 was explained by the co--catalytic role of amino groups. The
%proposed experimental reaction mechanism was not verified on the molecular level,
%therefore in \textbf{Paper V} we aimed to obtain in--depth insight into the
%active sites for the aldol condensation reaction on the two catalysts,
%unraveling the true role of the amino group. The catalytic performance of
%UiO--66 and amino--functionalized UiO--66(\ce{NH2}) was evaluated for the
%cross-- and self--aldol condensation of propanal and benzaldehyde. To rationalize the
%reaction mechanism it was recommended to first use the extended cluster model followed
%by periodic calculations for the most probable mechanism. In line with what
%was earlier observed for UiO--66, the catalytic activity of this material has
%its origin in the presence of defects which are coordinatively unsaturated Lewis
%acid sites. For the aldol condensation on UiO--66 we proposed a six steps
%mechanism, shown in Figure \ref{fig:figure_Paper5b} which is initiated by the
%adsorption of two reactants on the adjacent Zr atoms. Furthermore, the
%bi--functional nature of UiO--66 was unraveled. To initiate the reaction, propanal was
%solely activated by deprotonation of the $\alpha$--carbon atom to the
%\ce{{\textmu}3}--O atom, which acts as a Br\o{}nsted
%base. We also investigated a pathway in which aldol condensation is initially
%Br\o{}nsted acid catalyzed by means of hydroxyl group from the inorganic
%framework but found higher free energy barriers \cite{Hajek2015}.
%
%\begin{figure}[!htp]
%	\centering
%	\includegraphics[width=1.0\textwidth]{figure_Paper5b}
%	\caption[Comparison of the free energy profiles for the cross--aldol
%	condensation on UiO--66(\ce{NH2}) and UiO--66. Level of theory for H, C, N, O: B3LYP/6--311++G(d,p)--D3//B3LYP/6--31G(d); Zr:
%	LANL2TZ// LANL2DZ, calculated at 393 K. R
%	corresponds to the reactants in gas phase and the cluster, P corresponds to the final
%	product and water in gas phase and the cluster.] {Comparison of
%	the free energy profiles for the cross--aldol condensation on UiO--66(\ce{NH2}) and UiO--66.
%	Level of theory for H, C, N, O: B3LYP/6--311++G(d,p)--D3//B3LYP/6--31G(d); Zr:
%	LANL2TZ//LANL2DZ, calculated at 393 K. R
%	corresponds to the reactants in gas phase and the cluster, P corresponds to the final
%	product and water in gas phase and the cluster. Reproduced from Ref.
%	\cite{Hajek2015} with permission of Elsevier, copyright 2015.}
%	\label{fig:figure_Paper5b}
%\end{figure}
%\npar
%To examine the catalytic performance of the amino--functionalized material, we
%studied the cross--aldol condensation which is initiated by the
%adsorption on the adjacent Zr Lewis acid.
%Nevertheless, the effects of the amino substitution during different steps in the reaction profile appears to be
%negligible, as it only lowers the adsorption of propanal and benzaldehyde. The
%strong adsorption of the reactants and the slightly lower activation free energy
%barrier for the UiO--66(\ce{NH2}) could point towards a higher activity of this
%material compared to the pristine UiO--66 (Figure \ref{fig:figure_Paper5b}).
%This observation was in agreement with many experimental works dedicated to
%condensation reactions in which the systematic increase in activity was obtained
%when the material was functionalized by electron--donating amino groups. It was commonly accepted that the cooperative work between the Zr Lewis sites and the amino Br\o{}nsted base sites resulted in high yields for
%this type of reactions. In some experimental works the direct role of the amino
%group as a base site was proposed \cite{Timofeeva2014, Yang2014, Gascon2009,
%Panchenko2014}. In this line, a further detailed theoretical mechanistic
%investigation on UiO--66(\ce{NH2}) was performed. The resulting free energy barriers at 393 K are highlighted in Figure \ref{fig:figure_Paper5c}. The
%deprotonation of propanol to the corresponding carbocation is about 80 kJ/mol higher activated on the amino group as a base site. Moreover the
%formation of hemiaminal was also investigated as this is an intermediate
%structure in the Knoevenagel condensation \cite{Cortese2011,
%Panchenko2014, Yang2014}. For this pathway, the free energy barrier between the
%pre--reactive complex and transition state goes up to 205.4 kJ/mol which
%indicates that the hydrogen bonding interaction between the amino group and the
%carboxyl oxygen of propanal does not reveal catalytic properties (Figure
%\ref{fig:figure_Paper5d}). Even though the imine formation is very highly
%activated it could happen with assistance of guest species which facilitate the proton shuttle, such as water. Hereby, it was demonstrated that on the
%amino--functionalized UiO--66(\ce{NH2}) the imine formation can happen as soon
%as water is present on the catalyst (Figure \ref{fig:figure_Paper5d}). 
%
%\begin{figure}[!htp]
%	\centering
%	\includegraphics[width=1.0\textwidth]{figure_Paper5c}
%	\caption[The free energy barrier of the deprotonation on the oxo--atom (TSa:
%	purple), on the amino group (TSb: orange) and imine formation (TSc:
%	blue) on the cluster model. Level of theory for H, C, N, O: B3LYP/6--311++G(d,p)--D3//B3LYP/6--31G(d); Zr:
%	LANL2TZ// LANL2DZ, calculated at 393 K. Some critical distances are given in
%	\AA. The 'zig--zag' line in the schematic representation of the TS states corresponds to the terephthalate linker.] {The free energy barrier of the deprotonation on
%	the oxo--atom (TSa: 	purple), on the amino group (TSb: orange) and imine formation (TSc:
%	blue) on the cluster model. Level of theory for H, C, N, O: B3LYP/6--311++G(d,p)--D3//B3LYP/6--31G(d); Zr:
%	LANL2TZ//LANL2DZ, calculated at 393 K. Some critical distances are given in \AA. The 'zig--zag' line in the schematic representation of the TS states corresponds to the terephthalate linker. Figure reproduced from Ref. \cite{Hajek2015} with permission of Elsevier, copyright 2015.}
%	\label{fig:figure_Paper5c}
%\end{figure}
%\npar
%
%\begin{figure}[!htp]
%	\centering
%	\includegraphics[width=1.0\textwidth]{figure_Paper5d}
%	\caption[The free energy barrier of the hemiaminal formation with water as a
%	co--catalyst (TSc--w1 and TSc--w2: pink) and without assisting water molecule
%	(TSc: blue) on the cluster model. Level of theory for H, C, N, O: B3LYP/6--311++G(d,p)--D3//B3LYP/6--31G(d); Zr:
%	LANL2TZ//LANL2DZ, calculated at 393 K. Some critical distances are given in \AA. The 'zig--zag'
%	line in the schematic representation of the TS states corresponds to the
%	terephthalate linker.]{The free energy barrier of the hemiaminal
%	formation with water as a co--catalyst (TSc--w1 and TSc--w2: pink) and without assisting water molecule
%	(TSc: blue) on the cluster model. Level of theory for H, C, N, O: B3LYP/6--311++G(d,p)--D3//B3LYP/6--31G(d); Zr:
%	LANL2TZ//LANL2DZ, calculated at 393 K. Some critical distances are given in \AA. The 'zig--zag'
%	line in the schematic representation of the TS states corresponds to the
%	terephthalate linker. Figure reproduced from Ref. \cite{Hajek2015} with
%	permission of Elsevier, copyright 2015.}
%		\label{fig:figure_Paper5d}
%\end{figure}
%\npar
%Furthermore, in the jasminaldehyde condensation water is produced as a side product and could assist in the reaction mechanism in the steps where proton transfers are involved. In \textbf{Paper VIII} it was indeed shown that polar protic solvents like methanol form a hydrogen--bonded network allowing protons to shuttle between the solvent, adsorbed water and the inorganic brick. As protons can easily shuttle among various positions on the defective sites, the active sites show a remarkable dynamic behavior \cite{Caratelli2017a}.
%\npar
%In the complementary experimental study performed in the group of Prof. Dirk De Vos the superior initial conversion of propanol which results in the higher initial selectivity for the cross aldol product on UiO--66(\ce{NH2}) was also confirmed. However, after prolonged reaction time the experimental data predicts a comparable performance of both catalysts.
%\npar
%Summarizing, in \textbf{Paper V} we unraveled that amino functionalized BDC linkers possess an electron--donating character but this group as a base is not actively involved in the reaction mechanism, indicating that the proposed mechanism of aldol condensation is similar on the pristine and amino substituted UiO--66.
%\npar 
%These findings are confirmed in \textbf{Paper VII} where it was observed that amino groups yield a decrease in the reaction barriers which is in alignment with experimental observations performed in  the group of Dr. Francesc X. Llabr\'es i Xamena. They have an active and substantial role in modulating the electronic structure of the material but a passive role in the reaction mechanism \cite{Caratelli2017}.
%
%
%\section{Contributions to other articles}
%In addition to the main research projects presented in the previous sections, a small contribution was made to study the nature of active sites in Ca--modified ZSM--5 for MTO conversion. It concerned a combined experimental--theoretical \textbf{Paper XI} and is briefly summarized below \cite{Yarulina2016}.
%
%
%\subsection*{Acid site modification with Ca}
%Within the field of catalysis, many researchers are continuously searching for suitable modifications of the catalyst to improve its performance, to direct the product selectivity and to prolong the lifetime of the catalyst. Within the MTO process it is commonly accepted that there exists a dual--cycle mechanism, involving the presence of two cycles, a methylation/cracking cycle leading especially to propylene production and an aromatic cycle, responsible for ethylene and aromatics formation. The latter feature is responsible for catalyst deactivation. The aromatics formation can be suppressed by a combination of a decreased number and strength of acid sites and an altered accessible pore size.  In \textbf{Paper XI} the ZSM--5 acidity was modified by Ca incorporation. Experimentally a reduction of the aromatics formation has been measured. To give some molecular insight, periodic DFT calculations have been performed to investigate the consequence of incorporating Ca into H--ZSM--5. CaOCaOH\ce{^+} species are formed and may show multiple structures coordinated to the framework oxygen atoms near the Al substitution. In the paper it was demonstrated that CaOCaOH\ce{^+} moieties might replace the original Br\o{}nsted acid sites and form Lewis acid sites, which were found to catalyze methanol conversion. \textbf{Paper XI} was the onset of a more elaborate study which is still going on, and in which various researchers of the CMM are involved (Dr. Kristof De Wispelaere and Simon Bailleul). The project was initiated and performed in the collaboration with the group of Prof. Jorge Gascon.
%
%
%\clearpage{\pagestyle{empty}\cleardoublepage}