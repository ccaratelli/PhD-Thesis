\graphicspath{{figures/chapter3/}}
% Header
\renewcommand\evenpagerightmark{{\scshape\small Major research results}}
\renewcommand\oddpageleftmark{{\scshape\small Chapter 3}}

%\renewcommand{\bibname}{References}
\hyphenation{}
\chapter[Major research results]%
{Major research results}
\label{ch3}
This Chapter illustrates the main research results obtained in the framework of this thesis. The main goal of this work was the study of the nature of active sites on UiO--66 and MOF--808 upon activation processes, and how the solvent and the functionalization influenced their behavior. The role of active sites on defective UiO--66 was studied for Fischer esterification of free fatty acids (FFA), and it became clear that solvent played an unexpected active and beneficial role in the reaction mechanism. Different molecular modeling techniques based on static and dynamic methods have been applied to gain insight into the interaction between active sites and reactants in the material, as has been introduced in Chapter 2. Contrary to reactions in zeolites, processes in MOFs are often performed ad mild conditions in the presence of a solvent, which adds complexity to the model. So far, solvent in MOFs had been studied only with classic approaches, but with the increase in computational power it was possible to include a full \textit{ab initio} treatment of water and methanol solvent in the UiO--66 pores and to go towards an \textit{operando} description of activation processes in MOFs. The most important scientific results will be highlighted in this Chapter. More details are to be found in the original articles, enclosed in Part II.

\section{Activation processes in zirconium--MOFs}
One of the main challenges in MOF research is the understanding of how active sites are created and how they impact the properties of the material. For this purpose, molecular modeling offers a platform that allows to study such activation processes and nature of active sites at the molecular level. In this sense, UiO--66, characterized by an exceptional stability, represents a perfect case study where different activation and PSM processes can take place without disrupting the stability of the structure. 
%The findings on this MOF can be extended to other materials that are more difficult to study. 
%This MOF represents a showcase example, as the findings can in principle be extended to other MOFs that are more difficult to study.
%%
\begin{figure}[!htbp]
%\vspace{-2cm}
	\centering
	\includegraphics[width=1.0\textwidth]{UiO-66-activation}
	\caption{Schematic representation of the UiO-66 structure with possible configurations of the bricks that give rise to coordinatively unsaturated Zr atoms. The colors indicate the coordination of the Zr atoms.}
	\label{fig:UiO-66-activation}
\end{figure}
%%
\subsection*{Missing linker defects}
In \textbf{PAPER I} we investigated the local defect topology upon removal of a linker on UiO--66 as starting point to investigate the catalytic role of defect coordinating species in the mechanism of Fischer esterification. This research was done in collaboration with the group of Dr. Francesc X. Llabr\'es i Xamena as experimental partners. When studying the catalytic behaviour of UiO--66, an essential ingredient to understand the reactivity of the inorganic SBU is the knowledge of the molecular structure of the active sites upon removal of a linker. On these sites, defect coordinating species can be adsorbed and have an impact on the catalytic properties of the MOF by introducing additional sites that can play an active role in reactions. 
%%%%%%
\begin{figure}[!htbp]
%\vspace{-2cm}
	\centering
	\includegraphics[width=1.0\textwidth]{adsorption-water}
	\caption{Coordination free energies at reaction temperature of 351 K of one, two and three water molecules at coordinatively unsaturated Zr-bricks in defective UiO--66 with respect to a water coordination free site (site R). The structure of the opposite site B corresponds with configuration 2’ with two water molecules and consistently used in all periodic calculations considered in the figure. Free energies (in black) are given in kJ/mol, and their decomposition into enthalpic $\Delta$H (blue) and entropic -T$\Delta$S (grey) contributions. Energies are resulting from periodic calculations with PBE-D3 level of theory. In each configuration Lewis acid and Brønsted sites are indicated.}
	\label{fig:adsorption-water}
\end{figure}
%%%%%%
Upon removal of one of the twelve negatively charged BDC linkers from the inorganic \ce{Zr6O4(OH)4} SBU, the positive charge left on the brick has to be compensated. Charge neutralization can be accomplished by either coordinating a negative ion such as hydroxyl group to one of the zirconium atoms, or by removing a proton from the brick. This latter case is characterized by two zirconium open metal sites, and is the type of active site that is obtained upon thermal treatment of the brick at T $>$ 423 K. It was previously accepted that these Lewis sites were responsible for the catalytic activity of UiO--66, but recent studies point towards the active role of Br\o{}nsted sites in the neighborhood of defective zirconium atoms \cite{canivet2014water, oien2014detailed, canivet2016origin, ling2016dynamic, liu2016probing, klet2016evaluation, vandichel2016water, ghosh2014water}. The presence of multiple sites makes it difficult to establish a simple structure--activity relation. In particular, it is important to understand from a mechanistic point of view how the presence of defect coordinating species may affect the catalytic activity on Zr--MOFs. 
\npar
The coordination of water species near the active sites can occur with different configuration schematically displayed in Fig. \ref{fig:adsorption-water}. In all structures the presence of possible Lewis and Br\o{}nsted sites which may play a role in reactions is highlighted. The simplest case taken as reference configuration is the unsaturated site obtained upon thermal activation, containing two $\mu_3$--oxygens bridging the zirconium atoms which may act as Br\o{}nsted sites. From this structure, the coordination of one physisorbed water molecule to one of the zirconium atoms is energetically favorable. A decrease in energy is observed when the molecule is deprotonated to the oxo atom as in configuration 1’. The chemisorption of this water molecule shields the Lewis character of the zirconium atom, but introduces an additional Br\o{}nsted site in proximity to the open metal site. 
\npar
The most stable configurations are observed in presence of 2 or 3 adsorbed water molecules on the adjacent zirconium atoms. Two physisorbed water molecules of configuration 2’ can be adsorbed to the two zirconium atoms, followed by an immediate dissociation of one of the two molecules into a hydroxyl group and a proton on the adjacent $\mu_3$--oxygen. This configuration is characterized by a free energy difference of -94.4 kJ/mol at reaction temperature of 351 K compared to the dehydrated defective site. The presence of two zirconium coordinating species has been reported by Lillerud \cite{oien2014detailed} by means of Single--crystal X-Ray diffraction (SXRD), showing that the material is most stable when all zirconium atoms are fully coordinated. When starting from the thermally activated material that contains open Lewis sites, at standard condition, water present in the atmosphere will immediately coordinate to restore the 8--fold coordination of the zirconium atoms to give structure 2. This structure is consistent with the one proposed by the group of Farha \cite{klet2016evaluation} who identified three types of protons from potentiometric titration: $\mu_3$--OH, \ce{Zr-OH} and \ce{Zr-OH2}. 
\npar
A third bridging hydroxyl species as charge neutralizing species was proposed by Yaghi \cite{trickett2015definitive} from XRD data. They propose two physisorbed water molecules bridged by an \ce{OH-} counterion that is stabilized by a hydrogen--bond interaction with the neighboring $\mu_3$--OH of the brick. However, a recent study Ling and Slater \cite{ling2016dynamic} did not succeed in finding a corresponding minimum on the PES. In this work, we observe that the $\mu_3$--OH atom immediately deprotonates in proximity of such \ce{OH-} anion (configuration 3). Similarly as in the previous case, this structure can further stabilized by a deprotonation of one of the two physisorbed water molecules, to yield configuration 3’ of Fig. \ref{fig:adsorption-water}, in agreement with a previous report by Vandichel et al. \cite{vandichel2016water}. Adsorption of reactants involved on the Fischer esterification reaction was also taken into account. Similar considerations on the deprotonation processes can be drawn when considering methanol instead of water as reported more in detail in \textbf{PAPER I} and \textbf{PAPER II}. The energies obtained clearly demonstrate that water molecules preferentially adsorb on the zirconium atoms and the only limit to the adsorption of more water molecules lies mainly in the entropic penalty. 

\subsection*{Nature of active sites for Fischer esterification}
 Among the reactions that can be catalyzed by UiO--66, Fischer esterification, shown in Fig. \ref{fig:esterification-mechanism} is an important process in the production of biodiesel, a biofuel that is obtained from renewable sources, such as oils and animal fats. UiO--66 has been shown to be a stable and reusable Lewis catalyst with high conversion rate for the reaction \cite{cirujano2015conversion, cirujano2015zirconium}. Experimental findings performed on both hydrated and dehydrated UiO--66 show that water has a beneficial role in the process, but a theoretical rationalization of the underlying causes was missing. Moreover, amino functionalization was shown to increase the reaction rate. To address these questions, the role of active sites on UiO--66 during the Fischer esterification reaction was studied in \textbf{PAPER I}. Two possible lowest activated reaction pathways were identified for the hydrated and dehydrated active site.
\begin{figure}[!htbp]
%\vspace{-2cm}
	\centering
	\includegraphics[width=1.0\textwidth]{esterification-mechanism}
	\caption{Top: Fischer esterification reaction; bottom: three different types of UiO--66 active sites.}
	\label{fig:esterification-mechanism}
\end{figure}

\begin{figure}[!htbp]
%\vspace{-2cm}
	\centering
	\includegraphics[width=1.0\textwidth]{esterification}
	\caption{Mechanism and free energy profile for the esterification of propionic acid with methanol on a hydrated and defective UiO-66 material (blue), a hydrated defective UiO-66 material with amino functionalization of the BDC linkers (red), and on a dehydrated defective UiO-66 (black). Periodic calculations at B3LYP-D3//PBE-D3 level of theory, T=351 K. R corresponds with an empty frame with one linker defect and a pool with all reactants to guarantee mass balance. In P the defective Zr-brick is coordinated with two water molecules (configuration 2’). P’ corresponds to the empty frame with the ester as final product and remaining water molecules in gas phase.}
	\label{fig:esterification}
\end{figure}
\npar
The proposed reaction mechanism is shown in Fig. \ref{fig:esterification}. From the reactant configuration, the physisorbed water molecule is displaced by the carboxylic acid that coordinates on the zirconium atom. The resulting configuration was identified after a series of static calculations probing possible geometries. In this configuration, the acid carbonyl group is bonded to the zirconium, and at the same time methanol is hydrogen bonded to the hydroxyl group coordinated to zirconium. The adsorption of the acid on the Lewis acid site gives to the carboxylic carbon a more electrophilic character, making it more prone to interact with the alcohol. The oxygen of methanol is at the same time made more nucleophilic due to the hydrogen--bonding interaction with the Br\o{}nsted basic site situated in close proximity. This favors the condensation between activated carboxylic carbon of the acid and methanol. Two TS are involved in the process, in which first a tetrahedric intermediate is formed, and then water is removed. In both, the hydroxyl group plays a role, first as proton acceptor, then as proton donor, while the acid maintains the bond with the carbonyl oxygen. The low energy barriers associated to the two TS are 28.9 and 30.6 kJ/mol at 351 K. The reaction can proceed both ways, until an equilibrium is reached. This mechanism is characterized by a dual participation of Br\o{}nsted and Lewis sites, and overlays with the experimental findings.
\npar
\begin{figure}[!htbp]
%\vspace{-2cm}
	\centering
	\includegraphics[width=1.0\textwidth]{esterification-failed}
	\caption{Transition states belonging to three cases where no suitable mechanism for Fischer esterification was found, suggesting that the reaction needs a concerted participation of Lewis and Br\o{}nsted sites}
	\label{fig:esterification-failed}
\end{figure}
Upon thermal treatment at 423 K, UiO--66 loses the adsorbed solvent molecules without compromising the structure of the inorganic SBU, contrary to the dehydroxylation with release of two water molecule that takes part at T $>$ 523 K \cite{shearer2013situ}. In principle, these open metal sites should be more catalytically active, but a decrease in catalytic activity is experimentally observed. In the proposed mechanism, two Lewis acid sites, the zirconium atoms, and one Br\o{}nsted basic site, the $\mu_3$--oxygen, play an active role in the reaction. This mechanism is characterized by three TS, in which: 1) methanol is deprotonated to the oxo atom, 2) an adduct is formed between the electrophilic carbon and the oxygen of methanol 3) the $\mu_3$--OH group deprotonates to form water. This reaction is characterized by higher activation barriers (a total barrier of 90 kJ/mol at 351 K), which explain the lower catalytic activity of the dehydrated material.
\npar
Other mechanisms that do not make use of either Lewis or Br\o{}nsted sites were investigated without success (Fig. \ref{fig:esterification-failed}), as the energy barriers were too high to be likely to occur. Both proposed mechanisms are characterized by a dual participation of Lewis and Br\o{}nsted sites that work complementary to each other. The presence of acid and basic centers within molecular distances has been shown to be essential in the performance of the catalytic reaction as they cooperate in a concerted way during the chemical transformation. Most of the previous mechanistic studies on the UiO--66 material merely focused on the Lewis acidity of the undercoordinated sites, but it has become more an more clear that the bifunctional nature of the UiO--66 catalyst will play an important role in its future applications.

\subsection*{Role of defect topology}
The adsorption of species present in the reaction environment of PSLE process was studied in the framework of \textbf{PAPER IV}. In this process, we wanted to see how defect--free and defective UiO--66 material would interact with water and methanol species. To obtain this information, we made use of a larger unit cell, to be able to represent different amount of missing linker defects. This increase in complexity reflects in the higher computational cost of modeling such structure. The defective structures taken into account in the modeling are obtained by removal of linkers from the conventional unit cell containing four inorganic \ce{Zr6(O)4(OH)4} bricks \cite{cavka2008new}. Different numbers of missing linkers were modeled, to represent the different amount of defects in the experimental samples. A case with a low amount of defects is modeled by a unit cell with one missing linker. In this structure, two bricks are 12--fold coordinated, and two bricks are 11--fold coordinated, with an average of 11.5 linkers per brick. A second type of unit cell was taken with three missing linkers, corresponding to a higher amount of defects, with an average of 10.5 linkers per brick. De Vos et al. and Rogge et al.\cite{devos2017missing, rogge2016thermodynamic} shown in their comprehensive studies that there are multiple topologically diverse possibilities to remove linkers from a 4--brick unit cell. In this work, we chose two possible, discrete cases to represent different coordination of the inorganic brick with this amount of defects. In a first unit cell with three missing linkers, two bricks are 10--fold coordinated and two are 11--fold coordinated. The other represents a more extreme case, with one 9--fold coordinated brick and three 11--fold coordinated bricks. The structures taken into account are denoted as $\mathrm{(10_{b}, 10_{b}, 11, 11)_{334}}$ and $\mathrm{(9_{c}, 11, 11, 11)_{333}}$ in the work of De Vos \textit{et al}\cite{devos2017missing}. 
%
\begin{figure}[H]
%\vspace{-2cm}
	\centering
	\includegraphics[width=1.0\textwidth]{psle-capping}
	\caption{Energy diagrams for defective UiO--66 unit cells. Each dot represents a possible distribution of missing-linker defects (1 or 3 in total) within the unit cell; the connecting dotted line represents a weighted average. Values are normalized by the number of missing linkers in the unit cell. (a) Enthalpy difference between defect sites capped in different ways versus the non-defective material at T = 298, 313, 373, 473 K. (b) Temperature-dependence of the free energy difference of the defective structures indicated above (capped with \ce{H2O/OH-}, \ce{H2O/MeO-} and \ce{MeOH/MeO-}) versus the non-defective structure. A representation of the clusters with different missing linker connectivities is also provided.}
	\label{fig:psle-capping}
\end{figure}
\npar
The zirconium atoms on the defect sites have been capped by all species which can be present in the framework during the activation process, studying different combinations by means of water, hydroxo species, methanol, methoxide and formate. In our models, all zirconium atoms have been capped and are fully (8--fold) coordinated, being the most stable state at experimental conditions. On each defect site there is always a negatively charged species, to compensate for the removal of a carboxylate. Given the symmetry of the unit cell, when possible, multiple choices for the positioning of these species were taken into account and modeled. The removal of a negative charge can be also compensated by removal of a proton, as obtained in the dehydrated material, with subsequent physisorption of two neutral species, but these configurations are higher in energy, as seen in \textbf{PAPER I}. These results show a clear enthalpic preference for the methanol/methoxide pair in detriment of formate and water. The defective material synthesized with formate modulator is promptly attacked by methanol species present in solution. The obtained results are in line with a previous report \cite{yang2016tuning} and highlight the preference of zirconium active sites for \ce{MeOH}/\ce{MeO-} substitution as opposed to \ce{H2O}/\ce{OH-}, alternatively assumed as preferential charge balancing element for missing linker defects \cite{trickett2015definitive, ling2016dynamic}. No substantial difference in enthalpy is observed between the unit cells with different amount of defects (energy in each defective structure represented by a dot in Fig. \ref{fig:psle-capping}). This is an indication that the thermodynamics that governs the process does not depend on the starting number of missing linkers, in agreement with the experimental observations of \textbf{PAPER IV}. The effect of temperature on the free energy was also investigated on the relative stability of defective configurations with respect to the pristine unit cell, where the linker is not missing. The first observation is that configurations involving water and methanol are more stable at lower temperature, thus the formation of missing linker defect through ligand exchange is enthalpically driven and favored by lower temperatures. Secondly, at 473 K the energy of the non-defective framework is much lower than the defective cases, which explains why the synthesis of the non-defective UiO--66 is favorable at such high temperature\cite{shearer2014tuned}.

\subsection*{Activation by dehydration}
The other process that can lead to activation of the UiO--66 material is the reversible dehydration of the brick performed at T $>$ 523 K. The dehydration mechanism may have a decisive effect on certain catalytic reactions, where next to the Lewis acid site also the neighboring Br\o{}nsted base or acid site may take a cooperative role in the reaction mechanism. At elevated temperatures and low pressures, the zirconium core gets dehydrated and rearranged, and two water molecules are subsequently removed from the brick, as displayed in Fig. \ref{fig:UiO-66-activation}. The fully dehydrated \ce{Zr6O6} brick contains undercoordinated sites \cite{valenzano2011disclosing, decoste2013stability, shearer2013situ, vandichel2015active}, with coordination of the zirconium atoms ranging from 8 to 6. The structure of UiO--66 is however preserved and the brick can easily be hydrated again with a reversible mechanism, giving evidence that the inorganic SBU can undergo dynamic processes. During these structural rearrangements, the presence of a infrared band related to hydroxy groups was observed by Nishida \textit{et al.} \cite{nishida2014structural} and by Shearer \textit{et al.} \cite{shearer2013situ}. The mechanism of dehydration was first studied by Vandichel et al. by means of nudge elastic band calculations \cite{vandichel2016water}, showing the presence of loose hydroxyl groups and partially decoordinated linkers. These findings point towards an intrinsic mobility of the framework structure, that can accommodate rearrangements without disrupting its stability. However, the static study of this process possesses serious limitations. 
\begin{figure}[!htbp]
%\vspace{-2cm}
	\centering
	\includegraphics[width=1.0\textwidth]{translation-rotation}
	\caption{Umbrella sampling in two windows of CV = 1.45 and 1.51, showing two distinct motions of the linkers. On the left column, a translation of the linker L1 generates a chelated structure and a subsequent shift in the carboxylic oxygen connected to Zr2 (configuration 4t). On the right column, a rotation of the linker L1 and a partial decoordination of linker L2 forming a hydrogen bond with an $\mu_3$--OH hydroxyl group is shown (configuration 4r). A proton transfer between the carboxylic oxygen O3 and the bridging $\mu_3$--O is also observed and is an indication of the occurrence of an intrinsic dynamic acidity. Colors indicate the coordination number of zirconium atoms.}
	\label{fig:translation-rotation}
\end{figure}
\npar
Therefore, in \textbf{PAPER III}, we follow on the fly the fast dynamic of the UiO--66 material at dehydration temperature of 573 K by means of umbrella sampling simulations. The results show an intrinsic dynamic behavior of the material, with open metal sites being created by continuous changes in the network connectivity due to labile M--L bonds. We identify two types of motions of the linkers, namely translation along the axis connecting the two adjacent zirconium atoms, and rotation along the linker axis connecting the two inorganic SBUs, as shown in Fig. \ref{fig:translation-rotation}. At the same time, these motions are accompanied by a high mobility of hydroxy groups created by decoordination of the $\mu_3$--OH groups. These results show that linkers are more mobile than originally anticipated. The high connectivity between the two SBUs of UiO--66 allows all these reversible rearrangements with activation barriers that can be easily accessible at experimental conditions. 


\subsection*{Thermal activation of MOF--808}
\begin{figure}[!htbp]
%\vspace{-2cm}
	\centering
	\includegraphics[width=1.0\textwidth]{MOF-808-dehydration2}
	\caption{Schematic representation of as synthesized and upon activation MOF-808 structures}
	\label{fig:MOF-808-dehydration2}
\end{figure}
The active sites and coordination changes upon thermal activation have been studied in MOF--808 in \textbf{PAPER VI}. MOF--808 shares the \ce{Zr6O4OH4} brick with UiO--66, but the brick is connected to only six tritopic BTC linkers, making it the least connected MOF in the Zr--MOF family. The material possesses a high catalytic potential due to the intrinsic presence of defective sites of complex nature. In the as synthesized material (Fig. \ref{fig:MOF-808-dehydration2}), the zirconium atoms that are not connected to BTC linkers are capped by formate groups. The material is activated by hot filtration, whereby each formate is replaced by water and hydroxyl group, for a total of six water molecules and six hydroxyl groups per inorganic SBU. Each pair of adjacent zirconium atoms has a similar configuration as the stable configuration 2 of Fig. \ref{fig:adsorption-water}, but the presence of multiple Br\o{}nsted sites in close proximity gives rise to a more complex nature of the active sites. By thermal activation, similarly to UiO--66, water can be removed from the active sites to open Lewis acid sites for catalysis. In a second stage, up to four of the hydroxyl groups could be also decoordinated by extracting a $\mu_3$--OH proton from the brick to form water, giving rise to mixed--coordination bricks with 6 and 7--fold coordinated zirconium atoms.
\npar
\begin{figure}[!htbp]
%\vspace{-2cm}
	\centering
	\includegraphics[width=\textwidth]{MOF-808-volume}
	\caption{Change of volume in time of the three investigated structures with different Zr coordination}
	\label{fig:MOF-808-volume}
\end{figure}
In \textbf{PAPER VI}, the behavior and stability of the material upon activation processes was investigated by means of a series of independent MD simulations at 300 K at variable unit cell parameters. We show that the dehydration of the inorganic brick, shown in Fig. \ref{fig:MOF-808-dehydration2}, substantially affects the stability of the structure. In the hydrated form, the material possesses a high number of Br\o{}nsted sites that show dynamic acidity in the form of proton transfer between water and hydroxyl groups that are located in close proximity. This may be important for proton conductivity. The physisorbed water is removed upon thermal treatment leading to a homogeneous distribution of 7--fold coordinated zirconium atoms in the brick. Upon this dehydration, only a slight decrease in the unit cell volume is observed (Fig. \ref{fig:MOF-808-volume}, pink curve), and the structure remains stable, even though it possesses a high amount of undercoordinated sites. Moreover, these Lewis sites are located in proximity to Br\o{}nsted sites arising from hydroxyl groups, making dehydrated MOF--808 a dual heterogeneous catalyst. Further removal of the hydroxyl groups together with the $\mu_3$--OH protons causes a collapse of the structure. Due to the overall low structure connectivity the adsorbed hydroxyl groups have to be considered as inherent part of the framework composition and their separation results in the collapse of the material (Fig. \ref{fig:MOF-808-volume}, yellow curve). In contrast to UiO--66, that can undergo reversible dehydration processes where the coordination of the zirconium atoms can be lowered to 6 without disrupting the structure, MOF--808 cannot sustain such decrease in coordination. The physical and chemical properties of UiO--66 cannot be easily extended to MOF--808, even though they share the same inorganic SBU. 

\subsection*{Linker functionalization on UiO--66}
The following results concern the linker functionalization on UiO--66 and its effects on the catalytic activity of the material. The ease of functionalization of UiO--66 can be exploited to tune its catalytic properties. 
\begin{figure}[!htbp]
%\vspace{-2cm}
	\centering
	\includegraphics[width=1.0\textwidth]{psle-dangling-linker}
	\caption{Proposed PSE mechanism in non-defective and defective UiO--66. MeOH facilitates ligand exchange through the creation and stabilization of defects. Enthalpy differences are given in kJ/mol at 313 K (PSE temperature)}
	\label{fig:psle-dangling-linker}
\end{figure}
\subsubsection*{Post synthetic linker exchange}
As reported in Chapter 1, functionalization in UiO--66 can be induced by PSLE, exploiting the robustness of the framework, which can easily undergo coordination changes. In \textbf{PAPER IV}, the mechanism was investigated in a dual experimental--computational study performed in collaboration with the group of Prof. Rob Ameloot. The PSLE in UiO--66 that leads to functionalization of the BDC linkers with amino groups was performed in methanol at mild conditions (313 K). Results show that the initial amount of missing linkers did not have an effect on the final composition of the material, pointing towards low energy barriers for the exchange. Moreover, the process was accompanied by an initial lowering of the BET surface area. 
\npar
In light of these experimental observations and the computational findings reported in Fig. \ref{fig:psle-capping}, a hypothetical exchange mechanism was modeled starting from both pristine and defective material (Fig. \ref{fig:psle-dangling-linker}). In the proposed mechanism, linker exchange is initiated by coordination of methanol on the zirconium sites, causing a partial hydrolysis of one of the BDC linkers. This metastable state is characterized by a dangling linker which is stabilized by hydrogen bonds with other carboxylic oxygen atoms and the $\mu_3$--OH hydrogen from the brick, in a similar fashion as what observed in \textbf{PAPER III}. This structure causes a hindrance of the pore in agreement with the reduction in BET surface area. From this configuration, exchange of mono--coordinated BDC linkers with \ce{NH2}-BDC quickly ensues, with partial coordination of the new linker. In a following step, methanol is desorbed from the active site and the linker can connect to the zirconium atoms, reestablishing the binding between the two bricks. This preferential adsorption of BDC--\ce{NH2} is explained by the lowering of the enthalpy of about 7 kJ/mol per linker. When starting from a defective material, a similar mechanism can be proposed. In this second scenario, the synthetized material contains formate on both defective sides. Formate will be substituted by methanol present in solution, and the defect healing can proceed in a similar way as in the previous case. All these rearrangements are supposedly characterized by low activation barriers, as the final composition of the material does not depend on the initial percentage of functionalized or missing linkers. During the process, the zirconium atoms rapidly change their coordination number, however, giving preference to 8-fold coordinated active sites that enhance the stability of the structure. 


\subsubsection*{Effect of linker functionalization on Fischer esterification in UiO--66}
In order to understand the effect of the functionalization of the BDC linkers on the catalytic activity of defective UiO--66, the esterification reaction (Fig. \ref{fig:esterification}) was investigated also on the amino functionalized UiO--66-\ce{NH2}. Amino functinalization on UiO--66 brings an electrodonatin group that would in principle lower the strength of the Lewis acid site. However, experimental insights into the Fischer esterification reaction showed that the amino functionalized UiO--66 was more catalytically active than its non--functionalized counterpart. For this reason, it was speculated that amino groups located in proximity to the defect sites would play an active role during the reaction. Our computational results show no possible reaction pathway that involves proton transfers with the amino groups. We therefore studied the reaction following the same pathway that we propose for the hydrated brick (Fig. \ref{fig:esterification}) in the non--functionalized material. 
\npar
\begin{figure}[!htbp]
%\vspace{-2cm}
	\centering
	\includegraphics[width=0.9\textwidth]{ester-amino}
	\caption{Network of hydrogen bonds between acid, methanol, hydroxyl group and amino groups. a) reactive complex 8, b) an additional water molecule present in solution.}
	\label{fig:ester-amino}
\end{figure}
The decrease in Lewis acidity upon amino functionalization is indeed confirmed by the increase of the Zr--O distances of the adsorbates \cite{vermoortele2012electronic}. However, for UiO--66--\ce{NH2}, stronger stabilization of the adsorbates is observed, as well as a slight decrease in the overall energy barrier of about 11 kJ/mol, which confirms the experimental findings \cite{cirujano2015conversion, cirujano2015zirconium}. The amino groups, although not playing an active role in the reaction mechanism, indirectly modulate the properties of the Lewis and Br\o{}nsted sites. A stronger adsorption of the reactants is caused by the formation of a network of hydrogen bonds with the amino groups that cannot be observed in the pristine material. Amino groups provide additional sites where solvent can form hydrogen bonds, as displayed in Fig. \ref{fig:ester-amino}. This indirect positive effect of amino groups on the catalytic properties was observed as well by Hajek et al. for aldol condensation \cite{vandichel2015active}. The effect of other functional groups was further analyzed (see Supplementary Material of \textbf{PAPER I}). Electron--withdrawing substituents such as --\ce{NO2} did not decrease the energy barriers, although increasing the Lewis acidity of the metal. Once again, this confirms the dual Lewis/Br\o{}nsted character of the UiO--66 catalyst, that can be enhanced by the presence of additional sites within molecular distance. 


\section{Role of solvent: towards operating conditions}
So far, a simple model to represent the active sites was used, where only the species immediately coordinated to the zirconium atoms were considered. A more complex model can be constructed by microsolvation, adding a small amount of solvent molecules. However, the findings of the previous section point towards a complex nature of the active sites in the material, where solvent may play an active role in reactive processes and structural modifications. 
For instance, water was shown to have a beneficial effect on the catalytic performance by providing additional Br\o{}nsted sites as well as stabilization of intermediates through hydrogen bonding. For this reason, in \textbf{PAPER II} and \textbf{PAPER V} we went a step forward in complexity by including an explicit treatment of the solvent in the pores of the UiO--66 material, as indicated in Fig. \ref{fig:uio-solvent}. This increase in complexity of the model allows to take into account the processes that can occur at operating conditions, and to represent the full solvent environment within the pores at realistic temperatures and pressures.

\begin{figure}[!htbp]
%\vspace{-2cm}
	\centering
	\includegraphics[width=1\textwidth]{uio-solvent}
	\caption{Schematic representation of solvent being inserted in the UiO--66 unit cell.}
	\label{fig:uio-solvent}
\end{figure}

\subsection*{Interaction between UiO--66 and solvent}
To understand how a protic solvent such as water and methanol interacts within the material, three independent \textit{ab initio} MD simulations were performed in which a full loading of methanol and two different loadings of water are included in the pores, as well as for the empty material. Molecular simulations allow to investigate how the material is influenced by the presence of the solvent and how the solvent behaves when it is confined in the pores. For the sake of the analysis of these properties, the system can be divided in two components: material and solvent. This way, the properties of the solvated material can be compared to the empty one, and the properties of the confined solvent can be compared to the bulk solvent. Fig. \ref{fig:vibrational} displays the vibrational density of states obtained from the power spectrum of the velocity autocorrelation function for the atoms of material and solvent. 
\npar
\begin{figure}[!htbp]
%\vspace{-2cm}
	\centering
	\includegraphics[width=1\textwidth]{vibrational}
	\caption{Top: schematic representation of the empty pore, pore with the solvent, and confined solvent without the material. Bottom: vibrational density of states obtained from the velocity autocorrelation function power spectra of selected atoms of the simulation. Bottom left: solvated material compared to the empty material. Bottom right: water in the pores compared to bulk water.}
	\label{fig:vibrational}
\end{figure}
\begin{figure}[!htbp]
%\vspace{-2cm}
	\centering
	\includegraphics[width=1\textwidth]{gr-water}
	\caption{Radial distribution functions or pair correlation functions g(r) between oxygen and hydrogen of water (O(w), H(w)), and different atoms of the material obtained from the simulation with 80 water molecules in the unit cell. Full lines indicate the g(r), dashed lines indicate its integral. Left panel: RDFs between water and linker carbons and hydrogens (C(l), H(l)); middle panel: RDFs between water and oxygen atoms of the linkers and bricks (O(l), µ3-OH, µ3-O); right panel: RDFs between water and zirconium atoms of defective and pristine bricks (Zr(def), Zr(pris)). }
	\label{fig:gr-water}
\end{figure}
The spectrum of the UiO--66 atoms can under 3000 cm$^{-1}$ shows no shifts in the peaks between empty and solvated pores, giving insight on the stability of the framework in protic solvents. For the OH stretching at 3750 cm$^{-1}$, however, there is a broadening when solvent is included. This is due to the strong interactions between $\mu_3$--OH hydrogens of the bricks and oxygens of solvent, that form hydrogen bonds. To further investigate this, we analysed the vibrational density of states for the solvent and compared it to the bulk solvent simulated in the same unit cell and at the same conditions. We can notice the appearance of a peak at $\nu >$ 3700 cm$^{-1}$ in the confined solvent, which is due to O--H bonds that do not form hydrogen bonds. This peak is due to the water molecules whose hydrogen atoms are pointed towards the linkers and do not interact with them, in line with previous reports on hydrophobic confinement \cite{coudert2006dipole, dalla2016water, cicero2008water}. 
\npar
These results point towards a dual hydrophobic--hydrophilic interaction between solvent and material, where on the one hand, the solvent experiences a hydrophobic confinement due to the interaction with the linkers, on the other it binds strongly to the $\mu_3$--OH hydrogens of the bricks. To further gain insight into this behavior, we analyzed the RDFs for specific pairs of atoms of material and solvent. The results in case of water are displayed in Fig. \ref{fig:gr-water}. The RDF between carbon atoms of the linkers and water clearly shows a hydrophobic confinement behavior, being nearly zero at distances below 3 \AA. On the other hand, a strong interaction is observed between water and oxygen atoms belonging to bricks and linkers, showing formation of a network of hydrogen bonds. The coordination between zirconium atoms and water is also analyzed. In the defective brick, the first peak is due to the coordination between water and zirconium on the defect site. 
\npar
From this data, we can conclude that solvent does not leave the active site during the simulation time, as the interaction is rather strong, in agreement with the previous static calculations and dynamic results on small models. These simulations give indication of the changes in behavior of the solvent upon confinement and the strong interactions around the bricks and in particular on the active site. However, such analysis gives only an insight on the average behavior and not on the dynamic processes that can occur, which will be examined in the next two sections. 


\subsection*{Interaction between methanol molecules on the UiO--66 defective site}
In \textbf{PAPER II}, we used a multilevel modeling approach to analysed the behaviour of methanol solvent around the active site generated by linker removal. We observe a breakage of the symmetry for the two active sites generated by linker removal. The site denoted as Site A shows a trigonal network, similar to what reported in static calculations where three molecules were taken into account. This configuration is not broken during the simulation, giving evidence of its stability. Moreover, proton transfers in a similar fashion to what was reported by Ling and Slater \cite{ling2016dynamic} are observed between hydroxyl group and physisorbed water, with the bridging methanol molecule shuttling the proton from one site to the other (Fig. \ref{fig:proton-shuttle}). No proton transfers are observed from the $\mu_3$--OH proton to the hydrogen--bonded methanol molecule, as this configuration is higher activated. Site B, however, shows a more complex evolution, with chains of hydrogen bonds connecting up to 5--6 methanol molecules and forming closed loops. This evolution is reported in Fig. \ref{fig:methanol-topology}, where the number of molecules involved in these loops is followed during the simulation time. The system alternates between 4 to 6--membered rings that are anchored on the defect--coordinating water species and the $\mu_3$--OH group. 
\begin{figure}[!htbp]
%\vspace{-2cm}
	\centering
	\includegraphics[width=0.5\textwidth]{proton-shuttle}
	\caption{Dynamic Br\o{}nsted acidity in one of the structures established on the active site in defective UiO--66 and liquid methanol in the pores.}
	\label{fig:proton-shuttle}
\end{figure}
\npar
The behaviour is similar to what observed for bulk methanol\cite{kashtanov2005chemical, pagliai2003hydrogen, morrone2002ab}, in which a mixture of chain and ring structures formed by six to eight methanol molecules was observed. In this case, the chains are shorter because of confinement, nevertheless, the pore size of defective UiO--66 allows these structures to form. Moreover, open chains of methanol molecules are observed that for part of the simulation connect the two active sites and can in principle provide a way to transfer protons from one site to the other. However, during the simulation time this event is not observed due to the barrier associated to the charge separation. 
\begin{figure}[!htbp]
%\vspace{-2cm}
	\centering
	\includegraphics[width=1\textwidth]{methanol-topology}
	\caption{Top: Ring configurations observed at site A and site B originating from the interaction between the Zr-bonded hydroxo and water and the solvent molecules. Bottom: Appearance of the various structures during the simulation. The frequency of occurrence of the different structures is also reported. A threshold of 2.2 Å for the donor-acceptor distance was chosen to determine a hydrogen bond and observations were smoothed over 0.5 ps.}
	\label{fig:methanol-topology}
\end{figure}
\npar
Proton mobility in the pores of the material was further analyzed. A charge displacement was induced by artificially removing a proton from the active site and inserting it in different positions in the pore of the material (Fig. \ref{fig:proton-transfer}) and the system response was followed. In the methanol solvent, the proton can be either stabilized by solvent molecules and maintain its position in the unit cell, or a Grotthuss charge transfer mechanism can occur in which the charge defect travels through a chain of hydrogen bonds, similarly as what observed by Morrone et al. \cite{morrone2002ab}. In one of the simulations, the proton is transported towards the other active site (Fig. \ref{fig:proton-transfer}). A configuration is obtained where the original site remains deprotonated and the other is protonated. This configuration is retained for the whole remaining simulation time and is stabilized by the presence of solvent molecules. The zirconium--oxygen distance in the protonated case increases, and the two water molecules physisorbed on the site are more mobile and prone to leave the active site, and a proton transfer could be the initiator to reactant exchanges on the active site. Static periodic calculations were performed to investigate the energy difference related to this charge separation which report a value of 89.4 kJ/mol difference in free energy. Such difference between static and dynamic picture gives indication of the positive role of a protic solvent in stabilizing charged configurations that can be reflected in the stabilization of charged intermediates during catalytic processes. These simulations shed light on the role of Br\o{}nsted sites in the material. The $\mu_3$--OH group is heavily involved in the stabilization of supramolecular structures around the active site, but does not deprotonate to the solvent, contrary to water and hydroxyl groups that show dynamic acidic behavior. These findings show the importance of a solvent beyond being a substrate in the reaction, as it can exchange protons, affect reaction mechanisms, and stabilize intermediates through its remarkable interactions with the active sites. In the next section we will see how solvent can be exchanged on the active sites and induce structural rearrangements in the material.

\begin{figure}[!htbp]
%\vspace{-2cm}
	\centering
	\includegraphics[width=1\textwidth]{proton-transfer}
	\caption{Three snapshots of the molecular dynamics simulation which starts from a deprotonated site A and a protonated methanol molecule, with corresponding schematic representation of the process (above). 1) starting structure with protonated solvent 2) a snake-like chain of hydrogen bonds is formed which leads a proton to site B 3) site B is protonated, while site A is missing a proton.}
	\label{fig:proton-transfer}
\end{figure}

\subsection*{Activated processes related to dynamic changes in the zirconium coordination number}
In \textbf{PAPER V}, we focus on the dynamic changes in coordination of the zirconium atoms on the defective sites. Previous regular MD simulations do not show any change in the zirconium coordination, as such processes are rare events characterized by an activation barrier that does not allow their sampling at normal conditions. However, Zr--O bond breaking is a fundamental event in many processes, such as in the exchange of solvent around the active sites, dehydration, defect formation or PSLE. For this reason, we made use of enhanced sampling by means of a series of independent MTD simulations to investigate the coordination changes around the defective zirconium atoms in water solvent. 
\npar
\begin{figure}[!htbp]
%\vspace{-2cm}
	\centering
	\includegraphics[width=\textwidth]{CVdef}
	\caption{Coordination numbers used in the simulation a: coordination number CN$_W$ between zirconium and all water oxygens. Also the linker that shows dynamic movement which induces changes in the zirconium coordination number is visualized. b: coordination number CNL between each zirconium atom  and linker oxygen atoms. n$_{OW}$ and n$_{OL}$ are the number of oxygen atoms considered in the two cases, $r_i$ is the zirconium-oxygen distance, $r_0$ a cutoff distance of 2.9 \AA. In yellow, the zirconium atoms considered in the CN. In green, the oxygen atoms that have a weight close to one and substantially different from zero in the summation.}
	\label{fig:CVdef}
\end{figure}
\begin{figure}[!htbp]
%\vspace{-2cm}
	\centering
	\includegraphics[width=1\textwidth]{coordination-scheme}
	\caption{ Coordination and value of the CV during (a) the exchange of solvent and (b) linker decoordination in the MTD simulations. top paths: stepwise pathway which goes through undercoordinated zirconium; bottom paths: concerted pathways which go through overcoordinated zirconium.}
	\label{fig:coordination-scheme}
\end{figure}
Two CVs were chosen, as displayed in Fig.\ref{fig:CVdef}: a first one, denoted as CN$_W$, representing the coordination between zirconium and oxygen atoms of water, and a second one, denoted as CN$_L$, describing the coordination between zirconium and oxygen atoms of a defect--bridging linker. Both CVs refer to the difference between the coordination state and its equilibrium state. Therefore, starting from an equilibrium value of zero, when a bond is broken, the coordination will decrease by one, or increase by one if an additional bond is formed. Using these two coordination numbers, it is possible to describe many events taking place around the brick, which are schematically reported in Fig. \ref{fig:coordination-scheme}. In the equilibrated structure, the state defined by the two CNs is (0,0), where the numbers refer to the values of CN$_W$ and CN$_L$, respectively. Undercoordination and creation of open Lewis acid sites, which can occur by desorption of water of by breakage of Zr--linker bond, can this way be followed. However, on the defective site the system can display also overcoordination, when an additional water molecules coordinates to the zirconium atom increasing its coordination number to a value of 9. 
\npar
\begin{figure}[!htbp]
%\vspace{-2cm}
	\centering
	\includegraphics[width=1\textwidth]{MTD-water1}
	\caption{Top: evolution of the collective variable CN$_W$ and free energy profile for the first MTD simulation along the CV representing the coordination between the zirconium atom Zr1, highlighted in yellow, and all water molecules (green curve). CN$_L$ between Zr1 and the neighbouring defect-bridging linker oxygen (red curve) is also monitored; bottom: snapshots on the linker decoordination triggered by overcoordination}
	\label{fig:MTD-water1}
\end{figure}
In a first case study, the coordination CN$_W$ between one of the defective zirconium atom and the water oxygens was biased. At the same time the CN$_L$ between the same zirconium atom and the oxygen belonging to the defect bridging linker is monitored, as displayed in Fig.\ref{fig:MTD-water1}. In this simulation, fluctuations of the coordination towards negative values are observed, where the open Lewis acid site is created. Via this undercoordinated pathway, solvent molecules can be exchange on the active site in a stepwise fashion. After this process has been sampled, the system evolves towards an overcoordination of the zirconium atom, to which an additional water molecule is coordinated. From this metastable configuration, the system evolves with structural rearrangements which are reported in the bottom of Fig.\ref{fig:MTD-water1}. After the second water molecule has entered the coordination sphere of zirconium, it transfers a proton to the neighboring hydroxyl group and pushes the parent water molecule closer to the linker. The water molecule forms a hydrogen bond interaction with the linker which is in turn decoordinated from the zirconium atom, restoring the equilibrium coordination of 8 in the zirconium atom. After these events, the linker recoordinates to the zirconium, to decoordinate again few ps later. In the simulation, we observe a plethora of events such as undercoordination and overcoordination of zirconium, proton transfers, exchange of solvent molecules, and reversible linker decoordination. These fluctuations show how the role of solvent is crucial in inducing such coordination changes and in stabilizing partially decoordinated linkers that can lead to structural rearrangements, while maintaining the stability of the material. 
\npar
\begin{figure}[!htbp]
%\vspace{-2cm}
	\centering
	\includegraphics[width=1\textwidth]{MTD-water2}
	\caption{Top: free energy profile for the first MTD simulation along the CV representing the coordination between two zirconium atoms and a linker, middle: evolution of the collective variable, bottom: snapshots of the linker decoordination}
	\label{fig:MTD-water2}
\end{figure}
In a second case study, we biased the coordination CN$_L$ between one linker and two defective zirconium atoms, and at the same time monitored the coordination CN$_W$ between the latter atoms and solvent water. In this case, we are directly biasing the coordination between zirconium and the linker. The evolution of the collective variable is displayed in Fig. \ref{fig:MTD-water2}, as well as snapshots that offer mechanistic insights on the different processes. During the simulation, different states are explored, in which the linker is partially decoordinated, can translate to give chelated structures, similarly to what observed in \textbf{PAPER III}, or be in a dangling state where both Zr--O bonds are broken, giving an indication of what postulated in \textbf{PAPER IV}. In this case, the linker is decoordinated without intermediate role of solvent, and represents an alternate mechanism by which PSLE, defect creation, or even hydrolysis of the material could occur. This dangling linker scenario is stabilized by hydrogen bonding interactions with the solvent. The process would be completed if two water molecules would diffuse to the zirconium atom and restore the 8--fold coordination, while at the same time one of the two deprotonates to the linker. However this is not observed during the simulation time, as water does not have the time to diffuse, while hindered by the motions of the linker.
\npar
By means of this series of simulations, we conclude that both solvent exchange around the active sites and structural rearrangements can either occur via a stepwise mechanism where zirconium atoms are undercoordinated or a concerted one, mediated by an overcoordination. The mobility of the linkers observed in \textbf{PAPER III} is herein observed without rearrangements of the bricks that are induced at dehydration conditions. In the presence of a solvent, such processes can undergo at much milder conditions, without disrupting the stability of the material. The role of solvent in allowing these processes is crucial, as it provides hydrogen bonds that can stabilized charged configurations, and can itself strongly interact with the zirconium atoms.

\clearpage{\pagestyle{empty}\cleardoublepage}